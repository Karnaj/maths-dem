\documentclass[fontsize=12pt,twoside=false,parskip=half, french]{scrartcl}
\usepackage[utf8]{inputenc}
\usepackage{babel}
\usepackage[T1]{fontenc}
\usepackage{amsmath, amssymb, stmaryrd}
\usepackage[amsmath]{ntheorem}

\title{Non dénombrabilité des parties de $\N$}
\date{}
\author{}

\input{../outils}
\setcounter{secnumdepth}{0}
\begin{document}
\maketitle
   \begin{Theoreme}
      L’ensemble des parties de $N$ n’est pas dénombrabre.
   \end{Theoreme}
   \subsection{Démonstration}
      Pour montrer ce résultat, supposons cet ensemble dénombrabre. Alors il existe une bijection
      $\phi$ de $\N$ dans $\partie(\N)$. Prenons alors l’ensemble
      \[
         A = \enstq{n \in \N}{n \not \in \varphi(n)}.
      \]      
      $A$ est une partie de $\N$, il existe un entier $n$ tel que $\phi(n) = A$. On a alors deux possibilités.
      \begin{description}
         \item Si $n \in A$, alors $n \not \in \phi(n)$, ABSURDE puisque $A = \phi(n)$.
         \item Si $n \not \in A$, alors $n \not \in \phi(n)$, et donc $n \in A$, également ABSURDE.
      \end{description}
      Par suite, $\partie(\N)$ n’est pas dénombrabre.
\end{document}