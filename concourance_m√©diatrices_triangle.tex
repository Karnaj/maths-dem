\documentclass[fontsize=12pt,twoside=false,parskip=half,french]{scrartcl}
\usepackage[utf8]{inputenc}
\usepackage{babel}
\usepackage[T1]{fontenc}
\usepackage{amsmath, amssymb, stmaryrd}
\usepackage[amsmath]{ntheorem}

\title{Concourance des médiatrices d'un triangle}
\date{}
\author{}

\input{outils}
\setcounter{secnumdepth}{0}
\begin{document}
\maketitle
   \begin{Theoreme}[Concourance des médiatrices d'un triangle]
      Les médiatrices d'un triangle sont concourantes.
   \end{Theoreme}
   \subsection{Démonstration}
      Soit $ABC$ un triangle et $O$ le point d'intersection des médiatrices de
      $[AB]$ et $[BC]$ ($O$ existe car le triangle est non dégénéré). On a 
      que $AO = BO$ (car $O$ est sur la médiatrice de $[AB]$) et $BO = CO$
      (car $O$ est sur la médiatrice de $[BC]$) d'où $BO = CO$, c'est-à-dire
      que $O$ est sur la médiatrice de $AC$. 
\end{document}