\documentclass[fontsize=12pt,twoside=false,parskip=half, french]{scrartcl}
\usepackage[utf8]{inputenc}
\usepackage{babel}
\usepackage[T1]{fontenc}
\usepackage{amsmath, amssymb, stmaryrd}
\usepackage[amsmath]{ntheorem}

\title{Sous-groupes de $\varZquotient{n}$}
\date{}
\author{}

\input{../outils}
\setcounter{secnumdepth}{0}
\begin{document}
\maketitle
   \begin{Theoreme}
      Il y a autant de sous-groupes dans $\varZquotient{n}$ que de diviseurs de $n$.
   \end{Theoreme}
   \subsection{Démonstration}
      On a déjà que si $H$ est un sous-groupe de $\varZquotient{n}$, alors l’ordre
      de $d$ divise $n$. Montrons que si $d$ divise $n$, alors $\varZquotient{n}$ 
      possède exactement un sous-groupe d’ordre $d$. Ceci donnera bien le résulatat.
      On a qu'il existe $d' \in \N$, $n = dd'$.
  
      $\langle \classe{d’} \rangle$ est un sous-groupe de $\varZquotient{n}$
      d’ordre $d$ donc il existe un sous-groupe d'ordre $d$.
      
      Inversement, si $H$ est un sous-groupe à $d$ éléments, alors, pour tout 
      $\classe{x}$ dans $H$, $d\classe{x} = 0$ (puisque l’ordre d’un élément
      divise l’ordre d’un groupe), d’où $n$ divise $dx$ avec $n = d’d$, donc
      $d’$ divise $x$. On a alors,
      \[
         x \in \enstq{k\classe{d’}}{ 0 \leq k < d} = \langle d’ \rangle.
      \]
      Donc $H \subset \langle d’ \rangle$, et on a égalité par cardinalité.
      
      PS : Par isomorphisme, on a ce résultat pour tout groupe cyclique d’ordre $n$.
\end{document}