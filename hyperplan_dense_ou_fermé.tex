\documentclass[fontsize=12pt,twoside=false,parskip=half]{scrartcl}
\usepackage[utf8]{inputenc}
\usepackage[french]{babel}
\usepackage[T1]{fontenc}
\usepackage{amsmath, amssymb, stmaryrd}
\usepackage[amsmath]{ntheorem}

\title{Topologie des hyperplans}
\date{}
\author{}

\input{../outils_maths}
\setcounter{secnumdepth}{0}
\begin{document}
\maketitle
   \begin{Theoreme}
      Soit $(E, \norme{.})$ un EVN. Les hyperplans de $E$ sont fermés ou denses dans $E$.
   \end{Theoreme}
   \subsection{Démonstration}
      Commençons par montrer que si $H$ est un sous-espace vectoriel de $E$, alors
      $\adh{H}$ en est un (c’est un résultat classique).
      
      Soient $(u, v) \in \adh{H}^2$ et $\lambda \in \R$. Il esiste $(u_n)$ et $(v_n)$
      deux suites de $H$ telle que $u_n \to u$ et $v_n \to v$. On a 
      
      On a $\lambda u_n + v_n \to \lambda u + v$ et $\lambda u_n + v_n \in H$, d’où
      \[
         \lambda u + v \in \adh{H}.
      \]
      Donc $\adh{H}$ est un SEV.
      
      Soit maintenant $H$ un hyperplan de $E$. Par définition, $H$ est de codimension 1.
      Puisque, $H \subset \adh{H}$, alors $\adh{H}$ est de codimension 1 ou 0.
      \begin{description}
         \item S’il est de codimension 0, alors $\adh{H} = E$, et donc 
               $H$ dense dans $E$.
         \item Sinon, on a $H \subset \adh{H}$, or ils ont même codimension, donc $H = \adh{H}$ et $H$ est fermé.
      \end{description}
      Le résultat est donc démontré. 
\end{document}