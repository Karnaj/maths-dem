\documentclass[fontsize=12pt,twoside=false,parskip=half]{scrartcl}
\usepackage[utf8]{inputenc}
\usepackage[french]{babel}
\usepackage[T1]{fontenc}
\usepackage{amsmath, amssymb, stmaryrd}
\usepackage[amsmath]{ntheorem}

\title{Formule de la moyenne}
\date{}
\author{}

\input{../outils_maths}
\setcounter{secnumdepth}{0}
\begin{document}
\maketitle
   \begin{Theoreme}[Formule de la moyenne]
      Soient $f$ et $g$ continues de $\interv[a; b]$ dans $\R$ avec $g \geq 0$. Alors il existe $c \in \interv[a; b]$
      tel que
      \[
         \int_a^b f(t)g(t) \dt = f(c) \int_a^b g(t) \dt
      \]
   \end{Theoreme}
   \subsection{Démonstration}
      $f$ est continue sur l’intervalle fermé $\interv[a; b]$, donc (\emph{théorème des bornes}) son image est un 
      intervalle fermé $\interv[m; M]$. On a alors
      \[
         m\int_a^b f(t)\dt \leq \int_a^b f(t)g(t) \dt \leq M\int_a^b g(t) \dt.
      \]
      Donc, il existe $y \in \interv[m; M]$, tel que
      \[
         \int_a^b f(t)g(t) \dt = y\int_a^b g(t) \dt.
      \]
      Le \emph{théorème des valeurs intermédiaires} donne l’existence de $c \in \interv[a; b]$ tel que $y = f(c)$.
      On a bien le résultat voulu.
      
      Notons que le résultat s’applique si $g$ est de signe constant (quitte à changer $g$ par $-g$).
\end{document}