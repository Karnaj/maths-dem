\documentclass[fontsize=12pt,twoside=false,parskip=half, french]{scrartcl}
\usepackage[utf8]{inputenc}
\usepackage{babel}
\usepackage[T1]{fontenc}
\usepackage{amsmath, amssymb, stmaryrd}
\usepackage[amsmath]{ntheorem}

\title{Théorème de Liouville}
\date{}
\author{}

\input{../outils}
\setcounter{secnumdepth}{0}
\begin{document}
\maketitle
   \begin{Theoreme}
      Soit un entier $p > 5$. Alors l’équation
      \[
         (p - 1)! + 1 = p^m
      \]
      n’a aucune solution dans $N^*$.
   \end{Theoreme}
   \subsection{Démonstration}
      Puisque $p > 5$, alors $(p - 1)! + 1$ est un nombre impaire, donc $p^m$ est impair c’est-à-dire 
      que $p$ est impair. Et puisque $p > 5$, on obtient $4 < p - 1 < 2(p - 1)$, d’où
      \[
         (p - 1)^2 = 2 \times \frac{p - 1}{2} \times (p - 1) \text{ divise } (p - 1)!
      \]
      Supposons alors que $(p - 1)! + 1 = p^m$. Puisque, $(p - 1)! = p^m - 1$,
      alors $(p - 1)^2$ divise $p^m - 1 = (p - 1)(1 + p + \ldots + p^{m - 1})$, d’où
      \[
         (p - 1) \text{ divise } S = \sum_{k = 1}^{m - 1} p^k. 
      \]
      Mais, $p \equiv 1 \mod p - 1$, d’où $S \equiv m \mod p - 1$, ce qui prouve que
     $p - 1$ divise $m$. Donc $m \geq p - 1$, d’om
     \[
      p^m \geq p^{p - 1} \geq (p - 1)^{p - 1} \geq (p - 1)!.
     \]
     Par suite, $(p - 1)! + 1 < p^m$, d’où l’équation n’a pas de solution.
\end{document}