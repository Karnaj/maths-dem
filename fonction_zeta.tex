\documentclass[fontsize=12pt,twoside=false,parskip=half, french]{scrartcl}
\usepackage[utf8]{inputenc}
\usepackage{babel}
\usepackage[T1]{fontenc}
\usepackage{amsmath, amssymb, stmaryrd}
\usepackage[amsmath]{ntheorem}

\title{Fonction zéta de Riemann}
\date{}
\author{}

\input{../outils}
\setcounter{secnumdepth}{0}
\begin{document}
\maketitle
    On définit la fonction zéta de Riemann en posant pour $x \in I = \interv]1; +\infty[$
    \[
      \zeta(x) = \sum_{n = 1}^{+\infty} \frac{1}{n^x}.
    \]
   \begin{Theoreme}[Continuité de la fonction zéta]
      La fonction $\zeta$ est de classe $C^\infty$ sur $\interv]1; +\infty[$.
   \end{Theoreme}
   \subsection{Démonstration}
      La fonction zéta est bien définie sur $\interv]1; +\infty[$ (\emph{série de Riemann}). Posons
      \[
         f_n(x) = \frac{1}{n^x} = \e^{-x \ln x}.
      \]
      Les $f_n$ sont de classe $C^\infty$ sur $I$ et on montre par récurrence que pour tout entier $k$, 
      \[
         f_n^{(k)}(x) = \frac{(- \ln n)^k}{n^x}.
      \]
      Soit $\interv[a; b] \subset I$. On a pour $x \in \interv[a; b]$,
      \[
         \abs{f_n^{(k)}(x)} \leq \frac{(- \ln n)^k}{n^a}.
      \]
      Considérons $\rho \in \interv]1; a[$, on a
      \[
         \frac{(- \ln n)^k}{n^a}  = o\left(\frac{1}{n^\rho}\right)
      \]
      avec la série des $\frac{1}{n^\rho}$ qui converge (\emph{série de Riemann} avec $\rho > 1$). Par 
      \emph{comparaison de séries à termes positifs}, la série des $u_n^{(k)}$ converge normalement, donc uniformément
      sur tout segment de $I$, ce qui montre que $\zeta$ est de classe $C^\infty$ sur $\interv]1; +\infty[$ avec 
      \[
         \zeta^{(k)}(x) = \sum_{n = 1}^{+\infty} \frac{(- \ln n)^k}{n^x}.
      \]
      
      PS : ceci permet de montrer que $\zeta$ est décroissante et convexe. On montre en utilisant le théorème de la double
      limite (il y a aussi convergence uniforme en $+\infty$) que $\zeta$ tend vers $1$ en $+\infty$.
\end{document}