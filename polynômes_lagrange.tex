\documentclass[fontsize=12pt,twoside=false,parskip=half, french]{scrartcl}
\usepackage[utf8]{inputenc}
\usepackage{babel}
\usepackage[T1]{fontenc}
\usepackage{amsmath, amssymb, stmaryrd}
\usepackage[amsmath]{ntheorem}

\title{Polynômes de Lagrange}
\date{}
\author{}

\input{../outils}
\setcounter{secnumdepth}{0}
\begin{document}
\maketitle
   \begin{Theoreme}
      Soit $n \in \N$, et $x_0, \ldots, x_n$ $n + 1$ complexes deux à deux distincts.
      Il existe $(n + 1)$ polynômes uniques $L_0, \ldots, L_n$ tel que 
      $\forall (i, j) \in \intervalleEntier{1}{n}^2, L_i(x_j) = \delta_{i, j}$.
      Ils forment une base de $\C_n[X]$.
   \end{Theoreme}
   \subsection{Démonstration}
      Soit $i \in \intervalleEntier{1}{n}$. $L_i$ est de degré au plus $n$ et on connaît $n$ racines, donc il existe $\lambda_i$ tel que
      \[
         L_i = \lambda_i\prod_{j \neq i} (X - x_j).
      \]
      Et l’égalité $L_i(x_i) = 1$ permet de trouver la valeurs de $\lambda_i$. On
      trouve
      \[
         L_i = \prod_{j \neq i} \frac{(X - x_j)}{x_i - x_j}.
      \]
      Réciproquement, un tel $L_i$ est bien défini puisque les $x_i$ sont 
      distincts, est de degré inférieur ou égal à $n$ et vérifie bien 
      $L_i(x_j) = \delta_{i, j}$. D’où l’existence et l’unicité.
      
      Montrons qu’il s’agit d’une base. La famille $(L_k)$ a le bon cardinal,
      montrons qu’elle est libre. Soit $(\lambda_0, \ldots, \lambda_n) \in \C^{n + 1}$.
      \begin{align*}
         \sum_{i = 0}^n \lambda_i L_i = 0 
              & \implies \forall 0 < j \leq n, \sum_{i = 0}^n \lambda_i L_i(j) = 0\\
              & \implies \forall 0 < j \leq n, \sum_{i = 0}^n \lambda_i \delta_{i, j} = 0\\
              & \implies \forall 0 < j \leq n, \lambda_j = 0.
      \end{align*}
      Par suite, c’est bien une base de $\C_n[X]$.
      
      PS : Ceci permet de montrer que si $(y_0, \ldots, y_n)$ sont $n + 1$ complexes,
      il existe un unique polynôme de degré au plus $n$ tel que $P(x_j) = y_j$ pour tout
      $j$. En effet, si $P$ est un tel polynôme, ses coordonnées dans la base
      des $(L_i)$ est $(y_0, \ldots, y_n)$ \ie il est entièrement déterminé.  
\end{document}