\documentclass[fontsize=12pt,twoside=false,parskip=half, french]{scrartcl}
\usepackage[utf8]{inputenc}
\usepackage{babel}
\usepackage[T1]{fontenc}
\usepackage{amsmath, amssymb, stmaryrd}
\usepackage[amsmath]{ntheorem}

\title{Inégalité de Jensen}
\date{}
\author{}

\input{../outils}
\setcounter{secnumdepth}{0}
\begin{document}
\maketitle
   \begin{Theoreme}[Inégalité de Jensen]
      Soit $f$ continue de $\interv[a; b]$ dans $\R$, et soit $g$ de $\R$ dans
      $\R$ continue et convexe. Alors
      \[
         g\left( \frac{1}{b - a} \int_a^b f(t) \dt \right) \leq
         \frac{1}{b - a} \int_a^b g(f(t)) \dt.
      \]
   \end{Theoreme}
   \subsection{Démonstration}
      Posons
      \[
         I_n = \frac{1}{n} \sum_{k = 0}^{n - 1} f\left(a + k \frac{b - a}{n}\right).
      \]
      La suite $(I_n)$ est une suite de sommes de Riemmann et donc
      \[
         \lim_{n \to \infty} I_n = \frac{1}{b - a} \int_a^b f(t) \dt.
      \]
      La convexité de $g$ donne que
      \begin{equation}
         g(I_n) \leq \frac{1}{n} \sum_{k = 0}^{n - 1} g \circ f\left(a + k \frac{b - a}{n}\right) = G_n  \label{ineq}
      \end{equation}
      Puisque $g \circ f$ est continue, on a 
      \[
         \lim_{n \to \infty} G_n = \frac{1}{b - a} \int_a^b f(f(t)) \dt.
      \]
      Le passage à la limite de \eqref{ineq} permet alors d’obtenir le résultat.
\end{document}
