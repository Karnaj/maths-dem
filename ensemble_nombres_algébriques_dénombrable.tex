\documentclass[fontsize=12pt,twoside=false,parskip=half, french]{scrartcl}
\usepackage[utf8]{inputenc}
\usepackage{babel}
\usepackage[T1]{fontenc}
\usepackage{amsmath, amssymb, stmaryrd}
\usepackage[amsmath]{ntheorem}

\title{Dénombrabilité des nombres algébriques}
\date{}
\author{}

\input{../outils}
\setcounter{secnumdepth}{0}
\begin{document}
\maketitle
   Un réel $x$ est algébrique s'il est racine d'un polynôme de $\Q[X]$.
   \begin{Theoreme}[Dénombrabilité des nombres algébriques]
      L'ensemble des nombres algébriques est dénombrable.
   \end{Theoreme}
   \subsection{Démonstration}
      %Notons qu'on peut supposer les coefficients des polynômes entiers (en 
      %multipliant tous les coefficients de $P \in \Q[X]$ par le PPCM de leurs 
      %démoninateurs, on obtient un polynôme $Q$ à coefficients entiers et 
      %$P(x) = 0$ si et seulement si $Q(x) = 0$.
      
      On note que $\Q[X]$ est dénombrable en tant qu'union dénombrable d'ensemble 
      dénombrables. En effet,
      \[
        \Q[X] = \bigcup_{n \in \N} Q_n[X]
      \]
      et $Q_n[X]$ est dénombrable ($Q_n[X]$ est isomorphe à $\Q^{n + 1}$ car
      un polynômes de $Q_n[X]$ est une suite d'éléments de $Q^{n + 1}$).
      
      Les polyômes de $\Q[X]$ sont dénombrables et l'ensemble des racines d'un
      polynôme est fini, donc l'ensemble des nombres algébriques est dénombrable
      en tant qu'union dénombrables d'ensemble fini.
      
      En effet, puisque $\Q[X]$ est dénombrable, il existe une suite de polynômes
      $P_n$ tel que $\Q[X] = \enstq{P_n}{n \in \N}$  et on a alors, en notant $E$
      l'ensemble des nombres algébrique
      \[
        E = \bigcup_{n \in \N} \enstq{x \in \R}{P_n(x) = 0}.
      \] 
      
      PS : cela assure par exemple l'existence de nombre transcendants.
\end{document}