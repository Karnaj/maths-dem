\documentclass[fontsize=12pt,twoside=false,parskip=half, french]{scrartcl}
\usepackage[utf8]{inputenc}
\usepackage{babel}
\usepackage[T1]{fontenc}
\usepackage{amsmath, amssymb, stmaryrd}
\usepackage[amsmath]{ntheorem}

\title{Polynôme caractéristique de $AB$}
\date{}
\author{}

\input{../outils}
\setcounter{secnumdepth}{0}
\begin{document}
\maketitle
   \begin{Theoreme}
      Soient $A$ et $B$ deux matrices de $M_n(R)$. $AB$ et $BA$ ont même polynôme caractéristique.  
   \end{Theoreme}
   \subsection{Démonstration}
      Montrons le résultat pour $A$ inversible. On a 
      \[
         \det(XI_n - AB) = \det A \det(XA^{-1} - B) = \det(XA^{-1} - B) \det A = \det(XI_n - BA).
      \]
      On a bien $\polca{AB} = \polca{BA}$.
      
      Si $A$ n’est pas inversible, la densité de $GL_n(\R)$ dans $M_n(\R)$ et la continuité du
      déterminant permettent d’obtenir le résultat. En effet, il existe une suite $A_n$ de matrices 
      inversibles qui a pour limite $A$. On a pour tout $n \in \N$, $\polca{A_nB} = \polca{BA_n}$.
      En passant à la limite, on obtient le résultat.
\end{document}