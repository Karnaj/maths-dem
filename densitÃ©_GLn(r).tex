\documentclass[fontsize=12pt,twoside=false,parskip=half, french]{scrartcl}
\usepackage[utf8]{inputenc}
\usepackage{babel}
\usepackage[T1]{fontenc}
\usepackage{amsmath, amssymb, stmaryrd}
\usepackage[amsmath]{ntheorem}

\title{Densité de $GL_n(\R)$ dans $M_n(\R)$}
\date{}
\author{}

\input{../outils}
\setcounter{secnumdepth}{0}
\begin{document}
\maketitle
   \begin{Theoreme}
      $GL_n(R)$ est dense dans $M_n(\R)$.
   \end{Theoreme}
   \subsection{Démonstration (limites de suite)}
      Soit $A \in M_n(R)$. Montrons qu’il existe une suite $A_n$ de matrices inversibles qui tend vers $A$.
      $A$ possède un nombre fini de valeurs propres, donc pour $n_0$ assez grand $A - \frac{1}{n_0}I_n$ est 
      inversible. Prenons alors
      \[
         A_n = A - \frac{1}{n + n_0}I_n.
      \]
      $A_n$ est inversible pour tout $n \in \N$, et $A_n$ tend vers $A$.
      
      Le résultat est alors démontré.
\end{document}