\documentclass[fontsize=12pt,twoside=false,parskip=half, french]{scrartcl}
\usepackage[utf8]{inputenc}
\usepackage{babel}
\usepackage[T1]{fontenc}
\usepackage{amsmath, amssymb, stmaryrd}
\usepackage[amsmath]{ntheorem}

\title{Intégrale de Wallis}
\date{}
\author{}

\input{../outils}
\setcounter{secnumdepth}{0}
\begin{document}
\maketitle
   Soit $n \in \N$, l’intégrale de Wallis est 
   \[
      I_n = \int_0^{\frac{\pi}{2}} \cos^n t \dt.
   \]
   \begin{Theoreme}
      On a $nI_n = (n - 1)I_{n - 2}$. De plus
      \[ 
          I_{2p} = \frac{(2p)!}{2^{2p}(p!)^2}\frac{\pi}{2} \text{ et }
          I_{2p + 1} = \frac{2^{2p}(p!)^2}{(2p + 1)!}.
      \]
   \end{Theoreme}
   \section{Démonstration}
      On a en faisant une \emph{intégration par parties} pour $n \geq 2$,
      \begin{align*}
         I_n &= \int_0^{\frac{\pi}{2}} \cos t \cos^{n - 1} t \dt\\
             &= \left[ \sin t \cos^{n - 1}\right]_0^{\frac{\pi}{2}} + (n - 1) \int_0^{\frac{\pi}{2}} \sin^2 t \cos^{n - 2} t \dt\\
             &= (n - 1) \int_0^{\frac{\pi}{2}} (1 - \cos^2 t) \times \cos^{n - 2} t \dt\\
             &= (n - 1) (I_{n - 2} - I_n).
      \end{align*}
      On obtient bien la relation de récurrence voulue. Elle permet d’obtenir les 
      deux formules générales par récurrence, l’initialisation étant faite
      puisque $I_0 = \frac{\pi}{2}$ et $I_1 = 1$.
   % \subsection{Démonstration}

   % \begin{Theoreme}[Équivalent en $+\infty$]
      % On a
      % \[
         % I_n \sim \sqrt{\frac{\pi}{2n}}.
      % \]
   % \end{Theoreme}
   % \section{Démonstration}
      % On a pour $n \in \N$ et $t \in \interv[0; \frac{\pi}{2}], 0 \leq cos^{n + 1} t \leq cos^{n} t$,
      % en intégrant cette inégalité, on obtient que la suite $(I_n)$ est décroissante.
\end{document}