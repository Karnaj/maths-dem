\documentclass[fontsize=12pt,twoside=false,parskip=half]{scrartcl}
\usepackage[utf8]{inputenc}
\usepackage[french]{babel}
\usepackage[T1]{fontenc}
\usepackage{amsmath, amssymb, stmaryrd}
\usepackage[amsmath]{ntheorem}

\title{Matrices compagnons}
\date{}
\author{}

\input{../outils_maths}
\setcounter{secnumdepth}{0}
\begin{document}
\maketitle
   Soit $(a_0, \ldots, a_{n - 1} \in \R^n$ et le polynôme $P = (-1)^n(X^n - (a_{n - 1}X^{n - 1} + \ldots + a_0))$.
   On appelle matrice compagnon du polynôme $P$ la matrice
   \[
      A = 
      \begin{pmatrix}
         0      & \ldots & \ldots & 0      & a_0\\
         1      & 0      &        & \vdots & a_1\\
         0      & 1      & \ddots & \vdots & \vdots\\
         \vdots & \ddots & \ddots & 0      & a_{n - 2}\\
         0      & \ldots & 0      & 1      & a_{n - 1}
      \end{pmatrix}
   \]
   \begin{Theoreme}[Matrices compagnons]
      $\chi_A = P$ et $A$ est diagonalisable si et seulement si $A$ est scindé à racines simples.
   \end{Theoreme}
   \subsection{Démonstration}
      En développant par rapport à la dernière colonne, on montre que $\chi_A = P$.
      % En effet,
      % \[
         % \det (XI_{n + 1} - A) = P(X).
      % \]
      Ce résultat peut aussi être montré par récurrence sur $n \in \N^*$. Pour passer du range $n$ au rang $n + 1$,
      il faudra alors développer par rapport à la première ligne.
      % \begin{align*}
         % \chi_a(X) &= \det (XI_{n + 1} - A)\\ 
         % &= - X \begin{vmatrix}
            % X      & \ldots & \ldots & 0      & -a_0\\
            % 1      & X      &        & \vdots & -a_1\\
            % 0      & 1      & \ddots & \vdots & \vdots\\
            % \vdots & \ddots & \ddots & X      & -a_{n - 2}\\
            % 0      & \ldots & 0      & 1      & X - a_{n - 1}
         % \end{vmatrix} +
         % (-1)^{n + 1}a_0
         % \begin{vmatrix}
            % 1      & -X     & 0      & \ldots & 0\\
            % 0      & 1      & -X     & \vdots & \vdots\\
            % 0      & 0      & \ddots & \vdots & 0\\
            % \vdots & \ddots & \ddots & 1      & -X\\
            % 0      & \ldots & 0      & 0      & 1
         % \end{vmatrix}
      % \end{align*}
      
      Si $P$ est scindé à racines simples, $A$ est bien sûr diagonalisable (son polynôme caractéristique est scindé à 
      racines simples). Pour établir la réciproque, étudions les sous-espaces propres de $A$. Soit $\lambda$ une valeur
      propre de $A$. On a
      % et soit $X = \transposee{(x_1, \ldots, x_n)} \in \Ker(A - \lambda I_{n})$. On a 
      % \[
        % AX = X\lambda I_n \iff       
            % \begin{systeme}
               % a_0 x_n = \lambda x_1\\
               % x_1 + a_1x_n = \lambda x_2\\
               % \vdots\\
               % x_{n - 1} + a_{n - 1}x_n = \lambda x_n
            % \end{systeme} \iff 
            % \begin{cases}
               % \begin{systeme}
                  % x_1 = 1
                  % a_0 x_n = \lambda x_1\\
                  % x_1 + a_1x_n = \lambda x_2\\
                  % \vdots\\
                  % x_{n - 1} + a_{n - 1}x_n = \lambda x_n
               % \end{systeme}
            % \end{cases}
      % \]
      \[
         \rg(a - \lambda I_n) = \rg       
            \begin{pmatrix}
              -\lambda & \ldots   & \ldots & 0        & a_0\\
               1       & -\lambda &        & \vdots   & a_1\\
               0       & 1        & \ddots & \vdots   & \vdots\\
               \vdots  & \ddots   & \ddots & -\lambda & a_{n - 2}\\
               0       & \ldots   & 0      & 1        & a_{n - 1} -\lambda 
            \end{pmatrix} = n - 1.
      \]
      En effet les $n - 1$ premiers vecteurs colonnes sont libres. Les sous-espaces propres de $A$ sont donc de 
      dimension $1$. Si $A$ est diagonalisable, $E$ est la somme directe des sous-espaces propres, d’où il y en
      a $n$, donc $n$ valeurs propres distinctes. Par suite, $\chi_A$ est scindé à racines simples.
\end{document}
