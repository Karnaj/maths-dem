\documentclass[fontsize=12pt,twoside=false,parskip=half]{scrartcl}
\usepackage[utf8]{inputenc}
\usepackage[french]{babel}
\usepackage[T1]{fontenc}
\usepackage{amsmath, amssymb, stmaryrd}
\usepackage[amsmath]{ntheorem}

\title{Théorème de Césaro}
\date{}
\author{}

\input{../outils_maths}
\setcounter{secnumdepth}{0}
\begin{document}
\maketitle
   \begin{Theoreme}[Théorème de Césaro]
      Soit $\suite{u}$ une suite numérique de limite $\ell$. Alors,
      \[
         \frac{1}{n}\sum_{k = 1}^n u_k \to \ell.
      \]
   \end{Theoreme}
   \subsection{Démonstration}
      On peut écrire $u_n = \ell + v_n$ avec $v_n = o(1)$. Alors
      \[
        \frac{1}{n}\sum_{k = 1}^n u_k = \ell + \frac{1}{n}\sum_{k = 1}^n v_k. 
      \]
      On a que $v_n = o(1)$ et la série « somme des $1$ » diverge et est à termes positifs, le \emph{théorème de sommation 
      des relations de comparaison} donne
      \[
         \sum_{k = 1}^n v_k = o\left(\sum_{k = 1}^n 1\right) = o(n).
      \]
      Par suite,
      \[
         \frac{1}{n}\sum_{k = 1}^n u_k = \ell + \frac{1}{n} o(n) = \ell + o(1).
      \]
      Le résultat est démontré.
\end{document}