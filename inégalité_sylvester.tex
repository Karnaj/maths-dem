\documentclass[fontsize=12pt,twoside=false,parskip=half,french]{scrartcl}
\usepackage[utf8]{inputenc}
\usepackage{babel}
\usepackage[T1]{fontenc}
\usepackage{amsmath, amssymb, stmaryrd}
\usepackage[amsmath]{ntheorem}

\title{Inégalité de Sylvester}
\date{}
\author{}

\input{../outils}
\setcounter{secnumdepth}{0}
\begin{document}
\maketitle
   \begin{Theoreme}[Inégalité de Sylvester]
      Soit $E$ un espace vectoriel de dimension $n$ et soit $a$ et $b$ deux 
      endomorphismes de $E$. Alors
      \[
         \rg(a) + \rg(b) - n \leq \rg(ab) \leq \min(\rg(a), \rg(b)).
      \]
   \end{Theoreme}
   \subsection{Démonstration}
      On a 
      \begin{itemize}
         \item $\Img(ab) \subset \Img(a)$, donc $\rg(ab) \leq \rg(a)$. 
         \item $\Img(ab) = a(\Img(b))$ avec $\dim\left(a(\Img(b))\right) \leq \dim(\Img(b)$
               donc $\rg(ab) \leq rg(b)$.
      \end{itemize}
      Donc
      \[
        \rg(ab) \leq \min(\rg(a), \rg(b)).
      \]
      Pour l'autre inégalité, soit $a'$ la restriction de $a$ à $\Img(b)$. On
      a alors $\Img(a') = \Img(ab)$ et $\Ker(a') = \Ker(a) \cap \Img(b)$. On a
      par le théorème du rang
      \[
         \rg(a') = \dim(\Img b) - \dim(\Ker(a')
      \]
      c'est-à-dire
      \begin{align*}
         \dim(\Img(ab)) &= \dim(\Img b) - \dim\left( \Ker(a) \cap \Img(b) \right)\\
                       &\geq \dim(\Img b) - \dim(\Ker(a))\\
                       &\geq rg(b) - (n - \rg(a)).
      \end{align*}
      Et donc on a bien
      \[
        \rg(a) + \rg(b) - n \leq \rg(ab).
      \]
\end{document}