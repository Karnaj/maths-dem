\documentclass[fontsize=12pt,twoside=false,parskip=half, french]{scrartcl}
\usepackage[utf8]{inputenc}
\usepackage{babel}
\usepackage[T1]{fontenc}
\usepackage{amsmath, amssymb, stmaryrd}
\usepackage[amsmath]{ntheorem}

\title{Crochet de Lie}
\date{}
\author{}

\input{../outils}
\setcounter{secnumdepth}{0}
\begin{document}
\maketitle
   \begin{Theoreme}
      Soit $A, B$ dans $M_n(\C)$ telles que $AB - BA = B$. Montrer que $B$ est nilpotente.
   \end{Theoreme}
   \subsection{Démonstration}
      Commençons par essayer les puissances de $B$. On part de la relation $AB - BA = B$ qu’on multiplie
      par $B$ à droite (ce qui donne $AB^2 - BAB = B^2$) et à droite (ce qui donne $BAB - B^2A = B^2$).
      En additionnant les deux, $2B^2 = AB^2 - B^2A$.

      Montrons par récurrence que $AB^k - B^kA = kB^k$. La relation est vraie pour $k = 1$ et $k = 2$.
      Supposons la vraie au rang $k -1$. On a
      \begin{align*}
         AB^k - BA^k &= (AB^{k - 1} - B^{k - 1}A)B + B^{k - 1}(AB - BA)\\
                     &= (k - 1)B^k + B^k = kB^k.
      \end{align*}
      Le résultat est démontré.

      Reste à montrer que $f$ est nilpotente. Pour cela, considérons l’endomorphisme
      \[
         f \colon M \to AM - MA.
      \]
      Si $B$ n’est pas nilpotente, alors $B^k$ n’est pas nulle à partir d’un certain rang. Or $B^k$ non nulle implique
      que $k$ est dans le spectre de$f$. En effet, $f(B^k) = kB^k$. ABSURDE, car le spectre de $f$ est finie.
      Donc $B$ est nilpotente.
\end{document}
