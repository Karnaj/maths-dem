\documentclass[fontsize=12pt,twoside=false,parskip=half, french]{scrartcl}
\usepackage[utf8]{inputenc}
\usepackage{babel}
\usepackage[T1]{fontenc}
\usepackage{amsmath, amssymb, stmaryrd}
\usepackage[amsmath]{ntheorem}

\title{Intégrale de Bertrand}
\date{}
\author{}

\input{../outils}
\setcounter{secnumdepth}{0}
\begin{document}
\maketitle
    Soit $\alpha$ et $\beta$ deux réels. L’intégrale de Bertrand est l’intégrale
   \[
      \int_{\e}^{+\infty} \frac{\dt}{t^\alpha \left(\ln t \right)^\beta}.
   \]
   \begin{Theoreme}[Intégrale de Bertrand]
      L’intégrale de Bertrand converge si et seulement si $\alpha > 1$ ou $\alpha = 1$ et $\beta > 1$.
   \end{Theoreme}
   \subsection{Démonstration}
      Posons
      \[
         f \colon t \to \frac{1}{t^\alpha \left(\ln t \right)^\beta}.
      \]
      $f$ est définie, continue et positive sur $\interv[\e ; +\infty[$.
      
      Si $\alpha < 1$,
      \[
         tf(t) = \frac{t^{1 -\alpha}}{\left(\ln t \right)^\beta} \to +\infty
      \]
      donc $f(t) \geq 1/t$ pour $t$ assez grand, or l’intégrale de $1/t$ sur $\interv[\e$; $+\infty[$ diverge, donc \emph{par comparaison de fonctions positives}, l’intégrale de Bertrand diverge.
      
      Si $\alpha > 1$, on prend $m \in \interv]1; \alpha[$. Alors
      \[
         t^mf(t) = \frac{1}{t^{\alpha - m}}{\left(\ln t \right)^\beta} \to 0
      \]
      c’est-à-dire $f(t) = o(1/t^m$), avec l’intégrale de $1/t^m$ sur $\interv[\e ; +\infty[$ qui converge (intégrale de Riemann avec $m > 1$), donc \emph{par comparaison de fonctions positives}, l’intégrale de Bertrand converge.
         
      Si $\alpha = 1$, on fait le changement de variable $u = \ln t$.
      \[
         \int_{\e}^x \frac{\dt}{t \left(\ln t \right)^\beta} = \int_1^{\ln x} \frac{\du}{u^\beta}
      \]
      converge si et seulement si $\beta > 1$ (\emph{Riemann}).
% \section{Séries de Bertrand}
   % Soit $\alpha$ et $\beta$ deux réels positifs. Posons
   % \[
      % u_n = \frac{1}{n^\alpha \left(\ln t \right)^\beta}.
   % \]
   % La série de Bertrand est la série numérique
   % \[
      % \sum_{n \geq 2} u_n.
   % \]
   % \begin{Theoreme}[Série de Bertrand]
      % La série de Bertrand converge si et seulement si $\alpha > 1$ ou $\alpha = 1$ et $\beta > 1$.
   % \end{Theoreme}
   % En effet, étudions les différents cas. 
   
   % \subsection{Cas $\alpha > 1$}
      % Prenons $1 < \gamma < \alpha$. On a,
      % \begin{align*}
         % u_n &= o\left(\frac{1}{n^\gamma}\right)\\
         % \frac{1}{n^\gamma} &> 0\\
         % \sum_{n > 2} &\frac{1}{n^\gamma} \text{ converge ((série de Riemann avec $\gamma > 0$))}
      % \end{align*}
      % Donc, par comparaison de séries à termes positifs, la série de Bertrand converge.
      
   % \subsection{Cas $\alpha < 1$}
      % On a $\frac{1}{n} = o(u_n)$, or la série harmonique diverge, donc la série de Bertrand diverge.
   
   % \subsection{Cas $\alpha = 1$ et $\beta < 1$}
      % On a 
\end{document}