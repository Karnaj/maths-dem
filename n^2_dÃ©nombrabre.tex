\documentclass[fontsize=12pt,twoside=false,parskip=half]{scrartcl}
\usepackage[utf8]{inputenc}
\usepackage[french]{babel}
\usepackage[T1]{fontenc}
\usepackage{amsmath, amssymb, stmaryrd}
\usepackage[amsmath]{ntheorem}

\title{$\N^2$ est dénombrabre}
\date{}
\author{}

\input{../outils_maths}
\setcounter{secnumdepth}{0}
\begin{document}
\maketitle
   \begin{Theoreme}
      $\N^2$ est dénombrabre.
   \end{Theoreme}
   \subsection{Démonstration}
      Pour montrer ce résultat, il nous suffit d’exhiber une bijection entre $\N$ et $\N^2$. Ici, nous 
      allons plutôt créer une bijection entre $\N^*$ et $\N^2$. Pour cela, nous allons utiliser le fait
      que tout nombre s’écrit sous la forme du produit d’une puissance de deux et d’un nombre impair. Posons 
      \[
         \phi(k, l) = 2^k(2l + 1).
      \]
      L’existence et l’unicité de la décomposition en facteurs premiers de tout entier strictement positif
      donne alors la bijectivité de $\phi$. On peut néanmoins la montrer par le calcul. Par exemple, pour
      l’injectivité, supposons, $\phi(k, l) = \phi(p, q)$ avec $(k, l) \neq (p, q)$. On a alors, 
      \begin{align*}
         2^k(2l + 1) = 2^p(2q + 1) \implies 2^{k - p}(2l + 1) = (2q + 1).
      \end{align*}
      Si $k \neq p$, alors $2q + 1$ est pair, impossible, donc $k = p$. Alors, on a bien $(k, l) = (p, q)$. 

      Notons que la décomposition en facteurs premiers donne aussi une bijection entre $\N^*$ et $\N^3$ et 
      en général entre $\N^*$ et $\N^k$ pour tout $k > 0$. En effet, en posant $p_1,p_2, \ldots, p_{k - 1}$, les $k - 1$
      premiers entiers premiers, tout entier $p$ s’écrit sous la forme du produit de puissances des $p_i$ et d’un 
      nombre qui n’est multiple d’aucun des $p_i$.
      \[
         p = \prod_{i = 1}^{k - 1} p_i^{\alpha_i} \times (p_1\cdots p_k \alpha_k + 1).
      \]
\end{document}