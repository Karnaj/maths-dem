\documentclass[fontsize=12pt,twoside=false,parskip=half, french]{scrartcl}
\usepackage[utf8]{inputenc}
\usepackage{babel}
\usepackage[T1]{fontenc}
\usepackage{amsmath, amssymb, stmaryrd}
\usepackage[amsmath]{ntheorem}

\title{Générateurs de $S_n$}
\date{}
\author{}

\input{../outils}
\setcounter{secnumdepth}{0}
\begin{document}
\maketitle
   \begin{Theoreme}
      Soit $n \in \N$. Le groupe symétrique $S_n$ est engendré par les transpositions.
   \end{Theoreme}
   \subsection{Démonstration}
      Il suffit de montrer que toute permutation $\tau$ se décompose en produits 
      de transpositions et puisqu’une permutation se décompose en produits de 
      cycles disjoints, il suffit de montrer que tout cycle se décompose en produits
      de permutation. Soit $\tau = (a_1 a_2 \cdots a_k)$ un cycle. Alors
      \[
         \tau = (a_1 a_2) (a_2 a_3) \cdots (a_{n - 1} a_n).
      \]
      D’où le résultat.
      
      PS : on a également $(a b) = (1 a) (1 b) (1 a)$, d’où $S_n$ est engendré 
      par les transpositions $(1, k)$ avec $k \in \intervalleEntier{1}{n}$.
\end{document}