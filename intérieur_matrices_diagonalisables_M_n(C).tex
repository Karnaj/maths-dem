\documentclass[fontsize=12pt,twoside=false,parskip=half,french]{scrartcl}
\usepackage[utf8]{inputenc}
\usepackage{babel}
\usepackage[T1]{fontenc}
\usepackage{amsmath, amssymb, stmaryrd}
\usepackage[amsmath]{ntheorem}

\title{Intérieur des matrices diagonalisables de $M_n(\C)$}
\date{}
\author{}

\newcommand*{\D}{\mathcal{D}_n(\C)}
\newcommand*{\Dn}{\mathcal{D}_n'(\C)}

\input{../outils}
\setcounter{secnumdepth}{0}
\begin{document}
\maketitle
   Soit $n \in \N$, on note $\D$ l'ensemble des matrices
   diagonalisables de $\M_n(\C)$ et $\Dn$ l'ensemble des
   matrices à $n$ valeurs propres distinctes (donc $\Dn \subset \D$). 
   \begin{Theoreme}[Intérieur de $\D$]
      On a $\Int{\D} = \Dn$.
   \end{Theoreme}
   \subsection{Démonstration}
      Montrons que $\Dn$ est ouvert. On a $M \in \Dn$ si et seulement si 
      les racines de $\chi_M$ sont distinctes. Notons 
      $(\lambda_{1,M}, \ldots, \lambda_{n, M})$ 
      les $n$ racines de $\chi_M$ et considérons
      \[
        f(M) = \prod_{i \neq j} (\lambda_{i,M} - \lambda_{j, M}).
      \]
      $f$ est continue en tant que fonction polynoimiale des coefficients
      de $M$. On en déduit que $\Dn$ est ouvert en tant qu'image réciproque de
      $\C^*$ par $f$.
      
      On a $\Dn \subset \D$ et puisque $\Dn$ est ouvert, $\Dn \subset \Int{\D}$.
      Montrons que $\Int{\D} \subset \Dn$. Supposons qu'il existe $M \in \Int{\D}$ 
      avec une valeur propre $\lambda$ d'ordre supérieur ou égal à $2$. Il existe
      $P$ inversible telle que $M = PDP^{-1}$ avec 
      $D = \Diag{\lambda, \lambda, \lambda_3, \ldots, \lambda_n}$. On pose pour
      tout entier $k > 0$
      \begin{alignat*}{3}
         \Delta_k = \begin{pmatrix}
            \lambda & \frac{1}{k}\\
            0 & \lambda
         \end{pmatrix}  
         & & \quad & &
         D_k = \begin{pmatrix}
             \Delta_k &   & & 0\\
                    &  \lambda_3 & &\\ 
                    &   &    \ddots & \\
                  0 &           & & \lambda_n
          \end{pmatrix} 
      \end{alignat*}
      Le polynôme minimal de $D_k$ est un multiple de celui de $D_k$ avec
      $\chi_{D_k} = (X - \lambda)^2$ (c'est ça où $X - \lambda$ et $X - \lambda$
      n'annnule pas $D_k$). Par suite $M_k = PD_kP^{-1}$ n'est pas diagonalisable
      puisque son polynôme minimal n'est pas scindé à racines simples.
      
      Mais on a $M_k \underset{+\infty}{\to} M$, donc $M \not\in \Int{\D}$.
      Donc $\Int{\D} \subset \Dn$ et on a l'égalité par double inclusion.
\end{document}