\documentclass[fontsize=12pt,twoside=false,parskip=half, french]{scrartcl}
\usepackage[utf8]{inputenc}
\usepackage{babel}
\usepackage[T1]{fontenc}
\usepackage{amsmath, amssymb, stmaryrd}
\usepackage[amsmath]{ntheorem}

\title{Densité des matrices diagonalisables dans $\M_n(\C)$}
\date{}
\author{}

\input{../outils}
\setcounter{secnumdepth}{0}
\begin{document}
\maketitle
   \begin{Theoreme}
      L’ensemble des matrices diagonalisables à coefficients dans $\C$ est dense
      dans $\M_n(\C)$.
   \end{Theoreme}
   \subsection{Démonstration}
      Soit $A$ dans $\M_n(\C)$. $A$ est triangulaire dans une base adaptée. Notons
      $\lambda_k$, $k \in \intervalleEntier{1}{n}$ les valeurs propres de $A$.
      Si elles sont distinctes, $A$ est diagonalisable.
      
      Sinon, notons $A’_{p}$ la suite de matrice diagonales telle que 
      \[
         \forall k \in \intervalleEntier{1}{n}, a’_{p, k, k} = \frac{1}{kp}.
      \]
      
      Posons $A_p = A + A’_p$. À partir d’un certain rang $N$, $A_p$ possède 
      $n$ valeurs propres distinctes, donc est diagonalisable. Mais on note que
      \[
         \normeInf{A - A_p} = \normeInf{A’_p} \leq \frac{1}{p}. 
      \]
      Donc la suite $(A_p)$ tend vers $A$. Ainsi, toute matrice est limite d’une
      suite de matrices diagonalisables pour la norme infinie (donc pour toutes 
      les normes vu qu’elles sont équivalentes en dimension finie). On en déduit
      que l’ensemble des matrices diagonalisables est dense dans $\M_n(\C)$.
      
      PS : attention, ce n’est pas le cas dans $\M_n(\R)$. Prenons
      \[
         A = \begin{pmatrix} 0 & -1 \\ 1 & 0 \end{pmatrix}. 
      \]
      Son polynôme caractéristique, $X^2 + 1$, n’admet pas de racines réelles, 
      d’où $A$ n’est pas diagonalisable. Supposons qu’il existe une suite $D_n$ 
      de matrices diagonalisables qui converge vers $A$. 
      
      Le discriminant du polynôme caractéristique des $D_k$ est continu (car  
      polynomial en les coefficients) et il est positif puisqu’elles sont 
      diagonalisables. Le passage à la limite donne que le discriminant de 
      $X^2 + 1$ est positif ou nul, ce qui est absurde.
\end{document}