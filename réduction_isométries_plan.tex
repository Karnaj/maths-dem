\documentclass[fontsize=12pt,twoside=false,parskip=half, french]{scrartcl}
\usepackage[utf8]{inputenc}
\usepackage{babel}
\usepackage[T1]{fontenc}
\usepackage{amsmath, amssymb, stmaryrd}
\usepackage[amsmath]{ntheorem}

\title{Réduction des isométries du plan}
\date{}
\author{}

\input{../outils}
\setcounter{secnumdepth}{0}
\begin{document}
\maketitle
   \begin{Theoreme}[Réduction des isométries du plan]
      Soit $f$ une isométrie positive du plan. Alors, il existe $\theta \in \R$ 
      tel que sa matrice représentative soit de la forme
      \[
         R(\theta) = 
         \begin{pmatrix}
            \cos \theta & -\sin \theta \\
            \sin \theta & \cos \theta
         \end{pmatrix}
      \]
   \end{Theoreme}
   \subsection{Démonstration}
      Soit $M$ une matrice de $SO_2(\R)$, 
      \[
         M = 
         \begin{pmatrix}
            a & b \\
            c & d
         \end{pmatrix}
      \]
      Puisque $a^2 + c^2 = 1$, il existe $\theta$ tel que $a = \cos \theta$ et
      $c = \sin \theta$.
      
      De plus, on a $\det M = 1$, c’est-à-dire $ad - bc$, par suite
      \[
         (a - d)^2 + (b - c)^2 = 2 - 2(ad - bc) = 0.
      \]
      Donc $a = d$ et $b = c$. On a bien la forme voulue.
      
      Un simple calcul montre que $R(\theta)R(\theta’) = R(\theta + \theta’)$,
      ce qui est bien normal, faire la composée de deux rotations d’angle $\theta$ 
      et $\theta’$ équivaut à faire une seule rotation d’angle $\theta + \theta’$.
      
      PS : dans le cas d’une isométrie du plan négative, on a $M$ de la forme
      \[
         S(\theta) = 
         \begin{pmatrix}
            \cos \theta & \sin \theta \\
            \sin \theta & -\cos \theta
         \end{pmatrix}
      \]
      En effet, $(a + d)^2 + (b - c)^2 = 2 + 2(ad - bc) = 0$, car $ad - bc = \det M
      = -1$. Par suite, $a = -d$ et $b = c$. Et puisque $a^2 + c^2 = 1$, on a 
      l’existence de $\theta$ tel que $a = \cos \theta$ et $c = \sin \theta$.
\end{document}
