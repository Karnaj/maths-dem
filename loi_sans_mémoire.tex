\documentclass[fontsize=12pt,twoside=false,parskip=half]{scrartcl}
\usepackage[utf8]{inputenc}
\usepackage[french]{babel}
\usepackage[T1]{fontenc}
\usepackage{amsmath, amssymb, stmaryrd}
\usepackage[amsmath]{ntheorem}

\title{Loi sans mémoire}
\date{}
\author{}

\input{../outils_maths}
\setcounter{secnumdepth}{0}
\begin{document}
\maketitle
   \begin{Theoreme}
      Soit $X$ une variable aléatoire à valeurs dans $\N^*$ tel que pour 
      tout $n, k$ dans $\N$,
      \[
         P(X > n + k \sachant X > n) = P(X > k).
      \]
      Alors $X$ suit une loi géométrique.
   \end{Theoreme}
   \subsection{Démonstration}
      En posant $q = P(X > 1)$, on a grâce à l’hypothèse que
      \begin{align*}
         P(X > n + 1) &= P(X > n + 1 \sachant X > n)P(X > n)\\ 
                      &= qP(X > n).
      \end{align*}
      Une récurrence donne alors (l’initialisation étant faite car 
      $P(X > 0) = 1$) que pour tout entier naturel $n$,
      \[
         P(X > n) = q^n.
      \]
      Ceci permet d’obtenir que
      \begin{align*}
         P(X = n) &= P(X > n - 1) - P(X > n)\\ 
                  &= q^{n -1} - q^n\\
                  &= (1 - q)^{n - 1}.
      \end{align*}
      On obtient alors que $X$ suit une loi géométrique de paramètre $1 - q$.
\end{document}