\documentclass[fontsize=12pt,twoside=false,parskip=half, french]{scrartcl}
\usepackage[utf8]{inputenc}
\usepackage{babel}
\usepackage[T1]{fontenc}
\usepackage{amsmath, amssymb, stmaryrd}
\usepackage[amsmath]{ntheorem}

\title{Projecteurs orthognaux}
\date{}
\author{}

\input{../outils}
\setcounter{secnumdepth}{0}
\begin{document}
\maketitle
   Soient $E$ un EVN de dimension $n$, $F$ et $G$ deux SEV de $E$ et $(e_0, \ldots,e_m)$ une base 
   orthonormale de $F$.
   \begin{Theoreme}
      Une projection $p$ sur $F$ est orthogonale si et seulement si pour tout $x$ de $E$, $\norme{p(x)} \leq \norme{x}$.
      On a alors que la distance de $x$ à $F$ est $\norme{x - p(x)}$ et que
      \[
         p(x) = \sum_{k = 0}^{m} \scalaire{e_k}{x} e_k.
      \]
   \end{Theoreme}
   \subsection{Démonstration}
      Si $p$ est orthogonale. Pour tout $y \in F$, le \emph{théorème de Pythagore} donne %(car $(x - p(x)) \in \orthogonal{F}$ et $(p(x) - y) \in F$)
      \[
         \norme{x - y}^2 = \norme{(x - p(x)) + (p(x) - y)} 
                         = \norme{x - p(x)}^2 + \norme{p(x) - y}^2
                         \geq \norme{x - p(x)}         
      \]
      avec égalité si et seulement si $y = p(x)$, d’où $\distance{x}{F} = \norme{x - p(x)}$. 
      
      De plus, puisque $p(x) \in F$, on a
      \[
         p(x) = \sum_{k = 0}^m \scalaire{e_k}{p(x)} e_k 
      \]
      avec $(\scalaire{e_k}{x - p(x)} = 0 $ (car $x - p(x) \in \orthogonal{F}$), donc 
      $\scalaire{e_k}{p(x)} = \scalaire{e_k}{x}$, c’est-à-dire
      \[
         p(x) = \sum_{k = 0}^{m} \scalaire{e_k}{x} e_k.
      \]
      Soit $x \in E$. Si $p$ est la projection orthogonale sur $F$, on a $x = p(x) + (x - p(x))$
      avec $p(x) \in F$ et $x - p(x) \in \orthogonal{F}$. Le \emph{théorème de Pythagore} donne alors
      \[
         \norme{x}^2 = \norme{p(x)}^2 + \norme{x - p(x)}^2 \geq \norme{p(x)}^2.
      \]
      Réciproquement, soit $p$ la projection sur $F$ parralèlement à un SEV $G$ avec
      pour tout $x \in E$, $\norme{p(x)} \leq \norme{x}$. Pour $(a, b) \in F \times G$
      et $\lambda \in \R$ considérons $x = a + \lambda b$. On a 
      \[
         \norme{p(x)} \leq \norme{x} \implies 2\lambda \scalaire{a}{b} + \lambda^2\norme{b}^2 \geq 0
      \]
      Si $\scalaire{a}{b} \neq 0$, cette expression n’est pas de signe constant au voisinage de $0$
      (équivalent à $2 \lambda\scalaire{a}{b}$. ABSURDE, donc $\scalaire{a}{b} = 0$.
\end{document}