\documentclass[fontsize=12pt,twoside=false,parskip=half, french]{scrartcl}
\usepackage[utf8]{inputenc}
\usepackage{babel}
\usepackage[T1]{fontenc}
\usepackage{amsmath, amssymb, stmaryrd}
\usepackage[amsmath]{ntheorem}

\title{Coefficient binomiaux}
\date{}
\author{}

\input{../outils}
\setcounter{secnumdepth}{0}
\begin{document}
\maketitle
   \begin{Theoreme}
      Soit $E$ un ensemble à $n$ éléments. Le nombre de parties à $k$ éléments de $E$ est
      \[
         \binom{n}{k} = \frac{n!}{k!(n- k)!}.
      \]
   \end{Theoreme}
   \subsection{Démonstration}
      Pour obtenir une partie de cardinal $k$, on choisit $k$ éléments distincts $x_1, \ldots, x_k$ dans 
      $\intervalleEntier{1}{n}$, on a $n$ choix pour $x_1$, $n - 1$ pour $x_2$, ... et $n - k + 1$ pour $x_k$. On en 
      déduit que le nombre de $k$-uplets $(x_1, \ldots, x_k)$ est
      \[
         n(n - 1)\cdots(n - k + 1) = \frac{n!}{(n - k)!}.
      \]
      Deux $k$-uplets définissent une même partie à $k$ éléments s’ils sont égaux à l’ordre près, la même partie est
      donc représentée par $k!$ $k$-uplets. D’où le résultat.
      
      PS : Ici, nous avons également montré la valeur du nombre d’arrangement $A_n^k$.
\end{document}