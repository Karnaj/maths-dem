\documentclass[fontsize=12pt,twoside=false,parskip=half, french]{scrartcl}
\usepackage[utf8]{inputenc}
\usepackage{babel}
\usepackage[T1]{fontenc}
\usepackage{amsmath, amssymb, stmaryrd}
\usepackage[amsmath]{ntheorem}

\title{Reste d'une série de Riemann}
\date{}
\author{}

\input{../outils}
\setcounter{secnumdepth}{0}
\begin{document}
\maketitle
    Soit $\alpha > 1$ de sorte que la série de Riemann associée converge. Alors,
   \begin{Theoreme}[Intégrale de Riemann]
      Soit $n > 0$ et $R_n$ le reste d'ordre $n$ de la série de Riemann. On a
      \[
         R_n \underset{+\infty}{\sim} \frac{1}{(\alpha - 1)n^{\alpha - 1}}.
      \]
   \end{Theoreme}
   \subsection{Démonstration}
      On passe par une \emph{comparaison série-intégrale}. Puisque $\alpha > 1$,
      la fonction $t \mapsto 1/t^\alpha$ est décroissante et intégrable sur
      $\interv[1; +\infty[$ (\emph{intégrale de Riemann}), et pour $k \leq 2$,
      on a
      \[
         \int_k^{k +1} \frac{\dt}{t^\alpha} \leq \frac{1}{k^\alpha} \leq \int_{k - 1}^{k} \frac{\dt}{t^\alpha}
      \]
      En sommant de $n + 1$ à $N$ on obtient
      \[
         \int_{n + 1}^{N +1} \frac{\dt}{t^\alpha} \leq \sum_{k = n}^{N} \frac{1}{k^\alpha} \leq \int_{n + 1}^{N} \frac{\dt}{t^\alpha}.
      \]
      Il ne reste plus qu'à faire tendre $N$ vers l'infini. On remarque d'abord que pour tout  $a > 0$
      \[
         \int_{a}^{x} \frac{\dt}{t^\alpha} = 
         \frac{x^{1-\alpha} - a^{1- \alpha}}{1 - \alpha},   
      \]
      d'où
      \[
         \lim_{x \to +\infty} \int_{a}^{x} \frac{\dt}{t^\alpha} = 
         \frac{a^{1- \alpha}}{\alpha - 1} = 
         \frac{1}{(\alpha - 1)a^{\alpha - 1}}.
      \]
      En faisant tendre $N$ vers l'infini dans notre ingéalité, on obtient donc
      \[
         \frac{1}{(\alpha - 1)(n + 1)^{\alpha - 1}} \leq 
         R_n \leq
         \frac{1}{(\alpha - 1)n^{\alpha - 1}}.
      \]
      D'où le résultat.
\end{document}