\documentclass[fontsize=12pt,twoside=false,parskip=half]{scrartcl}
\usepackage[utf8]{inputenc}
\usepackage[french]{babel}
\usepackage[T1]{fontenc}
\usepackage{amsmath, amssymb, stmaryrd}
\usepackage[amsmath]{ntheorem}

\title{Non dénombrabilité de $\R$}
\date{}
\author{}

\input{../outils_maths}
\setcounter{secnumdepth}{0}
\begin{document}
\maketitle
   \begin{Theoreme}
      L’ensemble des suites à valeurs dans $\{0, 1\}$ n’est pas dénombrabre.
   \end{Theoreme}
   \subsection{Démonstration}
      Pour montrer ce résultat, supposons cet ensemble dénombrabre. On peut alors
      numéroter tous les éléments de cet ensemble grâce à une suite $(U_n)$ tel que
      \[
         U_n = (U_n^k)_{k \in \N}.
      \]
      On considère alors une suite $u$ définie par
      \[
         v_k =
         \begin{cases}
            0 & \text{ si $U_n^k$ = 1} \\
            1 & \text{ sinon}
         \end{cases}.
      \]
      On a que $(v_k)$ est bien une suite à valeurs dans $\{0, 1\}$, pourtant $(v_k)$ 
      n’est pas un élément de $(U_n)$. ABSURDE. Donc cet ensemble de suite n’est pas 
      dénombrabre.
\end{document}