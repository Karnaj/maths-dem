\documentclass[fontsize=12pt,twoside=false,parskip=half, french]{scrartcl}
\usepackage[utf8]{inputenc}
\usepackage{babel}
\usepackage[T1]{fontenc}
\usepackage{amsmath, amssymb, stmaryrd}
\usepackage[amsmath]{ntheorem}
\usepackage{tikz,tkz-tab}

\title{Lemme de décomposition des noyaux}
\date{}
\author{}

\input{../outils}
\setcounter{secnumdepth}{0}
\begin{document}
\maketitle
   \begin{Theoreme}[Lemme de décomposition des noyaux]
      Soit $E$ un $K$-espace vectoriel non nul. Soient $P, Q \in K[X]$ et 
      $u$ un endomorphisme de $E$. Si $P$ et $Q$ sont premiers entre eux, alors
      \[
        \ker(PQ)(u) = \ker P(u) \oplus \ker Q(u)
      \]
   \end{Theoreme}
   \subsection{Démonstration}
      On a $P \wedge Q = 1$, donc il existe $U, V$, $UP + VQ = 1$. On a alors
      \[
         \Id = U(u) \circ P(u) + V(u) \circ Q(u). 
      \]
      Pour $x \in \ker(P)(u) \cap \ker(Q)(u)$, on a donc
      \[
         x = \left(U(u) \circ P(u)\right)(x) + \left(V(u) \circ Q(u)\right)(x) = 0
      \]
      d'où $\ker(P)(u)$ et $\ker(Q)(u)$ sont en somme directe.
      
      Reste à montrer l'égalité. 
      
      On a $(PQ)(u) = Q(u) \circ P(u)$, donc $\ker(P)(u) \subset \ker(PQ)(u)$
      et $\ker(Q)(u) \subset \ker(PQ)(u)$, d'où 
      $\ker(P)(u) \oplus \ker(P)(u) \subset \ker(PQ)(u)$.
      
      Réciproquement, considérons $x \in \ker(PQ)(u)$. On a
      \[
        x =  \underbrace{\left(U(u) \circ P(u)\right)(x)}_{a} + 
             \underbrace{\left(V(u) \circ Q(u)\right)(x)}_{b}.
      \]
      Mais
      \[
        Q(u)(a) = \left(Q(u) \circ U(u) \circ P(u)\right)(x) = 
                  \left(U(u) \circ (PQ)(u)\right)(x) = 0
      \]
      et de même $P(u)(b) = 0$, donc $a \in \ker P(u)$ et $b \in \ker Q(u)$,
      d'où l'inclusion inverse et donc l'égalité par double inclusion.
      
      PS : Une récurrence donne que si $P_1, \ldots, P_n$ sont premiers
      entre eux deux à deux, alors
      \[
         \ker(P_1\ldots P_n)(u) = \ker P_1(u) \oplus \ldots \oplus \ker P_n(u).
      \]
      Pour l'hérédite, il faudra montrer que $P_{n + 1}$ est premier avec
      $P_1\ldots P_n$. 
\end{document}