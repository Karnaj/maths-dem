\documentclass[fontsize=12pt,twoside=false,parskip=half]{scrartcl}
\usepackage[utf8]{inputenc}
\usepackage[french]{babel}
\usepackage[T1]{fontenc}
\usepackage{amsmath, amssymb, stmaryrd}
\usepackage[amsmath]{ntheorem}

\title{Groupe orthogonal d’ordre $n$}
\date{}
\author{}

\input{../outils_maths}
\setcounter{secnumdepth}{0}
\begin{document}
\maketitle
   \begin{Theoreme}[Groupe orthogonal d’ordre $n$]
      L’ensemble $\groupeOrthogonal_n(\R)$, des matrices othogonales est un sous-groupe compact de $(\GL_n(\R), \times)$.
   \end{Theoreme}
   \subsection{Démonstration (limites de suite)}
      On a bien $\groupeOrthogonal_n(\R) \subset \GL_n(\R)$ (ce sont des matrices d’inverse leur transposée). Montrons que
      c’est un sous-groupe.
      \begin{enumerate}
         \item $I_n \in \groupeOrthogonal_n(\R)$.
         \item Soit $(A, B) \in \groupeOrthogonal_n(\R)^2$. $\transposee{(AB)}AB = \transposee{B}\transposee{A}AB = \transposee{B}B = I_n$, donc
               $\groupeOrthogonal_n(\R)$ est stable parfois.
         \item Soit $A \in \groupeOrthogonal_n(\R)$. $\transposee{A^{-1}}A^{-1} = \transposee{\transposee{A}}\transposee{A} = A\transposee{A} = I_n$,
               donc $A^{-1} \in \groupeOrthogonal_n(\R)$.         
      \end{enumerate}
      Donc $\groupeOrthogonal_n(\R)$ est un sous-groupe de $(\GL_n(\R), \times)$.
      
      Montrons qu’il est compact. Nous sommes en dimension finie, il suffit de montrer qu’il est fermé borné.
      
      En tant qu’image réciproque du fermé ${I_n}$ par l’application continue $A \to \transposee{A}A$, il est fermé.
      
      Considérons la norme euclidienne associée au produit scalaire canonique sur $\M_n(\R)$. On a pour $A \in \groupeOrthogonal_n(\R)$, 
      \[
         \norme{A} = \sqrt{\Tr(\transposee{A}A)} = \sqrt{\Tr(I_n)} = \sqrt{n}.
      \]
      Par suite, $\groupeOrthogonal_n(\R)$ est bornée.
      
      Donc on a bien que $\groupeOrthogonal_n(\R)$ est un sous-groupe compact de $\M_n(\R)$.
\end{document}
\end{document}