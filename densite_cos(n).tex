\documentclass[fontsize=12pt,twoside=false,parskip=half, french]{scrartcl}
\usepackage[utf8]{inputenc}
\usepackage{babel}
\usepackage[T1]{fontenc}
\usepackage{amsmath, amssymb, stmaryrd}
\usepackage[amsmath]{ntheorem}

\title{Densité de $\enstq{\cos(n)}{n \in \N}$ dans $\interv[-1; 1]$}
\date{}
\author{}

\input{../outils}
\setcounter{secnumdepth}{0}
\begin{document}
\maketitle
   \begin{Theoreme}
      La suite de terme général $(\cos n)$ est dense dans $\interv[-1; 1]$.
   \end{Theoreme}
   \subsection{Démonstration}
      Pour démontrer le résultat, nous allons montrer que $H = \Z + 2 \pi \Z$ est un sous-groupe dense 
      de $(\R, +)$ puis utiliser la continuité et la périodicité de $\cos$.
      
      Les sous-groupes de $(\R, +)$ sont soit denses, soit de la forme $a\Z$. Supposons $H$ de la forme $a\Z$ avec 
      $a \in \R$. Alors, puisque $\Z \subset H = a\Z$, alors $a \in \Q$. 
      
      D’un autre côté, $2\pi\Z \subset H = a\Z$, avec $a$ rationnel, donc $\pi \in a\Q$. ABSURDE car $\pi$ est irrationnel.
      Donc $H$ est dense dans $\R$.
      
      Montrons maintenant le résultat. Soit $x \in \interv[-1; 1]$. Il existe $\theta \in \interv[0; \pi]$, tel que 
      $\cos \theta = x$ et puisque $H$ est dense dans $\R$, il existe deux suites d’entiers $\suite{a}$ et $\suite{b}$
      telle que $a_n + 2\pi b_n \to \theta$.
      On a 
      \[
         \cos(a_n) = \cos(a_n + 2\pi b_n) \to cos(\theta) = x.
      \]
      
      Donc, pour tout $x \in \interv[-1; 1]$ il existe une suite d’entiers $\suite{a}$ telle que $\cos(a) \to x$.
\end{document}