\documentclass[fontsize=12pt,twoside=false,parskip=half, french]{scrartcl}
\usepackage[utf8]{inputenc}
\usepackage{babel}
\usepackage[T1]{fontenc}
\usepackage{amsmath, amssymb, stmaryrd}
\usepackage[amsmath]{ntheorem}

\title{Théorème des restes chinois}
\date{}
\author{}

\input{../outils}
\setcounter{secnumdepth}{0}
\begin{document}
\maketitle
   \begin{Theoreme}[Isomorphisme d’anneaux]
      Soit $m$ et $n$ deux entiers premiers entre eux, alors l’application $\pi$
      définie par
      \[
         \fonction{\pi}{\Zquotient{mn}}{\Zquotient{n} \times \Zquotient{m}}{\classe{k}}{\classeUn{k}, \classeDeux{k}}
      \]
      est un isomorphisme d’anneaux.
   \end{Theoreme}
   \subsection{Démonstration}
      $\pi$ est bien définie puisque si $k = l [mn]$ alors $k = l[m]$ et $k = l[n]$, et donc $\classe{k} = \classe{l}$,
      $\classeUn{k} = \classeUn{l}$ et $\classeDeux{k} = \classeDeux{l}$ et c’est bien un morphisme d’anneaux.
      
      Il y a égalité des cardinaux de l’ensemble de départ et de celui d’arrivée, donc il nous suffit de montrer que
      $\pi$ est injective pour avoir que c’est un isomorphisme.
      
      Soit $\classe{x} \in \ker \pi$, alors $\classeUn{x} = \classeUn{0}$ et $\classeDeux{x} = \classeDeux{0}$, donc
      $m \mid x$ et $n \mid x$, d’où $mn \mid x$ puisque $m$ et $n$ sont premiers entre eux, donc $\classe{x} = \classe{0}$,
      d’où $\ker \pi = \left\{\classe{0}\right\}$. Donc $\pi$ est injective.
      
      Le résultat est démontré.
      
      Ce résultat permet de résoudre les systèmes du type
      \[
         \begin{systeme}
            x \equiv a &\mod n\\
            x \equiv b &\mod m
         \end{systeme}
      \]
\end{document}