\documentclass[fontsize=12pt,twoside=false,parskip=half, french]{scrartcl}
\usepackage[utf8]{inputenc}
\usepackage{babel}
\usepackage[T1]{fontenc}
\usepackage{amsmath, amssymb, stmaryrd}
\usepackage[amsmath]{ntheorem}

\title{Lemme de Borel-Cantelli}
\date{}
\author{}

\input{../outils}
\setcounter{secnumdepth}{0}
\begin{document}
\maketitle
   \begin{Theoreme}[Borel-Cantelli]
      Soit $\suite{A}$ une suite d’évènements telle que
      \[
         \sum_{k = 0}^{+\infty}P(A_n) < + \infty.
      \]
      Alors,
      \[
         P\left(\lim \sup A_n\right) = 0.
      \]
   \end{Theoreme}
   \subsection{Démonstration}
      % Posons
      % \[
         % B_n = \bigcup_{k \geq n} A_k.
      % \]
      % Alors, $B_n$ est une suite décroissante pour l’inclusion, 
      % le \emph{théorème de continuité décroissante} donne alors que
      % \[
         % P(\bigcap_{n \geq 0} B_n) = \lim_{n \to +\infty} P(B_n)
      % \]
      On a pour tout entier $k$,
      \[
         P\left(\lim \sup A_n\right) \leq P\left(\bigcup_{n \geq k} A_n\right) \leq \sum_{n \geq k} P(A_n). 
      \]
      Cette somme tend vers $0$ quand $k \to \infty$ (car reste d’une série convergente). Par suite, on a bien
      \[
          P(\lim \sup A_n) = 0.
      \]
\end{document}
