\documentclass[fontsize=12pt,twoside=false,parskip=half, french]{scrartcl}
\usepackage[utf8]{inputenc}
\usepackage{babel}
\usepackage[T1]{fontenc}
\usepackage{amsmath, amssymb, stmaryrd}
\usepackage[amsmath]{ntheorem}

\title{Bijection sous-groupe engendré par un élément}
\date{}
\author{}

\input{../outils}
\setcounter{secnumdepth}{0}
\begin{document}
\maketitle
   \begin{Theoreme}
      Soit $G$ un groupe et $g \in G$. Si $g$ est d’ordre $n$, alors 
      \[
         \langle g \rangle \simeq \varZquotient{n}.
      \]
   \end{Theoreme}
   \subsection{Démonstration}
      Définissons le morphisme
      \[
         \fonction{\phi}{\varZquotient{n}}{G}{\classe{m}}{g^m}
      \]
      On vérifie que c’est bien un morphisme et par définition de $g$ 
      (il est d’ordre $n$), $\phi$ est surjectif.
      
      Si $\phi(\classe{m}) = e_G$, la division euclidienne de $m$ par $n$ donne
      $m = qn + r$ avec $0 \leq r < n$.
      
      Ainsi, si $\phi(\classe{m}) = 0$, alors $g^{qn + r} = 0$, c’est-à-dire
      $g^r = 0$ d’où $r = 0$ (par définition de l’ordre).On en déduit que 
      $\ker \phi = \singleton{\classe{0}}$, $\phi$ est injectif.
      
      Par suite, $\phi$ est un isomorphisme et $\langle g \rangle \simeq \varZquotient{n}$.
      
      PS : ainsi, tout groupe cyclique est isomorphe à $\varZquotient{n}$.
\end{document}