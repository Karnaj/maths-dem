\documentclass[fontsize=12pt,twoside=false,parskip=half, french]{scrartcl}
\usepackage[utf8]{inputenc}
\usepackage{babel}
\usepackage[T1]{fontenc}
\usepackage{amsmath, amssymb, stmaryrd}
\usepackage[amsmath]{ntheorem}

\title{Suite de sinus itéré}
\date{}
\author{}

\input{../outils}
\setcounter{secnumdepth}{0}
\begin{document}
\maketitle
   \begin{Theoreme}
      Soit $\suite{u}$ définie par $u_0 \in \interv[0; \frac{pi}{2}]$ et pour tout entier $n$, $u_{n + 1} = \sin(u_n)$.
      \[
         u_n \sim \sqrt{\frac{\pi}{3}}.
      \]
   \end{Theoreme}
   \subsection{Démonstration}
      Nous allons d’abord montrer que $u_n$ tend vers 0. Posons $I = \interv[0; \frac{\pi}{2}]$.
      $I$ est stable par la fonction $\sin$, donc pour tout entier $n$, $u_n \in I$. De plus,
      pour tout $x \in I$, $\sin(x) \geq x$, donc $(u_n)$ est décroissante. Elle est 
      décroissante minorée (par $0$), donc elle converge vers un réel $l$. $l$ vérifie 
      $\sin(l) = l$, donc $l = 0$.
      
      Cherchons maintenant l’équivalent. On a 
      \begin{align*}
         \frac{1}{u_{n + 1}^2} - \frac{1}{u_n^2} = \frac{1}{\sin^2 u_n} - \frac{1}{u_n^2} &= \frac{1}{\left(u_n - \frac{u_n^3}{6} + O(u_n^2)\right)^2} - \frac{1}{u_n^2}\\
                                                 &= \frac{1}{u_n^2} \left( \frac{1}{1 - \frac{u_n^2}{3} + O(u_n^3)} - 1\right)\\
                                                 &= \frac{1}{u_n^2} \left( \frac{u_n^2}{3} + O(u_n^3) \right)\\
                                                 &= \frac{1}{3} + O(u_n) \sim \frac{1}{3}.                                                
      \end{align*}
      
\end{document}