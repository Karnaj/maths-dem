\documentclass[fontsize=12pt,twoside=false,parskip=half, french]{scrartcl}
\usepackage[utf8]{inputenc}
\usepackage{babel}
\usepackage[T1]{fontenc}
\usepackage{amsmath, amssymb, stmaryrd}
\usepackage[amsmath]{ntheorem}

\title{Critère logarithmique}
\date{}
\author{}

\input{../outils}
\setcounter{secnumdepth}{0}
\begin{document}
\maketitle
   \begin{Theoreme}[Critère logarithmique]
      Soient $\suite{u}$ et $\suite{v}$ deux suites à valeurs positives telles que 
      \[
         \frac{u_{n + 1}}{u_n} \leq \frac{v_{n + 1}}{v_n}.
      \]
      Alors la convergence de la série des $v_n$ entraîne celle de la série des $u_n$.
   \end{Theoreme}
   \subsection{Démonstration}
      Supposons que la série des $v_n$ converge. Soit $n \in \N$. On a
      \[
         \ln \left( \sum_{k = 0}^n \frac{u_{k + 1}}{u_k} \right) \leq \ln \left( \sum_{k = 0}^n \frac{v_{k + 1}}{v_k} \right)
      \]
      c’est-à-dire
      \[
        \ln \left(\prod_{k = 0}^n \frac{u_{k + 1}}{u_k} \right) \leq \ln \left(\prod_{k = 0}^n\frac{v_{k + 1}}{v_k} \right).
      \]
      Par télescopage, on obtient
      \[
         \ln \left(\frac{u_{n + 1}}{u_0}\right) \leq \ln\left(\frac{v_{n + 1}}{v_0}\right)
      \]
      Et donc 
      \[
         \frac{u_{n + 1}}{u_0} \leq \frac{v_{n + 1}}{v_0}.
      \]
      On a alors
      \[
         0 \leq u_{n + 1} \leq \frac{v_{n + 1} u_0}{v_0}
      \]
      d’où \emph{par comparaison de séries à termes positifs}, la série des $u_n$ converge.
\end{document}