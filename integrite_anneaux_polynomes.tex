\documentclass[fontsize=12pt,twoside=false,parskip=half, french]{scrartcl}
\usepackage[utf8]{inputenc}
\usepackage{babel}
\usepackage[T1]{fontenc}
\usepackage{amsmath, amssymb, stmaryrd}
\usepackage[amsmath]{ntheorem}

\title{Intégrité d'un anneau de polynômes}
\date{}
\author{}

\input{../outils}
\setcounter{secnumdepth}{0}
\begin{document}
\maketitle 
   \begin{Theoreme}
       Soit $A$ un anneau. $A$ est intègre si et seulement si $A[X]$ est intègre.
   \end{Theoreme}
   \subsection{Démonstration}
      Ce résultat demande surtout de se ramener aux définitions.
      
      Les éléments de $A$ sont des éléments de $A[X]$ (ce sont les polynômes constants),
      donc pour $a, b$ dans $A$, $a$ et $b$ sont aussi dans $A[X]$ d'où $ab = 0$
      implique $a = 0$ ou $b = 0$ donc $A$ est intègre.
      
      Supposons maintenant $A$ intègre. Soit $P$ et $Q$ deux polynômes. Montrons
      que $PQ \neq 0$. Si on considère $n$ et $m$ les degrés de $P$ et $Q$ et 
      en notant $P_i$ et $Q_j$ les coefficients des polynômes, on a 
      $P_n$ et $Q_m$ qui sont des éléments de $A$ non nuls,  avec $PQ_{n + m} = P_nQ_m$
      d'où $PQ_{n + m} \neq 0$ puisque $A$ est intègre. Par suite, $PQ$ est non nul.
\end{document}