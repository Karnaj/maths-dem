\documentclass[fontsize=12pt,twoside=false,parskip=half, french]{scrartcl}
\usepackage[utf8]{inputenc}
\usepackage{babel}
\usepackage[T1]{fontenc}
\usepackage{amsmath, amssymb, stmaryrd}
\usepackage[amsmath]{ntheorem}

\title{Nombre de mots de Dyck}
\date{}
\author{}

\input{../outils}
\setcounter{secnumdepth}{0}
\begin{document}
\maketitle
   Soit $n \in \N$. Un mot de Dyck est un mot $w$ composé de $n$ \texttt{(} 
   et $n$ \texttt{)} telles que lorsqu'on parcourt $w$, il y a toujours plus de 
   parenthèses ouvertes que fermées. Cela revient à être bien parenthésé. Ainsi,
   \texttt{(())()} est un mot de Dyck, mais pas \texttt{())(}.
   
   \begin{Theoreme}[Nombre de mots de Dyck]
      Soit $n \in \N$. Le nombre de mots de Dyck de longueur $2n$ est
      \[
         C_n = \binom{2n}{n} - \binom{2n}{n - 1}.
      \]
   \end{Theoreme}
   \subsection{Démonstration}
      Le nombre de mots de longueur $2n$ avec autant de \texttt{(} que de 
      \texttt{)} est $\binom{2n}{n}$ (on choisit la place des $n$ \texttt{(}).
      
      Comptons le nombre de ces mots qui ne sont pas de Dyck. Considérons le 
      premièr \texttt{)} qui pose problème \ie{} qui n'est pas associée
      à un \texttt{(}. Avant lui il y a autant de \texttt{(} que de \texttt{)} ; 
      après lui, il y a un \texttt{(} de plus que de \texttt{)}.
      
      On échange toutes les parenthèses après le premier \texttt{)} qui pose
      problème par la parenthèses opposées. On transforme ainsi tout mot avec
      autant de \texttt{(}  que de \texttt{)} et qui n'est pas de Dyck en un mot
      avec $n - 1$ \texttt{(} et $n + 1$ \texttt{)}. Cette opération est une 
      bijection ; le nombre de mots à $n$ parenthèses ouvrantes et $n$ 
      parenthèses fermantes qui ne sont pas de Dyck est donc $\binom{n - 1}{2n}$
      (on choisit les places des  $n - 1$ parenthèses ouvrantes).
      Ainsi, le nombre de mot de Dyck est bien
      \[
         C_n = \binom{2n}{n} - \binom{2n}{n - 1}.
      \] 
      
      PS : ces nombre sont les \emph{nombre de Catalans}. Ils vérifient
      la relation de récurrence suivante, qui donne également un moyen de calculer
      $C_n$.
      \[
      \begin{cases}
         C_0 &= 1\\
         C_{n + 1} &= \sum_{i = 0}^n C_iC_{n - i - 1}.
      \end{cases}
      \]
\end{document}