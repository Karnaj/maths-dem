\documentclass[fontsize=12pt,twoside=false,parskip=half]{scrartcl}
\usepackage[utf8]{inputenc}
\usepackage[french]{babel}
\usepackage[T1]{fontenc}
\usepackage{amsmath, amssymb, stmaryrd}
\usepackage[amsmath]{ntheorem}

\title{Indicatrice d’Euler}
\date{}
\author{}

\input{../outils_maths}
\setcounter{secnumdepth}{0}
\begin{document}
\maketitle
   Notons $\phi$ la fonction indicatrice d’Euler 
   % la fonction $\phi$ de $\N^*$ dans
   % $\N^*$ définie par
   % \[
      % \phi(n) = \card{\enstq{k \in \intervalleEntier{1}{n}}{k \wedge n = 1}}.
   % \]
   % On a bien évidémment que $\phi(n)$ est le nombre d’éléments inversibles 
   % de l’anneau $\varZquotient{n}$ et le nombre de générateurs du groupe 
   % $\varZquotient{n}$. 
   \begin{Theoreme}
      Soit $p$ premier et $\alpha$ un entier non nul, alors
      \[
         \phi(p) = p^\alpha - p^{\alpha - 1}. 
      \]
      Si $p_1^{\alpha_1} \ldots p_N^{\alpha_N}$ est la décomposition en éléments
      premiers d’un entier $n$, alors
      \[
         \phi(n) = n \prod_{k = 1}^N \left(1 - \frac{1}{p_k}\right)
      \]
   \end{Theoreme}
   \subsection{Démonstration}
      Si $p$ est premier, les seuls nombres de $\intervalleEntier{1}{p^\alpha}$
      qui ne sont pas premiers avec $p^\alpha$ sont les multiples de $p$. Il y 
      en a $p^{\alpha - 1}$. On en déduit que le nombre d’éléments inversibles est
      \[
         \phi(p) = \card \intervalleEntier{1}{n} - p^{\alpha - 1} = p^\alpha - 
         p^{\alpha - 1}.
      \]
      Le \emph{lemme chinois} permet d’avoir que si $n$ et $m$
      sont premiers entre eux, alors
      \[
         \phi(nm) = \phi(n)\phi(m) 
      \]
      Grâce à cela, si $n = p_1^{\alpha_1} \ldots p_N^{\alpha_N}$, alors,
      puisque les $p_k^{\alpha_k}$ sont premiers entre eux,
      \begin{align*}
         \phi(n) = \prod_{k = 1}^N \phi(p_k^{\alpha_k})
                 &= \prod_{k = 1}^N p_k^{\alpha_k} - p_k^{\alpha_k - 1}\\
                 &= \prod_{k = 1}^N p_k^{\alpha_k} \left(1 - \frac{1}{p_k^{\alpha_k}}\right)\\
                 &= \prod_{k = 1}^N p_k^{\alpha_k} \prod_{k = 1}^N\left(1 - \frac{1}{p_k^{\alpha_k}}\right)\\
                 &= n \prod_{k = 1}^N \left(1 - \frac{1}{p_k}\right).
      \end{align*}
      On obtient bien le résultat.
\end{document}
