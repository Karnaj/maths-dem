\documentclass[fontsize=12pt,twoside=false,parskip=half, french]{scrartcl}
\usepackage[utf8]{inputenc}
\usepackage{babel}
\usepackage[T1]{fontenc}
\usepackage{amsmath, amssymb, stmaryrd}
\usepackage[amsmath]{ntheorem}

\title{Théorème de Cayley-Hamilton}
\date{}
\author{}

\input{../outils}
\setcounter{secnumdepth}{0}
\begin{document}
\maketitle
   \begin{Theoreme}[Théorème de Cayley-Hamilton]
      Soit $f$ un endormorphisme de $E$. Alors
      \[
         \chi_f(f) = 0.
      \]
   \end{Theoreme}
   \subsection{Démonstration}
      Soit $x$ non nul de $E$. Il existe un plus petit entier $p > 0$ tel que $(x, f(x), \ldots, f^p(x))$ soit liée
      (en effet, $(x)$ est libre et $(x, f(x), \ldots, f^{n + 1}(x))$ est liée. Donc il existe 
      $(a_0, \ldots, a_{p - 1}) \in \R^{p}$ tel que
      \[
         f^p(x) = a_0x + a_1 f(x) + \ldots + a_{p - 1}f^{p - 1}(x).
      \]
      La famille $(x, f(x), \ldots, f^{p - 1}(x))$ est libre, complétons-la en une base de $E$. Dans cette base, 
      la matrice de $f$ dans cette base est de la forme
      \[
         \begin{pmatrix}
            A & B\\
            0 & C
         \end{pmatrix}
      \]
      Avec $A$ la matrice compagnon
      \[
         A = 
         \begin{pmatrix}
            0      & \ldots & \ldots & 0      & a_0\\
            1      & 0      &        & \vdots & a_1\\
            0      & 1      & \ddots & \vdots & \vdots\\
            \vdots & \ddots & \ddots & 0      & a_{n - 2}\\
            0      & \ldots & 0      & 1      & a_{n - 1}
         \end{pmatrix}
      \]
      Le polynôme caractéristique de $f$ est $\chi_f = \chi_A\chi_C$. Or le polynôme caractéristique de $A$ est
      \[
         \chi_A = (-1)^p(X^p - (a_{p - 1}X^{p - 1} + \ldots + a_0))
      \]
      Par suite, pour tout $x \in E$,
      \[
         \chi_A(f)(x) = (-1)^p(f^p(x) - (a_{p - 1}f^{p - 1}(x) + \ldots + a_0x)) = 0.
      \]
      On a bien que $f$ annule son polynôme caractéristique.
\end{document}
