\documentclass[fontsize=12pt,twoside=false,parskip=half, french]{scrartcl}
\usepackage[utf8]{inputenc}
\usepackage{babel}
\usepackage[T1]{fontenc}
\usepackage{amsmath, amssymb, stmaryrd}
\usepackage[amsmath]{ntheorem}

\title{Théorème de Carathéodory}
\date{}
\author{}

\input{../outils}
\setcounter{secnumdepth}{0}
\begin{document}
\maketitle
   \begin{Theoreme}[Théorème de Carathéodory]
      Soit $n \in \N$ et $A \subset \R^n$. L'enveloppe convexe de $A$ est 
      l'ensemble $\Gamma$ des points s'écrivant comme combinaison convexe
      d'au plus $n + 1$ éléments de $A$.
   \end{Theoreme}
   \subsection{Démonstration}
      Il est clair que $\Gamma \subset \Conv(A)$. Pour l'inclusion inverse, 
      montrons que toute combinaison convexe de $N > n + 1$ éléments de $A$ peut
      s'écrire comme combinaison convexe de $N - 1$ éléments. Soit 
      \[
        x = \sum_{k = 0}^N \lambda_kx_k
      \] 
      une combinaison convexe de $N > n + 1$ éléments de $A$. Les $x_k$
      sont $N > n + 1$ points dans $\R^n$, donc il existe $N$ scalaires $\mu_k$
      non tous nuls tels que
      \[
         \sum_{k = 0}^N \mu_kx_k = 0 \text{ et }  \sum_{k = 0}^N \mu_k = 0.
      \]
      Posons $I = \enstq{k \leq N}{\mu_k > 0}$. Les $\mu_k$ sont non tous nuls,
      donc $I$ est non vide. Posons
      \[
        \lambda = \min \enstq{\frac{\lambda_k}{\mu_k}}{k \in I} \text { et }
        \lambda'_k = \lambda_k - \lambda\mu_k.
      \]
      On a alors
      \begin{align*}
        \sum_{k = 1}^N \lambda'_k x_k 
          &= \sum_{k = 1}^N (\lambda_k - \lambda\mu_k) x_k\\
          &= x - \lambda \sum_{k = 1}^N \mu_k x_k\\
          &= x.
      \end{align*}
      De plus, en notant $i$ l'indice tel que $\lambda = \frac{\lambda_i}{\mu_k}$,
      on a $\lambda'_i = 0$, donc on a écrit $x$ comme combinaison linéaire de
      $N - 1$ points.
      
      Donc on a bien $\Gamma = \Conv(A)$.
\end{document}