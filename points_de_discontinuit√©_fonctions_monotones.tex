\documentclass[fontsize=12pt,twoside=false,parskip=half]{scrartcl}
\usepackage[utf8]{inputenc}
\usepackage[french]{babel}
\usepackage[T1]{fontenc}
\usepackage{amsmath, amssymb, stmaryrd}
\usepackage[amsmath]{ntheorem}

\title{Points de discontinuité de fonction monotone}
\date{}
\author{}

\input{../outils_maths}
\setcounter{secnumdepth}{0}
\begin{document}
\maketitle
   \begin{Theoreme}
      Soit $f$ une fonction monotone de $I = [a, b]$ dans $\R$. Alors l’ensemble des points de discontinuité de $f$ est 
      dénombrable.
   \end{Theoreme}
   \subsection{Démonstration}
      Quite à considérer $-f$, on peut supposer $f$ croissante.
      On va noter
      \[
         A_n = \enstq{x \in I}{f(x^+) - f(x^-) > \frac{1}{n}}.
      \]
      Montrons que $A_n$ est fini pour tout entier $n$. Considérons $p$ réel distincs de $A_n$, $x_1 < x_2 < \ldots < x_p$ 
      et considérons $p + 1$ réels $y_1, \ldots, y_p$ tels que
      \[
         y_0 < x_1 < y_1 < \ldots < x_p < y_p.
      \]
      On a alors pour tout $i \in \intervalleEntier{1}{n}$, $f(y_i) - f(y_{i- 1}) \geq \frac{1}{n}$. Ainsi,
      \[
         f(b) - f(a) \geq f(y_p) - f(y_0) = \sum_{i = 1}^n f(y_i) - f(y_{i - 1}) \geq \frac{p}{n}.
      \]
      On a $p \leq n(f(b) - f(a))$, ce qui permet de majorer le cardinal de $A_n$, montrant qu’il est fini.
      
      L’ensemble des points de discontinuité de $f$ est l’union des $A_n$ et est donc dénombrable en tant que réunion
      finie d’ensembles finis.
      
      PS : ce résultat s’étend aux fonctions monotones de $\R$ dans $\R$. En effet, on a que l’ensemble $E_n$
           des points de discontinuité de $f$ sur $\interv[-n; n]$ est dénombrable ; l’ensemble des points de discontinuité
           de $f$ sur $\R$ étant la réunion des $E_n$, il est dénombrable en tant que réunion dénombrable d’ensemble dénombrable.
\end{document}