\documentclass[fontsize=12pt,twoside=false,parskip=half, french]{scrartcl}
\usepackage[utf8]{inputenc}
\usepackage{babel}
\usepackage[T1]{fontenc}
\usepackage{amsmath, amssymb, stmaryrd}
\usepackage[amsmath]{ntheorem}

\title{Dénombrement}
\date{}
\author{}

\input{../outils}
\setcounter{secnumdepth}{0}
\begin{document}
\maketitle
   \begin{Theoreme}
      L’ensemble des $n$-uplets $(a_1, \ldots, a_n)$ d’entiers naturels tels que $a_1 + \ldots + a_n = N$ a pour cardinal
      \[
         \binom{N + n - 1}{N}.
      \]
   \end{Theoreme}
   \subsection{Démonstration}
      Ce résultat peut être montré (laborieusement) par une récurrence. Plutôt que de la faire, nous allons écrire la
      somme $a_1 + \ldots + a_n$ comme une succession de « $1$ » et de « $+$ ». Par exemple
      \begin{align*}
         2 + 0 + 1 + 3 &\text{ donne } 11 + + 1 + 111\\ 
         0 + 2 + 1 + 3 &\text{ donne } + 11 + 1 + 111\\ 
         0 + 2 + 3 + 0 &\text{ donne } + 11 + 111 + +        
      \end{align*}
      Cette écriture est faite avec $N + n - 1$ symboles ($N$ « 1 » et $n - 1$ « + »). Choisir un $n$-uplet 
      correspond à choisir la place des $N$ « 1 » (ou la place des $n - 1$ « + »). Par suite, le nombre de 
      $n$-uplets correspondant est
      \[
         \binom{N + n - 1}{N}.
      \]
      
      PS : ce résultat nous permet de connaître le cardinal de l'ensemble des $n$-upets $(a_1, \ldots, a_n)$ d'entiers naturels non-nuls de somme $N$. En effet, en posant
      \[
        f((a_1, \ldots, a_n) = (a_1 + 1, \ldots, a_n + 1)
      \]
      on a une bijection évidente entre l'ensemble des $n$-uplets d'entiers naturels de somme $N - n$ et l'ensemble des $n$-uplets d'entiers naturels non-nuls de somme $N$. Par suite,
      le cardinal de ce dernier ensemble est 
      \[
         \binom{N - 1}{N - n}.
      \]
\end{document}