\documentclass[fontsize=12pt,twoside=false,parskip=half]{scrartcl}
\usepackage[utf8]{inputenc}
\usepackage[french]{babel}
\usepackage[T1]{fontenc}
\usepackage{amsmath, amssymb, stmaryrd}
\usepackage[amsmath]{ntheorem}

\title{Inversibles de $\varZquotient{n}$}
\date{}
\author{}

\input{../outils_maths}
\setcounter{secnumdepth}{0}
\begin{document}
\maketitle
   \begin{Theoreme}[Inversibles de $\varZquotient{n}$]
      Soit $n \in \N$. Un élément $\classe{m}$ de $\varZquotient{n}$ est inversible si et seulement si
      $m \wedge n = 1$. 
   \end{Theoreme}
   \subsection{Démonstration}
      $\varZquotient{n}, +, \times)$ est un anneau commutatif. Soit $\classe{m} \in \varZquotient{n}$.
      
      $\classe{m}$ est inversible si et seulement si il existe $\classe{k} \in \varZquotient{n}$ tel que
      $\classe{m} \times \classe{k} = \classe{1}$, donc si et seulement si il existe $k \in \Z$ tel que
      \[
         km \equiv 1 \mod n.
      \]
      Ainsi, $\classe{m}$ est inversible si et seulement si il existe existe deux entiers $k$ et $l$ tel que
      \[
         km + ln = 1.
      \]
      Le \emph{théorème de Bézout} permet alors de dire que cela équivaut à $m \wedge n = 1$.
      
      PS : ceci permet de résoudre les équations $ax + b \equiv c \mod n$.
      \begin{align*}
         ax + b \equiv c \mod n &\iff \classe{a} \times \classe{x} + \classe{b} = \classe{c}\\
                                &\iff \classe{a} \times \classe{x} = \classe{c} - \classe{b}.
      \end{align*}
      Si $\classe{a}$ est inversible (donc si $a \wedge n$ = 1), on a
      \[
         \classe{x} = \classe{a}^{-1} \times \left(\classe{c} - \classe{b}\right).
      \]
      et l’ensemble des solutions est l’ensemble $\enstq{\classe{a}^{-1} \times \left(\classe{c} - \classe{b}\right) + nk}{k \in \Z}$.
      
      Sinon, il n’y a pas de solution.
\end{document}