\documentclass[fontsize=12pt,twoside=false,parskip=half, french]{scrartcl}
\usepackage[utf8]{inputenc}
\usepackage{babel}
\usepackage[T1]{fontenc}
\usepackage{amsmath, amssymb, stmaryrd}
\usepackage[amsmath]{ntheorem}

\title{Règle d'Abel}
\date{}
\author{}

\input{../outils}
\setcounter{secnumdepth}{0}
\begin{document}
\maketitle
   \begin{Theoreme}
      Soit $u_n$ une suite telle que $u_n = a_n v_n$ avec
         \begin{itemize}
            \item $\suite{a}$ une suite positive, décroissante de limite nulle.
            \item la série de terme général $v_n$ est bornée.
         \end{itemize}
      Alors, la série de terme général $u_n$ converge.
   \end{Theoreme}
   \subsection{Démonstration}
      Pour commencer, notons $S_n$ la suite des sommes partielles de $v_n$. On a
      qu’il existe $M$ tel que $\norme{S_n} < M$ pour tout $n$. De plus, on a
      avec une \emph{transformation d’Abel} que
      \begin{align*}
         \sum_{k = 0}^n u_k &= \sum_{k = 0}^n a_kv_k\\
                            &= a_0 v_0 + \sum_{k = 1}^n a_k\left(S_k - S_{k - 1}\right)\\
                            &= a_0 v_0 + \sum_{k = 1}^n a_k S_k - \sum_{k = 1}^n a_k S_{k - 1}\\
                            &= \sum_{k = 0}^n a_k S_k - \sum_{k = 0}^{n - 1} a_{k + 1} S_k\\
                            &= a_nS_n + \sum_{k = 0}^{n - 1} \left(a_{k - 1} - a_k\right) S_k.
      \end{align*}
      Montrons la convergence absolue de la première série.
      \[
         \sum_{k = 0}^{n - 1} \norme{\left(a_{k - 1} - a_k\right) S_k}
               \leq M \sum_{k = 0}^{n - 1} \norme{\left(a_{k - 1} - a_k\right)} = \left(a_0 - a_{n + 1}\right)M \leq a_0M.
      \]
      Et on a que le premier terme, $\left(a_nS_n\right)$ tend vers $0$ puisque
      $\suite{S}$ bornée et $\suite{a}$ décroissante de limite nulle.

      On en déduit bien la convergence de la série de terme général $u_n$.
\end{document}
