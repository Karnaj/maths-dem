\documentclass[fontsize=12pt,twoside=false,parskip=half, french]{scrartcl}
\usepackage[utf8]{inputenc}
\usepackage{babel}
\usepackage[T1]{fontenc}
\usepackage{amsmath, amssymb, stmaryrd}
\usepackage[amsmath]{ntheorem}

\title{Équivalent intégrale de Wallis}
\date{}
\author{}

\input{../outils}
\setcounter{secnumdepth}{0}
\begin{document}
\maketitle
   Soit $n \in \N$ et $I_n$ l’intégrale de Wallis
   \[
      I_n = \int_0^{\frac{\pi}{2}} \cos^n t \dt.
   \]
   \begin{Theoreme}
      On a en $+\infty$
      \[
         I_n \sim \sqrt{\frac{\pi}{2n}}.
      \]
   \end{Theoreme}
   \section{Démonstration}
      On a pour $n \in \N$ et $t \in \interv[0; \frac{\pi}{2}]$, 
      $0 \leq \cos^{n + 1} t \leq \cos^{n} t$, en intégrant cette inégalité, 
      on obtient que $\suite{I}$ est positive décroissante. Alors
      \[
         \frac{I_{n + 1}}{I_n} \leq \frac{I_{n + 1}}{I_n} \leq \frac{I_n}{I_n} = 1
      \]
      ce qui donne avec la relation de récurrence $nI_n = (n - 1)I_{n - 2}$,
      \[
         \frac{n + 1}{n + 2} \leq \frac{I_{n + 1}}{I_n} \leq 1.
      \]
      On en deduit que $I_n \sim I_{n + 1}$.
      
      De plus, $nI_nI_{n - 1}$ est constant. En effet, 
      \[
         nI_n = (n - 1)I_{n - 2} \implies nI_nI_{n - 1} = (n - 1)I_{n - 1}I_{n - 2}. 
      \]
      Elle est donc égale à $I_1I_0 = \frac{\pi}{2}$. On a alors
      \[
         \frac{\pi}{2} = nI_nI_{n - 1}^2 \sim nI_n^2. 
      \]
      Par suite, on a bien 
      \[
         I_n \sim \sqrt{\frac{\pi}{2n}}.
      \]
      
      PS : certaines opérations sont licites car $I_n > 0$, ce qui a bien été précisé.
\end{document}