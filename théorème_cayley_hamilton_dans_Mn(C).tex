\documentclass[fontsize=12pt,twoside=false,parskip=half, french]{scrartcl}
\usepackage[utf8]{inputenc}
\usepackage{babel}
\usepackage[T1]{fontenc}
\usepackage{amsmath, amssymb, stmaryrd}
\usepackage[amsmath]{ntheorem}

\title{Théorème de Cayley-Hamilton}
\date{}
\author{}

\input{../outils}
\setcounter{secnumdepth}{0}
\begin{document}
\maketitle
   \begin{Theoreme}[Théorème de Cayley-Hamilton]
      Soit $f$ un endormorphisme de $E$ un $\C$-espace vectoriel de dimension finie $n$. Alors
      \[
         \chi_f(f) = 0.
      \]
   \end{Theoreme}
   \subsection{Démonstration}
      Soit $A \in \M_n(\C)$ la matrice de $f$ dans une base. Si $A$ est diagonalisable, il existe une matrice inversible $Q$ et une matrice diagonale $D = \Diag{\lambda_1, \ldots, \lambda_n}$ telles que $A = P^-1DP$. On a 
      \[  
         \chi_A(A)
      \]
      
      On a donc le résultat pour les matrices diagonales. La densité de l'ensemble des matrices diagonables dans $M_n(\C)$ nous donne alors le résultat puisque $M \mapsto \chi_M(M)$ est continue de $\M_n(\C)$ dans $\M_n(\C)$ (ses composantes sont des fonctions polynomiales).
\end{document}
