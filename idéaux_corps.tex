\documentclass[fontsize=12pt,twoside=false,parskip=half, french]{scrartcl}
\usepackage[utf8]{inputenc}
\usepackage{babel}
\usepackage[T1]{fontenc}
\usepackage{amsmath, amssymb, stmaryrd}
\usepackage[amsmath]{ntheorem}

\title{Idéaux d’un corps}
\date{}
\author{}

\input{../outils}
\setcounter{secnumdepth}{0}
\begin{document}
\maketitle
   \begin{Theoreme}
      Soit $E$ un anneau. $E$ est un corps si et seulement si ses seuls idéaux sont ${0}$ et lui-même.
   \end{Theoreme}
   \subsection{Démonstration}
      Soit $E$ un corps et soit $I$ un idéal non nul de $A$. Soit $x \in E$ non nul, alors $x$ est
      inversible. Par suite, $xx^{-1} = 1 \in I$, d’où $I = E$.
      
      Réciproquement, si les seuls idéaux de $E$ sont ${0}$ et lui-même, montrons que tout $x$ 
      non nul de $E$ est inversible. Soit $x$ non nul. L’idéal engendré par $x$ n’est pas nul, 
      donc c’est $E$ tout entier. Par suite, $1$ est dans cet idéal, d’où il existe $y \in E$
      tel que $xy = 1$, d’où le résultat. 
\end{document}
