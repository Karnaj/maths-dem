\documentclass[fontsize=12pt,twoside=false,parskip=half, french]{scrartcl}
\usepackage[utf8]{inputenc}
\usepackage{babel}
\usepackage[T1]{fontenc}
\usepackage{amsmath, amssymb, stmaryrd}
\usepackage[amsmath]{ntheorem}

\title{Dimension du dual en dimension finie}
\date{}
\author{}

\input{../outils}
\setcounter{secnumdepth}{0}
\begin{document}
\maketitle
   \begin{Theoreme}[Dimension du dual]
      Soit $E$ un $K$-espace vectoriel de dimension $n$. $E^*$ est aussi de 
      dimension $n$.
   \end{Theoreme}
   \subsection{Démonstration}
      Soit $\mathcal{B} = (e_1, \ldots, e_n)$ une base de $E$. On pose $e_i^*$ la
      forme linéaire tel que $e_i^*(e_j) = \delta_{i, j}$, c'est-à-dire que pour
      $x = (x_1, \ldots, x_n)$ on a
      \[
        e_i^*(x) = x_i.
      \]
      Ce sont bien des formes linéaires (
      $e_i^*(\lambda x + y) = \lambda x_i + y_i = \lambda e_i^*(x) + e_i^*(y)$).
      
      Montrons que les $e_i^*$ forment une famille libre de $E^*$. Soit 
      $(\lambda_1, \ldots, \lambda_n) \in K$ tel que
      \[
        f = \sum_{i = 0}^n \lambda_i e_i^* = 0.
      \]
      Pour tout $i \in \intervalleEntier{1}{n}$, on a $f(e_i) = \lambda_i e_i$
      d'où $\lambda_i = 0$. Donc on a la liberté de la famille.
      
      Montrons qu'elle est génératrice. Soit $f \in E^*$.
      Pour tout $x = (x_1, \ldots, x_n) \in E$, on a 
      \[
        f(x) = \sum_{i = 0}^n x_i f(e_i) = \sum_{i = 0}^n e_i^*(x) f(e_i)
             = \sum_{i = 0}^n f(e_i) e_i^*(x).
      \]
      Donc
      \[
        f = \sum_{i = 0}^n f(e_i) e_i^*
      \]
      et les $e_i$ forment donc bien une famille génératrice de $E^*$.
      
      Donc, $\dim E^* = \dim E$. La base construite est appelée 
      \emph{base antéduale} de $\mathcal{B}$. On a
      \begin{alignat*}{3}
        \forall f \in E^*, f = \sum_{i = 0}^n f(e_i) e_i^* 
        & \quad & \text{ et } & \quad &
        \forall x \in E^*, x = \sum_{i = 0}^n e_i^*(x) e_i.
      \end{alignat*}
\end{document}