\documentclass[fontsize=12pt,twoside=false,parskip=half, french]{scrartcl}
\usepackage[utf8]{inputenc}
\usepackage{babel}
\usepackage[T1]{fontenc}
\usepackage{amsmath, amssymb, stmaryrd}
\usepackage[amsmath]{ntheorem}

\title{Groupe spécial linéaire d’ordre $n$}
\date{}
\author{}

\input{outils}
\setcounter{secnumdepth}{0}
\begin{document}
\maketitle
   Ici, $\K = \R$ ou $\C$ et $n > 0$.
   \begin{Theoreme}[Groupe spécial linéaire d’ordre $n$]
      L’ensemble $\SL_n(\K)$, des matrices de déterminant 1 est un sous-groupe
      fermé d'intérieur vide de $(\GL_n(\K), \times)$.
   \end{Theoreme}
   \subsection{Démonstration (limites de suite)}
      On a bien $\groupeOrthogonal_n(\R) \subset \GL_n(\R)$. Montrons que
      c’est un sous-groupe.
      \begin{enumerate}
         \item $I_n \in \SL_n(\K)$.
         \item Soit $(A, B) \in \SL_n(\K)^2$, $\det(AB) = \det(A)\det(B) = 1$,
               donc $\SL_n(\K)$ est stable parfois.
         \item Soit $A \in \SL_n(\R)$, $\det\left(A^{-1}\right) = \frac{1}{\det(M)} = 1$
               donc $A^{-1} \in \SL_n(\K)$.         
      \end{enumerate}
      Donc $\SL_n(\R)$ est un sous-groupe de $(\GL_n(\R), \times)$.
      
     En tant qu’image réciproque du fermé $\ens{1}$ par l’application continue 
     $\det$, il est fermé dans $\M_n(\K)$.
     
     Pour montrer qu'il est d'intérieur vide, nous allons montrer que toute matrice de 
     $\SL_n(\K)$ est limite d'une suite de matrice de $\M_n(\K) \subset \SL_n(\K)$.
     et est donc dans l'l'adhérence de $\M_n(\K) \subset \SL_n(\K)$ (et donc
     n'est pas dans l'intérieur de $\SL_n(\K)$.  
     
     Soit $M \in \SL_n(\K)$. La suite définie par $M_k = M - 2^k I_n$ converge 
     vers $M$. La suite $\suite(\det(M_k)$ ne peut prendre la valeur $1$ qu'un 
     nombre fini de fois (sinon $\chi_M$ est constant, or il est degré $n$). 
     Donc il existe $k$ à partir duquel $M_n \not\in \SL_n(\K)$, d'où $M$ 
     est limite d'une suite de matrice de $\M_n(\K) \subset \SL_n(\K)$. 
     
     On a bien $\SL_n(\K)$ fermé d'intérieur non vide.
     
     PS : On a $\groupeOrthogonal_n(\K) \subset \SL_n(\K) \cup \det^{-1}\left(\ens{-1}\right)$
     et on montre de manière similaire que $\det^{-1}\left(\ens{-1}\right)$ est
     d'intérieur vide d'où $\groupeOrthogonal_n(\K)$ est d'intérieur vide. En 
     fait, $\groupeOrthogonal_n(\K)$ est un sous-groupe compact de $(\GL_n(\K), \times)$. 
\end{document}