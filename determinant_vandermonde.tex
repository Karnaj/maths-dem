\documentclass[fontsize=12pt,twoside=false,parskip=half, french]{scrartcl}
\usepackage[utf8]{inputenc}
\usepackage{babel}
\usepackage[T1]{fontenc}
\usepackage{amsmath, amssymb, stmaryrd}
\usepackage[amsmath]{ntheorem}

\title{Déterminant de Vandermonde}
\date{}
\author{}

\input{../outils}
\setcounter{secnumdepth}{0}
\begin{document}
\maketitle
   Pour $(a_1, a_2, \ldots, a_n) \in \K^n$, on pose
   \[
      V_n(a_1, \ldots, a_n) =
      \begin{vmatrix}
         1      & a_1    & a_1^2  & \ldots & a_1^{n - 1} \\
         \vdots & \vdots & \vdots & \vdots & \vdots \\
         1      & a_n    & a_n^2  & \ldots & a_n^{n - 1} \\
      \end{vmatrix}
   \]
   \begin{Theoreme}[Déterminant de Vandermonde]
      \[
         V_n(a_1, \ldots, a_n) = \prod_{1 \leq i < j \leq n} (a_i - a_j)
      \]
   \end{Theoreme}
   \subsection{Démonstration (récurrence sur $n \in \N^*$)}
      Le résultat est vrai pour $n = 1$. Soit $n \in \N^*$. On suppose le résultat vrai pour $n$.

      Soit $(a_1, \ldots, a_{n + 1}) \in \K^{n + 1}$. S’il y a $i \neq j$ tel que $a_i = a_j$, deux lignes
      sont identiques alors le déterminant est nul,
      \[
         V_{n + 1}(a_1, \ldots, a_n, a_{n + 1}) = 0 = \prod_{1 \leq i < j \leq n + 1} (a_i - a_j).
      \]
      Sinon, on pose $P(x) = V_{n + 1}(a_1, \ldots, a_n, x)$. En développant par rapport à la dernière ligne,
      on a que $P$ est un polynôme de degré $n$ et de coefficient dominant $V_n(a_1, \ldots, a_n)$. De plus,
      $a_1, \ldots, a_n$ sont $n$ racines distinctes de $P$, donc on a
      \[
         P(x) = V_n(a_1, \ldots, a_{n - 1})\prod_{i = 1}^n (x - a_i).
      \]
      En utilisant l’hypothèse de récurrence et en évaluant $P$ en $a_{n + 1}$, on trouve
      \begin{align*}
         V_{n + 1}(a_1, \ldots, a_n, a_{n + 1})  &= \prod_{1 \leq i < j \leq n} (a_i - a_j)\prod_{i = 1}^n (a_{n + 1} - a_i)\\
                                                 &= \prod_{1 \leq i < j \leq n + 1} (a_i - a_j)
      \end{align*}
      et le résultat est établi.
\end{document}
