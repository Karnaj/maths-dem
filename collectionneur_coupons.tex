\documentclass[fontsize=12pt,twoside=false,parskip=half,french]{scrartcl}
\usepackage[utf8]{inputenc}
\usepackage{babel}
\usepackage[T1]{fontenc}
\usepackage{amsmath, amssymb, stmaryrd}
\usepackage[amsmath]{ntheorem}

\title{Le collectionneur de coupon}
\date{}
\author{}

\input{../outils}
\setcounter{secnumdepth}{0}
\begin{document}
\maketitle
    On considère un collectionneur qui cherche à obtenir tous les objets d'une collection
    (par exemple tous les dinausaures sachant qu'un dinausaure est donné dans une boîte de
    céréales) et que les dinausaures sont répartis équitablement dans les boîtes.
   \begin{Theoreme}
      En moyenne, ce collectionneur devra acheter $n \ln (n)$ boîtes de céréales pour
      avoir toute la collection.
   \end{Theoreme}
   \subsection{Démonstration}
     Notons $T_n$ la variable aléatoir qui représente le nombre de boîte de
     céréales à acheter pour avoir les $n$ dinausaures. Pour $0 < i \leq n$, on 
     note $t_i$ la variable aléatoire représentant le nombre de paquets 
     supplémentaites à acheter pour obtenir $i$ dinausaures sachant qu'on en a
     $i - 1$. On a
     \[
        T_n = \sum_{i = 1}^n t_i
     \]
    Quand $i - 1$ vignettes ont été obtenues, la probabilité d'obtenir une 
    nouvelle vignette en ouvrant un paquet est
    \[
      p_i = \frac{n - i + 1}{n}
    \]
    d'où $t_i$ suit une loi géométrique de paramètre $p_i$. Par suite,
    \[
       E(t_i) = \frac{n}{n - i + 1}.
    \]
    Par linéarité de l'espérance, on a alors
    \[
      E(T) = \sum_{i = 1}^n E(t_i) 
           = \sum_{i = 1}^n \frac{n}{n - i + 1}
           = \sum_{k = 1}^n \frac{n}{k}
           = n \sum_{k = 1}^n \frac{1}{k}.
    \]
    Et donc
    \[
      E(T) \underset{+\infty}{\sim} n \ln(n). 
    \]
\end{document}