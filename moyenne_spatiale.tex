\documentclass[fontsize=12pt,twoside=false,parskip=half, french]{scrartcl}
\usepackage[utf8]{inputenc}
\usepackage{babel}
\usepackage[T1]{fontenc}
\usepackage{amsmath, amssymb, stmaryrd}
\usepackage[amsmath]{ntheorem}

\title{Moyenne spatiale}
\date{}
\author{}

\input{../outils}
\setcounter{secnumdepth}{0}
\begin{document}
\maketitle
   \begin{Theoreme}[Moyenne spatiale]
      Soit $f$ continue telle que pour tout $x \in \R$ et tout $a > 0$,
      \[
         f(x) = \frac{1}{2a}\int_{x - a}^{x + a} f.
      \]
      Alors $f$ est affine.
   \end{Theoreme}
   \subsection{Démonstration}
      Les fonctions affines vérifient bien cette hypothèse. Montrons que ce sont les seules.
      
      Soit $f$ vérifiant la relation et $F$ une primitive de $f$. L’égalité devient
      \begin{equation}
         f(x) = \frac{1}{2a} (F(x + a) - F(x - a)). \label{eq:prim}
      \end{equation}
      On a alors pour tout $x \in \R$ et $a > 0$,
      \begin{align*}
         4af(x) &= F(x + 2a) - F(x - 2a)\\
                &= F(x + 2a) - F(x) + F(x) - F(x - 2a)\\
                &= 2af(x + a) + 2af(x - a).
      \end{align*}
      Et donc,
      \begin{equation}
         2f(x) = f(x + a) - f(x - a). \label{eq:dev}
      \end{equation}
      De plus, \eqref{eq:prim} implique $f$ dérivable sur $\R$. En dérivant \eqref{eq:dev}, on obtient alors
     \[
         0 = f’(x + a) - f’(x - a).
     \]
     Cette relation étant valable pour tout $x \in \R$ et $a > 0$, on en déduit que $f’$ est constante,
     c’est-à-dire que $f$ est affine.
\end{document}