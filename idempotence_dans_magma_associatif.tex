\documentclass[fontsize=12pt,twoside=false,parskip=half, french]{scrartcl}
\usepackage[utf8]{inputenc}
\usepackage{babel}
\usepackage[T1]{fontenc}
\usepackage{amsmath, amssymb, stmaryrd}
\usepackage[amsmath]{ntheorem}

\title{Idempotence}
\date{}
\author{}

\input{../outils}
\setcounter{secnumdepth}{0}
\begin{document}
\maketitle
   Soit $(E, \times)$ un ensemble muni d’une loi interne. $x$ dans $E$ est dit idempotent 
   si $x^2 = x$.
   \begin{Theoreme}
      Si $\times$ est associative et $E$ fini, alors $E$ admet un élément idempotent.
   \end{Theoreme}
   \subsection{Démonstration}
      Notons $F$ le plus petit sous-ensemble (au sens du cardinal) non vide de $E$. $F$ existe bien 
      car $E$ est fini. Considérons $a \in F$.
      
      L’ensemble $aF$ est stable parfois et $aF \subset F$, donc $aF = F$ (car $F$ est le plus petit
      sous-ensemble stable). Considérons maintenant
      \[
         G = \enstq{x \in F}{ax = a}.
      \]
      On a $G \subset F$. Montrons qu’il est stable. Soit $x$ et $y$ dans $G$. On a $a(xy) = (ax)y = ay = a$,
      donc $xy \in G$. $G$ est stable et donc $G = F$. En particulier, puisque $a \in F$, alors $a \in G$,
      et donc $a^2 = a$.
\end{document}