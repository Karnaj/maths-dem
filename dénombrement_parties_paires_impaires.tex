\documentclass[fontsize=12pt,twoside=false,parskip=half, french]{scrartcl}
\usepackage[utf8]{inputenc}
\usepackage{babel}
\usepackage[T1]{fontenc}
\usepackage{amsmath, amssymb, stmaryrd}
\usepackage[amsmath]{ntheorem}

\title{Dénombrement des parties paires d’un ensemble}
\date{}
\author{}

\input{../outils}
\setcounter{secnumdepth}{0}
\begin{document}
\maketitle
   \begin{Theoreme}
      Soit $E$ un ensemble non vide. $E$ a autant de parties paires que de parties impaires.
   \end{Theoreme}
   \subsection{Démonstration}
      % On note respectivement $N_{i, n}$ et $N_{p, n}$ le nombre de parties de $E$ de cardinal impair et pair d’un ensemble à $n$ éléments. On a
      % \[
      % \left\{ \begin{aligned}
                 % N_{i, n + 1} &= N_{i, n} + N_{p, n} = 2^n\\
                 % N_{p, n + 1} &= N_{p, n} + N_{i, n} = 2^n
              % \end{aligned}
      % \right. .
      % \]
      % En effet, pour avoir une partie de cardinal impair dans $E$ de cardinal $n + 1$, on écrit 
      % $E = A \cup \left\{a\right\}$ (avec $A = E \setminus \left\{a\right\}$ de cardinal $n$). Un ensemble de cardinal 
      % impair de $E$ est alors soit une partie de $A$ de cardinal impair, soit l’union de $\left\{a\right\}$ et d’une 
      % partie de $A$ de cardinal pair. Donc $N_{i, n + 1} = N_{i, n} + N_{p, n}$. De même, on montre que 
      % $N_{p, n + 1} = N_{i, n} + N_{p, n}$.
      % \[
         % \forall n \in \N^*, N_{i, n} = N_{p, n}.
      % \]
      
      % On peut aussi construire une bijection en se basant sur cela.  
      Soit $n$ le cardinal de $E$. 
      
      Si $n$ est impair, le nombre de parties de cardinal pair est égal au nombre de parties de cardinal impair, 
      par passage au complémentaire. 
      
      Si $n$ est pair, en considérant un élément $x$ de $E$, on a  que le nombre de parties de cardinal pair 
      (resp. impair) qui contient $n$ est égal au nombre de parties de cardinal impair (resp. pair) ne contenant pas $n$
      (on enlève l’élément $x$ de l’ensemble).
      
      PS : on en déduit bien sûr que le nombre de parties paires (et impaires) est $2^{n - 1}$. 
\end{document}