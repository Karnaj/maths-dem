\documentclass[fontsize=12pt,twoside=false,parskip=half,french]{scrartcl}
\usepackage[utf8]{inputenc}
\usepackage{babel}
\usepackage[T1]{fontenc}
\usepackage{amsmath, amssymb, stmaryrd}
\usepackage[amsmath]{ntheorem}

\title{Déterminant de Hürwitz}
\date{}
\author{}

\input{../outils}
\setcounter{secnumdepth}{0}
\begin{document}
\maketitle
   \begin{Theoreme}[Déterminant de Hürwitz]
      Soit $a, b \in \R$, et $x_1, \ldots, x_n \in \R$. On pose
      \[
         D_n(a, b) = \begin{vmatrix}
          x_1    & a      & \cdots & a\\
          b      & x_2    & \ddots & \vdots\\
          \vdots & \ddots & \ddots & a\\
          a      & \cdots & a      & x_n
         \end{vmatrix}
      \]
      Alors
      \[
      D_n(a, b) = 
      \begin{cases}
         \frac{bQ(a) - aQ(b)}{b - a} & \text{si $a \neq b$}\\
          aQ(a) - Q'(a) & \text{sinon}
      \end{cases}
      \text{ avec } Q(X) = \prod_{k = 1}^n (x_k - X).
      \]
   \end{Theoreme}
   \subsection{Démonstration}
      On va poser
      \[
         P(X) = 
         \begin{vmatrix}
            x_1 + X & a + X   & \ldots & a + X\\
            b + X   & x_2 + X & \ldots & \vdots\\
            \vdots  & \ddots  & \ddots & a + X\\
            b + X   & \ldots  & b + x  & x_n + X
         \end{vmatrix}
         =
         \begin{vmatrix}
            x_1 + X & a - x_1   & \cdots & \cdots & a - x_1\\
            b + X   & x_2 - b   & a - b  & \cdots & a - b\\
            \vdots  & 0         & \ddots & \ddots & \hdots\\
            b + X   & 0         & \cdots & 0      & x_n - b
         \end{vmatrix}.
      \]
      On a que $D_n(a, b) = P(0)$. Un développement par rapport à la première 
      ligne donne que $P$ est affine, $P(X) = \alpha X + \beta$.
      On a $P(-a) = Q(a)$ et $P(-b) = Q(b)$ (détermintant de matrices 
      triangulaires). Mais $P(-a) = -a\alpha + \beta$ et $P(-b) = -b\alpha + \beta$, d'où
      si $b \neq a$,
      \[
         D_n(a, b) = P(0) = \beta = \frac{bQ(a) - aQ(b)}{b - a}.
      \]
      Pour $D_n(a, a)$ on remarque que
      \[
        \frac{bQ(a) - aQ(b)}{b - a}
        = \frac{bQ(a) - aQ(a) + aQ(a) - aQ(b)}{b - a}
        = Q(a) - a\frac{Q(b) - Q(a)}{b - a}.
      \]
      D'où par continuité de la fonction polynomiale $b \mapsto D_n(a, b)$, 
      \[
        D_n(a, a) = \lim_{b \to a} = Q(a) - a\frac{Q(b) - Q(a)}{b - a} = Q(a) - aQ'(a).
      \]
\end{document}