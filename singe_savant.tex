\documentclass[fontsize=12pt,twoside=false,parskip=half, french]{scrartcl}
\usepackage[utf8]{inputenc}
\usepackage{babel}
\usepackage[T1]{fontenc}
\usepackage{amsmath, amssymb, stmaryrd}
\usepackage[amsmath]{ntheorem}

\title{Paradoxe du singe savant}
\date{}
\author{}

\input{../outils}
\setcounter{secnumdepth}{0}
\begin{document}
\maketitle
   Considérons qu’on met un singe devant une machine à écrire de 80 lettres (minuscules,
   majuscules et autres caractères).
   \begin{Theoreme}
      En tapant au hasard, le singe tapera presque sûrement le mot « savant ».
   \end{Theoreme}
   \subsection{Démonstration}
      Nous allons nous intéresser à la probabilité qu’il ne tape jamais le mot « savant ».
      Notons tout d’abord que chaque lettre est tapée indépendamment des autres. Ainsi,
      s’il doit taper 6 lettres, la probabilité que le mot tapé ne soit pas « savant »
      est
      \[
         p = 1 - \frac{1}{60^6}.
      \]
      Nous allons maintenant séparer son entrée en bloc de $6$. La probabilité 
      qu’il ne tape pas « savant » dans le premier bloc est de $p$, et chaque bloc
      étant tapé indépendammment, la probabilité que « savant » ne soit pas tapé 
      dans les $n$ premiers blocs est
      \[
         p_n = p^n.
      \]
      On a que $p_n$ tend vers $0$ en $+\infty$, donc la probabilité que le singe
      ne tape jamais le mot « savant » est nulle. Par suite, il tapera presque sûrement
      ce mot.
      
      En fait, le singe tapera presque sûrement n’importe quel texte donné.
\end{document}
