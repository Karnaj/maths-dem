\documentclass[fontsize=12pt,twoside=false,parskip=half]{scrartcl}
\usepackage[utf8]{inputenc}
\usepackage[french]{babel}
\usepackage[T1]{fontenc}
\usepackage{amsmath, amssymb, stmaryrd}
\usepackage[amsmath]{ntheorem}

\title{Théorème d’Euler}
\date{}
\author{}

\input{../outils_maths}
\setcounter{secnumdepth}{0}
\begin{document}
\maketitle
   \begin{Theoreme}[Théorème d’Euler]
      Notons $\phi$ l’indicatrice d’Euler. Si $a$ est premier avec $n$, alors
      \[
         a^{\phi(n)} \equiv 1 \mod n
      \]
   \end{Theoreme}
   \subsection{Démonstration}
      Si $a \wedge n = 1$, alors $\classe{a}$ est un élément du groupe des inversibles de  $\varZquotient{n}$ et ce groupe possède $\phi(n)$ éléments, d’où le résultat.

      PS : Si $p$ est premier, $\phi(p) = p - 1$ et on retrouve le \emph{petit théorème de Fermat},
      \[
         a \wedge n \implies a^{p - 1} \equiv 1 \mod n.
      \]
\end{document}
