\documentclass[fontsize=12pt,twoside=false,parskip=half, french]{scrartcl}
\usepackage[utf8]{inputenc}
\usepackage{babel}
\usepackage[T1]{fontenc}
\usepackage{amsmath, amssymb, stmaryrd}
\usepackage[amsmath]{ntheorem}

\title{Moyenne spatiale}
\date{}
\author{}

\input{../outils}
\setcounter{secnumdepth}{0}
\begin{document}
\maketitle
   Une application $f$ est dite contractante (respectivement strictement contractante) si elle est $k$-lipschitzienne avec
   $k \in \interv[0; 1]$ (respectivement $k \in \interv[0; 1[$).
   \begin{Theoreme}[Moyenne spatiale]      
      Soit $f$ une application contractante de $\R$ dans $\R$. Alors $f$ admet un unique point fixe.
   \end{Theoreme}
   \subsection{Démonstration}
      Montrons d’abord l’unicité du point fixe. Si $x$ et $y$ sont points fixes de $f$ alors
      \[
         \norme{x - y} = \norme{f(x) - f(y)} \leq k\norme{y - x}          
      \]
      or $k \in \interv[0; 1[$, donc $x = y$, l’unicité est démontrée.
      
      Montrons maintenant son existence. Pour cela étudions la suite récurrente $\suite{x}$ définie par
      $x_0 \in E$ et pour $n \in \N$, $x_{n + 1} = f(x_n)$. 
      
      Pour tout entier $n$, on a $\norme{x_{n + 1} - x_n} \geq k\norme{x_n - x_{n - 1}}$. Une récurrence donne alors
      \[
         \norme{x_{n + 1} - x_n} \geq k^n\norme{x_1 - x_0}
      \]
      Puisque $k \in \interv[0; 1[$, la série géométrique des $k^n$ converge ; par \emph{comparaison de séries à termes 
      positifs}, la série de terme général $\norme{x_{n + 1} - x_n}$ converge. Par suite, la série de terme général 
      $(x_{n + 1} - x_n)$ converge, d’où $x_n$ converge vers sa limite $l$.
      
      $f$ étant continue (car lipschitzienne), la relation $x_{n + 1} = f(x_n)$ donne (en passant à la limite) $f(l) = l$. Donc $f$ possède
      bien un point fixe.
\end{document}