\documentclass[fontsize=12pt,twoside=false,parskip=half, french]{scrartcl}
\usepackage[utf8]{inputenc}
\usepackage{babel}
\usepackage[T1]{fontenc}
\usepackage{amsmath, amssymb, stmaryrd}
\usepackage[amsmath]{ntheorem}

\title{Théorème de Radon}
\date{}
\author{}

\input{../outils}
\setcounter{secnumdepth}{0}
\begin{document}
\maketitle
   \begin{Theoreme}[Théorème de Radon]
      Soit $n \in \N$, et soit $(a_0, a_1, \ldots, a_{n + 1})$ un ensemble de
      $n + 2$ points de $\R^n$. Alors, on peut les partitionner en deux parties
      $A$ et $B$ dont les enveloppes convexes se rencontrent.
   \end{Theoreme}
   \subsection{Démonstration}
      On a $n + 2$ points dans $R^n$, donc il existe $n + 2$ scalaires $\lambda_k$
      non tous nuls tels que 
      \[
         \sum_{k = 0}^{n + 1} \lambda_k = 0 \text{ et } 
         \sum_{k = 0}^{n + 1} \lambda_ka_k = 0.
      \]
      On pose alors
      \[
      \left\{
      \begin{aligned}
         A &= \enstq{a_k}{\lambda_k > 0}\\
         B &= \enstq{a_k}{\lambda_k \leq 0}
      \end{aligned}.\right.
      \]
      Les $\lambda_k$ sont non tous nuls et de somme nulle donc $A$ et $B$ sont 
      non vides. Considérons
      \[
         x = \frac{\sum_{\lambda_k > 0} \lambda_ka_k}{\sum_{\lambda_k > 0} \lambda_k}.
      \]
      $x$ appartient à l'enveloppe convexe de $A$. De plus, on a
      \[
         \sum_{k = 0}^{n + 1} \lambda_ka_k = 0 \implies 
         \sum_{\lambda_k > 0} \lambda_ka_k = -\sum_{\lambda_k \leq 0} \lambda_ka_k
      \]
      et
      \[
         \sum_{k = 0}^{n + 1} \lambda_k = 0 \implies 
         \sum_{\lambda_k > 0} \lambda_k = -\sum_{\lambda_k \leq 0} -\lambda_k
      \]
      d'où 
      \[
         x = \frac{\sum_{\lambda_k \leq 0} \lambda_ka_k}{\sum_{\lambda_k \leq 0} \lambda_k}
      \]
      et $x$ appartient également à l'enveloppe convexe de $B$.
      
      On a partitionné les $a_k$ en deux ensembles $A$ et $B$ tels que l'intersection
      de leur enveloppe convexe est non vide.
\end{document}