\documentclass[fontsize=12pt,twoside=false,parskip=half, french]{scrartcl}
\usepackage[utf8]{inputenc}
\usepackage{babel}
\usepackage[T1]{fontenc}
\usepackage{amsmath, amssymb, stmaryrd}
\usepackage[amsmath]{ntheorem}

\title{Irrationalité de $\e$}
\date{}
\author{}

\input{../outils}
\setcounter{secnumdepth}{0}
\begin{document}
\maketitle
   \begin{Theoreme}[Irrationalité de $\e$]
      $\e$ est un nombre irrationnel.
   \end{Theoreme}
   \subsection{Démonstration}
      Supposons $\e$ rationnel. Il existe $a, b$ deux entiers strictement positifs, $b > 1$ tel que  $\e = \frac{a}{b}$.
      On pose
      \[
         x = b!\left(e - \sum_{n = 0}^b \frac{1}{n!}\right).
      \]
      On a 
      \[
         x =  b!\left(e - \sum_{n = 0}^b \frac{1}{n!}\right) = b! \frac{a}{b} - \sum_{n = 0}^b \frac{b!}{n!}
      \]
      d’où $x$ est un entier.
      
      D’un autre côté,
      \[
         x = b!\left(e - \sum_{n = 0}^b \frac{1}{n!}\right) = b! \sum_{n = 0}^{+\infty} \frac{1}{n!} - \sum_{n = 0}^b \frac{b!}{n!}
                                                           = \sum_{n = b + 1}^{+\infty} \frac{b!}{n!}
      \]
      On a alors
      \[
         x = \frac{1}{b + 1} + \frac{1}{(b + 1)(b + 2)} + \cdots 
      \]
      c’est-à-dire, en faisant intervenir la série géométrique de raison $0 < \frac{1}{b + 1} < 1$ 
      \[
         0 < x < \sum_{n = 1}^{+\infty} \frac{1}{(b + 1)^n} = \frac{1}{b + 1} \times \frac{1}{1 - \frac{1}{b + 1}} = \frac{1}{b} < 1.
      \]
      $x$ est un entier strictement positif et plus petit que $1$. ABSURDE, donc $\e$ est irrationel.
\end{document}