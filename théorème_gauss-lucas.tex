\documentclass[fontsize=12pt,twoside=false,parskip=half, french]{scrartcl}
\usepackage[utf8]{inputenc}
\usepackage{babel}
\usepackage[T1]{fontenc}
\usepackage{amsmath, amssymb, stmaryrd}
\usepackage[amsmath]{ntheorem}

\title{Théorème de Gauss-Jordan}
\date{}
\author{}

\input{../outils}
\setcounter{secnumdepth}{0}
\begin{document}
\maketitle
   \begin{Theoreme}[Théorème de Gauss-Jordan]
   On note $\Conv(P)$ l'enveloppe convexe des racines de $P \in \C[X]$.
   Soit $P \in \C[X]$ un polynôme non constant. Alors $\Conv(P') \subset \Conv(P)$.
   \end{Theoreme}
   \subsection{Démonstration}
      Soit $P \in \C[X]$ non constant. On peut écrire
      \[
         P = \lambda \prod_{j = 1}^r (X - \lambda_j)^{m_j}
      \]
      avec $\lambda \in \C^*$, $r \geq 1$ et $\lambda_k$ les racines distinctes
      de $P$ d'ordre de multiplicité $m_k$. On a
      \[
         P' = \lambda \sum_{k = 1}^r m_k (X - \lambda_k)^{m_k - 1} 
                                     \prod_{j \neq k} (X - \lambda_j)^{m_j}.
      \]
      Par suite, la décomposition en éléments simple de $P'/P$ est
      \[
        \frac{P'}{P} = \sum_{k = 1}^r \frac{m_k}{X - \lambda_k}.
      \]
      Soit $z$ une racine de $P'$. Si $z$ est une des racines de $P$, alors
      $z \in \Conv(P)$. Sinon on a
      \[
         0 = \frac{P'(z)}{P(z)} 
           = \sum_{k = 1}^r \frac{m_k}{z - \lambda_k}
           = \sum_{k = 1}^r m_k\frac{\conjugue{z - \lambda_k}}{\module{z - \lambda_k}^2}.
      \]
      En passant au conjugué on obtient
      \[
        0 = \sum_{k = 1}^r m_k\frac{z - \lambda_k}{\module{z - \lambda_k}^2}.
      \]
      Et donc
      \[  
        z = \frac{\sum_{k = 1}^r \frac{m_k}{\module{z - \lambda_k}^2}\lambda_k}
                 {\sum_{k = 1}^r \frac{m_k}{\module{z - \lambda_k}^2}}
      \]
      et est donc une combinaison convexe des $\lambda_k$ d'où le résultat.
\end{document}