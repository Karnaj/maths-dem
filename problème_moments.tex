\documentclass[fontsize=12pt,twoside=false,parskip=half, french]{scrartcl}
\usepackage[utf8]{inputenc}
\usepackage{babel}
\usepackage[T1]{fontenc}
\usepackage{amsmath, amssymb, stmaryrd}
\usepackage[amsmath]{ntheorem}

\title{Un théorème des moments}
\date{}
\author{}

\input{../outils}
\setcounter{secnumdepth}{0}
\begin{document}
\maketitle
   \begin{Theoreme}
      Soit $f$ continue de $\interv[0; 1]$ dans $^R$ tel que pour tout $k \in \N$
      \[
         \int_0^1 f(x)x^k \dx = 0.
      \]
      Alors $f$ est identiquement nulle.
   \end{Theoreme}
   \subsection{Démonstration}
      D’après le \emph{théorème de Weierstrass}, il existe une suite de polynômes $\suite{P}$ qui converge
      uniformément vers $f$ sur $\interv[0; 1]$. Par linéarité de l’intégrale, on a que
      \begin{equation}
         \int_0^1 fP_n = 0 \label{eq:int}
      \end{equation}
      pour tout $n$. 
      
      $f$ est continue sur un segment donc bornée par une constante $M \in \R_+$ (\emph{Théorème des bornes}).
      Par suite, pour tout $x \in  \interv[0; 1]$,
      \[
         \normeInf{fP_n - f^2} = M\normeInf{P_n - f}
      \]
      qui tend vers 0 puisque $\suite{P}$ converge uniformément vers $f$ sur $\interv[0; 1]$. Donc, $\suite{fP}$ converge 
      uniformément vers $f^2$ sur $\interv[0; 1]$.
      
      En passant à la limite sous l’intégrale \eqref{eq:int}, on a que
      \[
         \int_0^1 f^2 = 0.
      \]
      $f^2$ étant continue, positive, cela impose $f^2$ nulle, donc $f$ nulle.
\end{document}