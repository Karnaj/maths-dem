\documentclass[fontsize=12pt,twoside=false,parskip=half, french]{scrartcl}
\usepackage[utf8]{inputenc}
\usepackage{babel}
\usepackage[T1]{fontenc}
\usepackage{amsmath, amssymb, stmaryrd}
\usepackage[amsmath]{ntheorem}

\title{Carrés de $\varZquotient{n}$}
\date{}
\author{}

\input{../outils}
\setcounter{secnumdepth}{0}
\begin{document}
\maketitle
   \begin{Theoreme}[Carrés de $\varZquotient{n}$]
      Soit $p$ un nombre premier supérieur à 2. Le nombre de carrés de $\varZquotient{n}$ est
      \[
         \frac{n + 1}{2}.
      \]
      et $x$ est un carré si et seulement si $x^{\frac{p - 1}{2}} = 1$.
   \end{Theoreme}
   \subsection{Démonstration}
      Posons $\phi$ l’application qui à $x$ associe $x^2$. Dans le corps $\varZquotient{n}$,
      \[
         \phi(x) = \phi(y) \iff x = \pm y.
      \]
      Dans l’image de $\phi$, seul $0$ posède un seul antécédant (lui-même), les autres en possèdent deux distincs. Par suite, 
      % \[
         % \card \varZquotient{n}$ = 1 + 2(\card \Im \phi - 1)
      % \] donc
      le nombre de carrés de $\varZquotient{n}$ est
      \[
         \frac{n + 1}{2}.
      \]
      Si $x$ est un carré, il existe $y$ tel que $y^2 = x$, et donc
      $x^{\frac{p - 1}{2}} = y^{p - 1} = 1$. Réciproquement, si $x$ n'est pas un
      carré, il n'existe pas de $y$ tel que $x^{\frac{p - 1}{2}} = y^{p - 1}$, 
      et donc $x^{\frac{p - 1}{2}} \neq 1$.
      
      Donc $x$ est un carré si et seulement si $x^{\frac{p - 1}{2}} = 1$.
      
      PS : On pourrait montrer que $-1$ est un carré si et seulement si $p \equiv 1 \mod 4$.
\end{document}
