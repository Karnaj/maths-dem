\documentclass[fontsize=12pt,twoside=false,parskip=half,french]{scrartcl}
\usepackage[utf8]{inputenc}
\usepackage{babel}
\usepackage[T1]{fontenc}
\usepackage{amsmath, amssymb, stmaryrd}
\usepackage[amsmath]{ntheorem}

\title{Fonction réelle continue additive}
\date{}
\author{}

\input{../outils}
\setcounter{secnumdepth}{0}
\begin{document}
\maketitle
   \begin{Theoreme}
      Soit $f$ continue de $\R$ dans $\R$ telle que pour tout réels $x$ et $y$
      \[
        f(x + y) = f(x) + f(y). 
      \]
      Alors $f$ est linéaire. 
   \end{Theoreme}
   \subsection{Démonstration}
        On commence par montrer que pour tout réel $x$ et pour tout $n \in \N$,
        $f(nx) = nf(x)$. Cest vrai pour $n = 0$ ($f(0) = f(0) + f(0)$
        implique que $f(0)$ est nul) et s'il est vrai pour $n$, alors 
        $f((n + 1)x) = f(nx) + f(x)$ permet de montrer le résultat par récurrence
        sur $n$.
        
        Ce résultat s'étend ensuite à $ n \in \Z$. En effet, on a $f(-x) = f(x)$
        puisque $f(x) + f(-x) =f(x - x) = f(0) = 0$.
        
        En considérant $r = \frac{p}{q}$ un rationnel, avec $p \in Z$ et $q \in \N^*$,
        on obtient
        \[
            f(r) = f\left(p \times \frac{1}{q}\right) = p f\left(\frac{1}{q}\right)
                 = \frac{p}{q}\times q \times f\left(\frac{1}{q}\right) 
                 = \frac{p}{q}\times q \times f(1)
                 = rf(1). 
        \]
        En notant $a = f(1)$, on a pour tout $r \in \Q$, $f(r) = ar$.

        Et finalement on étend ce résultat à $\R$ par continuité et densité de $\Q$
        dans $\R$. En effet, pour tout réel $x$, il existe une suite $u_n$ de 
        rationnel qui converge vers $x$. Par continuité $f(u_n) \to f(x)$ avec
        $f(u_n) = au_n \to ax$. Par unicité de la limité, $f(x) = ax$.
        
\end{document}