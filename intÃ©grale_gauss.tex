\documentclass[fontsize=12pt,twoside=false,parskip=half, french]{scrartcl}
\usepackage[utf8]{inputenc}
\usepackage{babel}
\usepackage[T1]{fontenc}
\usepackage{amsmath, amssymb, stmaryrd}
\usepackage[amsmath]{ntheorem}

\title{Intégrale de Gauss}
\date{}
\author{}

\input{../outils}
\setcounter{secnumdepth}{0}
\begin{document}
\maketitle
   \begin{Theoreme}[Intégrale de Gauss]
      \[
         \int_0^{+\infty} \e^{-t^2} \dt = \frac{\sqrt{\pi}}{2}.
      \]
   \end{Theoreme}
   \subsection{Démonstration}
      La fonction $t \mapsto \e^{-t^2}$ est bien intégrable sur $\interv[0; +\infty[$ car $o(1/t^2)$.
      Posons $I$ son intégrale. On a 
      \begin{align*}
         I^2 &= \left( \int_0^{+\infty} \e^{-x^2} \dx \right) \left( \int_0^{+\infty} \e^{-y^2} \dy \right) \\
             &= \iint_{\R \times \R} \e^{-(x^2 + y^2)} \dx \dy.
      \end{align*}
      On fait alors un changement de variable polaires
      \[
         \begin{systeme}
            x &= r\cos \theta
            y &= r \sin \theta
         \end{systeme}
      \]
      Ici, on est sur un quart de plan, $r \in \interv[0; +\infty[$ et $\theta \in \interv[0; \frac{\pi}{2}]$. Alors,
      \begin{align*}
         I^2 &= \iint_{\R \times \interv[0; \frac{\pi}{2}]} r \e^{-r^2} \dr \dtheta\\
             &= \left( \int_0^{+\infty} \e^{-x^2} \dx \right) \left( \int_0^{\interv[0; \frac{\pi}{2}]} \dtheta \right)\\
             &= \frac{1}{2} \times \frac{\pi}{2}\\
             &= \frac{\pi}{4}.
      \end{align*}
      Par suite, on a bien
      \[
         I = \frac{\sqrt{\pi}}{2}.
      \]
\end{document}