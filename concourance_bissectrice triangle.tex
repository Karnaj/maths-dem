\documentclass[fontsize=12pt,twoside=false,parskip=half,french]{scrartcl}
\usepackage[utf8]{inputenc}
\usepackage{babel}
\usepackage[T1]{fontenc}
\usepackage{amsmath, amssymb, stmaryrd}
\usepackage[amsmath]{ntheorem}

\title{Concourance des bissectrices d'un triangle}
\date{}
\author{}

\input{outils}
\setcounter{secnumdepth}{0}
\begin{document}
\maketitle
   \begin{Theoreme}[Concourance des bissectrices d'un triangle]
      Les bissectrices d'un triangle sont concourantes.
   \end{Theoreme}
   \subsection{Démonstration}
      Soit $ABC$ un triangle. On note $d_A$, $d_B$ et $d_C$ les bissectrices
      de $ABC$ issues respectivement des sommets $A$, $B$ et $C$. Soit $I$
      le point d'intersection de $d_A$ et $d_B$. $I \in d_A$, donc
      la distance de $I$ à $[AB]$ est égale à la distance de $I$ à $[AC]$.
      De même, puisque $I \in d_B$, la distance de $I$ à $[AB]$ est égale à 
      la distance de $I$ à $[BC]$. Par suite, la distance de $I$ à $[AC]$ est 
      égale à la distance de $I$ à $[BC]$ c'est-à-dire $I \in d_C$.
\end{document}