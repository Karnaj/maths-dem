\documentclass[fontsize=12pt,twoside=false,parskip=half]{scrartcl}
\usepackage[utf8]{inputenc}
\usepackage[french]{babel}
\usepackage[T1]{fontenc}
\usepackage{amsmath, amssymb, stmaryrd}
\usepackage[amsmath]{ntheorem}

\title{Lemme fondamental du calcul des variations}
\date{}
\author{}

\input{../outils_maths}
\setcounter{secnumdepth}{0}
\begin{document}
\maketitle
   \begin{Theoreme}[]
      Soit $f$ une fonction continue de $\interv]a; b[$ dans $\R$. Si pour toute fonction
      $h$ de $\interv]a; b[$ dans $\R$, on a
      \[
         \int_a^b fh = 0
      \]
      Alors, $f$ est nulle.
   \end{Theoreme}
   \subsection{Démonstration}
      Si $f$ vérifie les hypothèses, supposons qu’il existe $x$ dans $\interv]a; b[$, 
      $f(x) \neq 0$. Par continuité, $f$ est de signe constant sur un intervalle $I \subset \interv]a; b[$ tel que $x \in I$. Considérons $v$ de classe $C^1$ sur 
      $\interv[a; b]$ strictement positive sur $I$ et nulle en dehors. On peut prendre 
      par exemple
      \[
         v(x) = \begin{cases}
            (d - x)^2(x - a)^2 & \text{si } x \in I\\
            0                  & \text{sinon}.
         \end{cases}
      \]
      On a que $fv$ est de signe constant (et non nulle) sur $I$ et nulle ailleurs, donc 
      son intégrale est non nulle, ABSURDE. Donc $f$ est nulle sur $\interv]a; b[$ et donc
      sur $\interv[a; b]$ par continuité.
\end{document}
