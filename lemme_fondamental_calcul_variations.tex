\documentclass[fontsize=12pt,twoside=false,parskip=half, french]{scrartcl}
\usepackage[utf8]{inputenc}
\usepackage{babel}
\usepackage[T1]{fontenc}
\usepackage{amsmath, amssymb, stmaryrd}
\usepackage[amsmath]{ntheorem}

\title{Lemme fondamental du calcul des variations}
\date{}
\author{}

\input{../outils}
\setcounter{secnumdepth}{0}
\begin{document}
\maketitle
   \begin{Theoreme}
      Soit $f$ une fonction continue de $\interv]a; b[$ dans $\R$. Si pour toute fonction
      $h$ de $\interv]a; b[$ dans $\R$, on a
      \[
         \int_a^b fh = 0
      \]
      Alors, $f$ est nulle.
   \end{Theoreme}
   \subsection{Démonstration (absurde)}
      Si $f$ vérifie les hypothèses, supposons qu’il existe $x$ dans $\interv]a; b[$, 
      $f(x) \neq 0$. Par continuité, $f$ est de signe constant sur un intervalle $I \subset \interv]a; b[$ tel que $x \in I$. Considérons $v$ de classe $C^1$ sur 
      $\interv[a; b]$ strictement positive sur $I$ et nulle en dehors. On peut prendre 
      par exemple
      \[
         v(x) = \begin{cases}
            (b - x)^2(x - a)^2 & \text{si } x \in I\\
            0                  & \text{sinon}.
         \end{cases}
      \]
      On a que $fv$ est de signe constant (et non nulle) sur $I$ et nulle ailleurs, donc 
      son intégrale est non nulle, ABSURDE. Donc $f$ est nulle sur $\interv]a; b[$ et donc
      sur $\interv[a; b]$ par continuité.
      
   \subsection{Démonstration (directe) si $f$ est continue sur $\interv[a; b]$}
      On considère le produit scalaire intégrale sur $\interv]a; b[$. $f$ est 
      orthogonale à toute fonction de $\interv]a; b[$ dans $\R$, donc en particulier, 
      $f$ est orthogonale à tout polynôme. Mais il existe une suite de polynômes
      qui converge uniformément vers $f$ (théorème de \emph{Stone-Weierstrass}).
      
      On a alors que pour cette suite de polynômes l'intégrale tend vers $0$. 
      En passant à la limite, on obtient que l'intégrale de $f^2$ est nulle 
      avec $f^2$ positive donc $f$ est nulle.
\end{document}