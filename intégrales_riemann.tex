\documentclass[fontsize=12pt,twoside=false,parskip=half]{scrartcl}
\usepackage[utf8]{inputenc}
\usepackage[french]{babel}
\usepackage[T1]{fontenc}
\usepackage{amsmath, amssymb, stmaryrd}
\usepackage[amsmath]{ntheorem}

\title{Intégrale de Riemann}
\date{}
\author{}

\input{../outils_maths}
\setcounter{secnumdepth}{0}
\begin{document}
\maketitle
    Soit $\alpha \in \R$. L’intégrale de Riemann est l’intégrale
   \[
      \int_1^{+\infty} \frac{\dt}{t^\alpha}.
   \]
   \begin{Theoreme}[Intégrale de Riemann]
      L’intégrale de Riemann converge si et seulement si $\alpha > 1$.
   \end{Theoreme}
   \subsection{Démonstration}
      Posons
      \[
         f \colon t \to \frac{1}{t^\alpha}.
      \]
      $f$ est définie, continue et positive sur $\interv[1 ; +\infty[$.
      
      Si $\alpha \leq 1$,
      \[
         \int_1^{x} \frac{\dt}{t^\alpha} \geq \int_1^{x} \frac{\dt}{t} = \ln(x) \to +\infty.
      \]
      donc l’intégrale de Bertrand diverge.
      
      Si $\alpha > 1$,
      \[
         \int_1^{x} \frac{\dt}{t^\alpha} = \left[-\frac{1}{(\alpha - 1) t^{\alpha - 1}}\right]_1^x \to \frac{1}{\alpha - 1}.
      \]
      donc l’intégrale de Riemann converge.
   
   \begin{Theoreme}[Somme de Riemann]
      Le \emph{théorème de comparaison série-intégrale} permet de montrer que la somme de Riemann 
      \[
         \sum_{n \geq 1} \frac{1}{n^\alpha}
      \]
      converge si et seulement si $\alpha > 1$.
   \end{Theoreme}
\end{document}