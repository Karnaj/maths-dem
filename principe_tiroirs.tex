\documentclass[fontsize=12pt,twoside=false,parskip=half]{scrartcl}
\usepackage[utf8]{inputenc}
\usepackage[french]{babel}
\usepackage[T1]{fontenc}
\usepackage{amsmath, amssymb, stmaryrd}
\usepackage[amsmath]{ntheorem}

\title{Multiple de 2017}
\date{}
\author{}

\input{../outils_maths}
\setcounter{secnumdepth}{0}
\begin{document}
\maketitle
   Le principe des tiroirs dit que si on a $n$ chaussettes à placer dans $p$ tiroirs avec $n > p$, 
   alors il existe un tiroir qui contient au moins deux chaussettes. On va l’utiliser pour montrer ce résultat.
   \begin{Theoreme}
      Montrer qu’il existe un multiple de $2017$ qui ne s’écrit qu’avec des $1$.
   \end{Theoreme}
   \subsection{Démonstration}
      On va noter $u_n$ le nombre dont l’écriture en base 10 est formée de $n$ $1$ : $u_1 = 1,u_2 = 11$, etc.
      Le principe des tiroirs donne qu’il y a au moins deux termes de cette suite congrus modulo 2017 (il y a une 
      infinité de termes dans la suite et 2017 restes possibles).
      
      Soient $u_n$ et $u_m$ deux de ces termes (avec $n > m$). 
      On a $u_n \equiv u_m[2017]$, donc $2017$ divise $u_n - u_m = 10^mu_{n - m}$, or $2017$ est premier avec 
      $10^{n - m}$, donc $2017$ divise $u_{n - m}$.
      
      $u_{n - m}$ est un multiple de $2017$ qui ne s’écrit qu’avec des $1$.
      
      PS : Ce résultat est vrai pour tout entier premier avec 10.
\end{document}