\documentclass[fontsize=12pt,twoside=false,parskip=half, french]{scrartcl}
\usepackage[utf8]{inputenc}
\usepackage{babel}
\usepackage[T1]{fontenc}
\usepackage{amsmath, amssymb, stmaryrd}
\usepackage[amsmath]{ntheorem}

\title{Formule du pion}
\date{}
\author{}

\input{../outils}
\setcounter{secnumdepth}{0}
\begin{document}
\maketitle
   \begin{Theoreme}[Formule de Chu-Vandermonde]
      Soit $k$ et $n$ deux entiers, $k < n$. 
      \[
         k\binom{k}{n} = n\binom{k - 1}{n - 1}.
      \]
   \end{Theoreme}
   % \subsection{Démonstration (numérique)}
      % On a $(1 + x)^{a + b} = $(1 + x)^a (1 + x)^b$. En utilisant la \emph{formule du binôme de Newton}, on obtient
      % \[
         % (1 + x)^{a + b} = \sum_{r = 0}^{a + b} \binom{r}{a + b} x^k
      % \]
      % et 
      % \begin{align*}
         % (1 + x)^a (1 + x)^b &= \sum_{i = 0}^a \binom{i}{a} x^i \sum_{j = 0}^{b} \binom{j}{b} x^j\\
                             % &= 
      % \end{align*}
   \subsection{Démonstration (ensembliste)}
      Soit $E$ un ensemble à $n$ éléments.  Notre but est de choisir $k$ éléments
      dans $E$, dont un élément particulier (le pion). Voici deux moyens de les
      choisir.
      \begin{itemize}
         \item On choisit les $k$ éléments ($\binom{k}{n}$ possibilités), puis
               on choisit le pion parmi ceux-là ($k$ possibilités).
         \item On choisir le pion ($n$ possibilités), puis on choisit $k - 1$
               éléments pour compléter ($\binom{k - 1}{n - 1}$ possibilités 
               puisque le pion a déjà été choisi).
      \end{itemize}
      D'où le résultat.
      
      Bien sûr, nous pourrions également le montrer par le calcul.
\end{document}