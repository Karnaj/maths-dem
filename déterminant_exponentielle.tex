\documentclass[fontsize=12pt,twoside=false,parskip=half,french]{scrartcl}
\usepackage[utf8]{inputenc}
\usepackage{babel}
\usepackage[T1]{fontenc}
\usepackage{amsmath, amssymb, stmaryrd}
\usepackage[amsmath]{ntheorem}

\title{Déterminant d'une exponentielle}
\date{}
\author{}

\input{../outils}
\setcounter{secnumdepth}{0}
\begin{document}
\maketitle
   Soit $K = \R$ ou $\C$ et $n \in \N$. On définit $\exp(A)$ pour $A \in \M_n(K)$
   par
   \[
      \exp(A) = \sum_{k = 0}^{+\infty} \frac{A^k}{k!}.
   \]
   Elle est bien définie et est continue sur $\M_n(\C)$. En effet, en munissant
   $M_n(\C)$ d'une norme d'algèbre, on a
   \[
     \sum_{k = 0}^{N} \frac{\norme{A^k}}{k!} \leq \sum_{k = 0}^{N} \frac{\norme{A}^k}{k!}
     \leq \exp(\norme{A}).
   \]
   et donc la série converge normalement sur tout compact.
   \begin{Theoreme}[Déterminant d'une exponentielle]
      Soit $A \in \M_n(K)$, alors
      \[
        \det(\exp(A)) = \exp(\Tr(A))
      \]
   \end{Theoreme}
   \subsection{Démonstration}
      Si $D = \Diag{\lambda_1,\ldots, \lambda_n}$, $D^k = \Diag{\lambda_1^k, \ldots, \lambda_n^k}$ d'où
      $\exp(D) = \Diag{\exp\left(\lambda_1^k\right), \ldots, \exp\left(\lambda_n^k\right)}.
      $
      On a alors 
      $\displaystyle \det(\exp(D)) = \prod_{k = 1}^n \exp(\lambda_k) 
                                   = \exp(\sum_{k = 1}^n \lambda_k)
                                   = \exp(\Tr(D)).
      $
      
      Le résultat est donc vrai sur les matrices diagonales. Considérons maintenant $M$ une matrice diagonalisable. Il existe $P \in \GL_n(\K)$,
      et $D$ diagonale, $D = P^{-1}MP$.
      
      On a par récurrence que $(P^{-1}MP)^k = P^{-1}M^kP$, d'où $\exp(D) = P^{-1}\exp(M)P$.
      Ainsi, $\exp(D)$ et $\exp(M)$ sont semblables et ont donc même trace 
      et même déterminant ce qui donne le résultat pour les matrices diagonalisables.

      La densité des matrices diagonalisables dans $M_n(\C)$ et la continuité
      des applications $\exp$, $\det$ et $\Tr$ permettent d'avoir le résultat
      pour toutes les matrices de $\M_n(\C)$ (et donc pour celles de $\M_n(\R)$.
      
      
\end{document}
