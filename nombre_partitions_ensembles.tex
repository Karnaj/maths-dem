\documentclass[fontsize=12pt,twoside=false,parskip=half, french]{scrartcl}
\usepackage[utf8]{inputenc}
\usepackage{babel}
\usepackage[T1]{fontenc}
\usepackage{amsmath, amssymb, stmaryrd}
\usepackage[amsmath]{ntheorem}

\title{Nombres de partitions d’un ensemble}
\date{}
\author{}

\input{../outils}
\setcounter{secnumdepth}{0}
\begin{document}
\maketitle
   \begin{Theoreme}[Nombres de Bell]
      Montrer que le nombre $P_n$ de partitions d’un ensemble à $n$ éléments suit la relation
      \[
         \begin{systeme}
            P_0       &= 1\\
            P_{n + 1} &= \sum{k = 0}^n \binom{n}{k} P_k
         \end{systeme}
      \]
   \end{Theoreme}
   \subsection{Démonstration}
      Montrons cette relation par construction. Un ensemble a $0$ éléments a $1$ partition
      (celle contenant l’ensemble vide).

      Soit $E$ un ensemble à $n + 1$ éléments et soit $x$ un élément de $E$.
      Une partition de $E$ est alors composée d’une partie $H$
      contenant $x$ et ayant $k$ éléments hormis $x$ ($k \in \intervalleEntier{0}{n}$) qu’on complète avec une partition de ce qui reste (donc d’un ensemble à $n - k$ 
      éléments). Pour tout $k$ entre $0$ et $n$, le nombre de parties $H$ possibles
      est $\binom{n}{k}$ et le nomnbre de partitions possibles de ce qui reste est 
      $P_{n - k}$, d’où
      \begin{align*}
         P_{n + 1} &= \sum{k = 0}^n \binom{n}{k} P_{n - k}\\
                   &= \sum_{i = 0}^n \binom{n}{n - i} P_i & (i = n - k)\\
                   &=  \sum_{i = 0}^n \binom{n}{i} P_i
      \end{align*}      
\end{document}