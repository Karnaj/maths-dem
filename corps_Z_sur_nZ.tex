\documentclass[fontsize=12pt,twoside=false,parskip=half]{scrartcl}
\usepackage[utf8]{inputenc}
\usepackage[french]{babel}
\usepackage[T1]{fontenc}
\usepackage{amsmath, amssymb, stmaryrd}
\usepackage[amsmath]{ntheorem}

\title{Condition pour que $\varZquotient{p}$ soit un corps}
\date{}
\author{}

\input{../outils_maths}
\setcounter{secnumdepth}{0}
\begin{document}
\maketitle
   \begin{Theoreme}[Condition pour que $\varZquotient{p}$ soit un corps]
      $\varZquotient{p}$ est un corps si et seulement si $p$ est premier.
   \end{Theoreme}
   \subsection{Démonstration}
      Rappelons tout d’abord que $(\varZquotient{p}, +, \times)$ est un anneau 
      commutatif et que $\classe{m}$ est inversible si et seulement si 
      $p \wedge m = 1$. 

      Supposons que $(\varZquotient{p}, +, \times)$ soit un corps. Pour tout 
      $m \in \intervalleEntier{2}{p - 1}$, $\classe{m}$ est inversible, donc 
      $m \wedge p = 1$. Par suite, $p$ est premier.
      
      Inversement, si $p$ est premier, pour tout $m \in \intervalleEntier{2}{p - 1}$,
      $m \wedge p = 1$, d’où $\classe{m}$ est inversible. Par suite, tous les 
      éléments sauf $\classe{0}$ sont inversible. On est bien dans un corps.
      
\end{document}