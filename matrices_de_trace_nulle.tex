\documentclass[fontsize=12pt,twoside=false,parskip=half, french]{scrartcl}
\usepackage[utf8]{inputenc}
\usepackage{babel}
\usepackage[T1]{fontenc}
\usepackage{amsmath, amssymb, stmaryrd}
\usepackage[amsmath]{ntheorem}

\title{Matrice de trace nulle}
\date{}
\author{}

\input{../outils}
\setcounter{secnumdepth}{0}
\begin{document}
\maketitle
   \begin{Theoreme}
      Soit $M \in \M_n(\R)$ de trace nulle. Alors, $M$ est semblable à une matrice à diagonale nulle.
   \end{Theoreme}
   \subsection{Démonstration}
      On considère $f$ l’endomorphisme canoniquement associé à $M$.
      Montrons le résultat par récurrence sur $n$. Si $n = 1$, le résultat est vrai.
      Soit $n > 1$. On suppose le résultat vrai à l’ordre $n - 1$.
      
      Si $f$ est une homothétie, le résultat est vrai et $M$ est la matrice nulle.
      
      Si $f$ n’est pas une homothétie, on montre par contraposée qu’il existe $x$ tel que $(x, f(x))$ est libre. 
      % si pour tout $x$, $(x, f(x)$ est liée, alors pour tout $x$, il existe $\lambda_x$ tel que $f(x) = \lambda_x x$.
      % Alors, 
      % \[
         % \begin{systeme}
            % f(x + y) &= \lambda_{x + y} (x + y) = \lambda_{x + y} x + \lambda_{x + y} y\\
            % f(x + y) &= f(x) + f(y) = \lambda_{x} x + \lambda_{y} y
         % \end{systeme}
      % \]
      % Donc, pour tout $(x, y)$, $\lambda_x = \lambda_y = \lambda$, et $f(x) = \lambda x$, $f$ est une homothétie.
      Considérons un tel $x$ et complétons $(x, f(x))$ en une base
      de l’espace. Dans cette base, la matrice de $f$ est
      \[
         M’ = 
         \begin{pmatrix}
            0 & L\\
            C & A
         \end{pmatrix}
      \]
      avec $C$ une matrice colonne, $L$ une matrice ligne et $A \in \M_{n - 1}(\R)$. $A$ est de trace nulle,
      l’hypothèse de récurrence donne que $A$ est semblable à $B$ de diagonale nulle. Il y a $P$ inversible telle que $A = PBP^{-1}$. Posons
      \[
         Q = 
         \begin{pmatrix}
           1 & 0\\
           0 & P
         \end{pmatrix}
      \]
      On a $Q$ inversible et en utilisant $Q$ comme matrice de passage, on a $M$ semblable à
      \[
         \begin{pmatrix}
            0 & L\\
            C & A
         \end{pmatrix} = Q \begin{pmatrix}
         0  & *\\
         * & B
         \end{pmatrix} Q^{-1}        
      \]
      qui est bien une matrice de trace nulle.
\end{document}