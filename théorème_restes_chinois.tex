\documentclass[fontsize=12pt,twoside=false,parskip=half]{scrartcl}
\usepackage[utf8]{inputenc}
\usepackage[french]{babel}
\usepackage[T1]{fontenc}
\usepackage{amsmath, amssymb, stmaryrd}
\usepackage[amsmath]{ntheorem}

\title{Théorème des restes chinois}
\date{}
\author{}

\input{../outils_maths}
\setcounter{secnumdepth}{0}
\begin{document}
\maketitle
   \begin{Theoreme}[Théorème des restes chinois]
      Soit $m$ et $n$ deux entiers premiers entre eux, et deux entiers $a$ et $b$. Le système
      \[
         \begin{systeme}
            x \equiv a &\mod n\\
            x \equiv b &\mod m
         \end{systeme}
      \]
      possède une seule solution modulo $mn$.
   \end{Theoreme}
   \subsection{Démonstration}
      Le système équivaut à l’existence de $k_1$ et $k_2$ tel que
      $\begin{systeme}
            x &= nk_1 + a\\
            x &= mk_2 + b
         \end{systeme}$.
         
      Le \emph{théorème de Bézout} donne l’existence de $u$ et $v$ tel que $un + vm = 1$. En multipliant la ligne 1 du 
      système par $mv$ et la ligne 2 par $nu$, on obtient
      \[
         \begin{systeme}
            mvx &= mvnk_1 + amv\\
            nux &= numk_2 + bnu
         \end{systeme}
      \]
      ce qui permet d’avoir en additionnant les deux lignes
      \[
         x = mn(vk_1 + uk_2) + amv + bnu = mn(vk_1 + uk_2) + amv + bnu
      \]
      $x$ bien est unique modulo $mn$ ($x \equiv amv + bnu \mod mn$).
      
 
      Réciproquement, si $x \equiv amv + bnu \mod mn$, on a modulo $n$,
      \[
         x = mn(vk_1 + uk_2) + amv + bnu \equiv amv \equiv a(1 - un) \equiv a \mod n 
      \]
      et de même $x \equiv b \mod m$.
      
      PS : ceci montre l’isomorphisme entre $\varZquotient{nm}$ et $\varZquotient{n} \times \varZquotient{m}$
      et cette démonstration permet de résoudre les systèmes de ce type en deux étapes.
      \begin{enumerate}
         \item On cherche une solution particulière $x_0$ (peut être fait en trouvant les $u$ et $v$ associé au
               \emph{théorème de Bézout} pour avoir la solution $amv + bnu$).
         \item L’ensemble des solutions est $\enstq{x_0 + knm}{k \in \Z}$.
      \end{enumerate}
\end{document}