\documentclass[fontsize=12pt,twoside=false,parskip=half, french]{scrartcl}
\usepackage[utf8]{inputenc}
\usepackage{babel}
\usepackage[T1]{fontenc}
\usepackage{amsmath, amssymb, stmaryrd}
\usepackage[amsmath]{ntheorem}

\title{Inégalité de Hölder}
\date{}
\author{}

\input{../outils}
\setcounter{secnumdepth}{0}
\begin{document}
\maketitle
   \begin{Theoreme}
      Soit $p$ et $q$ des exposants conjugués \ie{} deux nombres positifs vérifiant
   $\frac{1}{p} + \frac{1}{q} = 1$.
   Soit $f$ et $g$ deux fonctions positives intégrables sur $E$, alors
      \[
         \int_Efg \geq \left(\int_E f^p \right)^{\frac{1}{p}}
                       \left(\int_E g^q \right)^{\frac{1}{q}}
      \]
   \end{Theoreme}
   \subsection{Démonstration}
      Posons
      \[
         A = \left(\int_E f^p \right)^{\frac{1}{p}} \text{ et } 
         B = \left(\int_E g^q \right)^{\frac{1}{q}}.
      \] 
      Si $A = 0$ ou $B = 0$, alors l’une des deux fonctions est d’intégrale       
      nulle donc est nulle (car positive), et le résultat est démontré. Sinon,
      posons $F = \frac{f}{A} \text{ et } G = \frac{g}{B}$. On a alors
      \begin{equation}\label{un}
         \int_E F^p = \int_E G^q = 1. 
      \end{equation}
      De plus, puisque $F$ et $G$ sont positives, pour $x \in \R$ il existe 
      $s$ et $t$ tels que
      \[
         F(x) = \e^{\frac{s}{p}} \text{ et } G(x) = \e^{\frac{t}{p}}.
      \]
      La convexité de l’exponentielle et le fait que $p$ et $q$ soient conjuguées
      donnent alors
      \[
         \e^{\frac{s}{p} + \frac{t}{p}} \leq \e^{\frac{s}{p}}+ \e^{\frac{t}{p}}
                                        \leq \frac{1}{p}\e^s + \frac{1}{q}\e^t
      \]
      C’est-à-dire
      \[
         F(x)G(x) \leq \frac{1}{p}F(x)^p + \frac{1}{q}G(x)^q.
      \]
      En intégrant et en utilisant \eqref{un}, on obtient
      \[
         \int_E FG \leq \frac{1}{p} + \frac{1}{q} = 1. 
      \]
      Il ne reste plus qu’à faire réapparaître $f$ et $g$, ce qui donne
      \[
         \int_E \frac{fg}{AB} \leq 1.         
      \]
      ce qui donne bien le résultat voulu.
\end{document}