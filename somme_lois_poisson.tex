\documentclass[fontsize=12pt,twoside=false,parskip=half]{scrartcl}
\usepackage[utf8]{inputenc}
\usepackage[french]{babel}
\usepackage[T1]{fontenc}
\usepackage{amsmath, amssymb, stmaryrd}
\usepackage[amsmath]{ntheorem}

\title{Somme de lois de Poisson}
\date{}
\author{}

\input{../outils_maths}
\setcounter{secnumdepth}{0}
\begin{document}
\maketitle
   \begin{Theoreme}
      Soient $X$ et $Y$ deux variables aléatoires indépendantes suivant des lois
      de Poisson de paramètres respectifs $\lambda$ et $\mu$. Alors
      $X + Y$ suit une loi de Poisson de paramètre $\lambda + \mu$.
   \end{Theoreme}
   \subsection{Démonstration}
      Notons $Z = X + Y$. $Z$ est à valeurs dans $\N$ et
      \[
         P(Z = k) = \sum_{i = 0}^k P(X = i, Y = k - i).
      \]
      $X$ et $Y$ étant indépendantes,       
      \begin{align*}
         P(Z = k) &= \sum_{i = 0}^k P(X = i)P(Y = k - i)\\
                  &= \sum_{i = 0}^k \e^{-\lambda} \frac{\lambda^i}{i!} \times
                                    \e^{-\mu} \frac{\mu^{k - i}}{(k - i)!}\\
                  &= \sum_{i = 0}^k \e^{-(\lambda + \mu)} 
                                    \frac{\lambda^i \mu^{k - i}}{i!(k - i)!}  
                                    \times \frac{k!}{k!}\\
                  &= \frac{\e^{-(\lambda + \mu)}}{k!} 
                     \sum_{i = 0}^k \binom{i}{k} \lambda^i \mu^{k - i}.
      \end{align*}
      La \emph{formule du binôme de Newton} permet de conclure,
      \[
         P(Z = k) = \frac{\e^{-(\lambda + \mu)} (\lambda + \mu)^k}{k!}
      \]
      On obtient bien que $X + Y$ suit une loi de Poisson de paramètre 
      $\lambda + \mu$.
      
      PS : ce résultat peut être obtenu plus rapidement en utilisant les fonction
      génératrices. En effet, la fonction génératrice d’une somme de variables 
      aléatoires indépendantes est le produit des fonctions génératrices et 
      la fonction génératrice d’une variable aléatoire $X$ suivant une loi de
      Poisson de paramètre $\lambda$ est $G_X(t) = \e^{\lambda(t - 1)}$.
\end{document}