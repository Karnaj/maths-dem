\documentclass[fontsize=12pt,twoside=false,parskip=half, french]{scrartcl}
\usepackage[utf8]{inputenc}
\usepackage{babel}
\usepackage[T1]{fontenc}
\usepackage{amsmath, amssymb, stmaryrd}
\usepackage[amsmath]{ntheorem}

\title{Nombre de Mersenne}
\date{}
\author{}

\input{../outils}
\setcounter{secnumdepth}{0}
\begin{document}
\maketitle
   \begin{Theoreme}
      Soit $(a, n) \in \N^2, a > 1, n > 1$. Si $a^n - 1$ est premier, alors $a = 2$
      et $n$ est premier.
   \end{Theoreme}
   \subsection{Démonstration}
      On a 
      \[
         a^n - 1 = (a - 1)(1 + a + \ldots + a^{n - 1}),
      \]
      donc puisque $a > 1$, alors $(a - 1)$ divise $a^n - 1$. Mais, $a^n - 1$
      est premier, donc $a = 2$.
      
      Notons maintenant $n = pq$ avec $p$ et $q$ deux entiers. On a
      \[
         a^n - 1 = 2^n - 1 = (2^p)^q - 1.
      \]
      Par suite, $2^q - 1$ divise $a^n -1$, d’où $q = 1$ ou $q = n$ puisque $a^{n - 1}$
      est premier. On en d"duit bien que $n$ est premier.
      
      PS : La réciproque est fausse. Par exemple, $2^11 - 1 = 2047 = 23 \times 49$.
\end{document}