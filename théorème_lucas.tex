\documentclass[fontsize=12pt,twoside=false,parskip=half, french]{scrartcl}
\usepackage[utf8]{inputenc}
\usepackage{babel}
\usepackage[T1]{fontenc}
\usepackage{amsmath, amssymb, stmaryrd}
\usepackage[amsmath]{ntheorem}

\title{Théorème de Lucas}
\date{}
\author{}

\input{../outils}
\setcounter{secnumdepth}{0}
\begin{document}
\maketitle
   \begin{Theoreme}[Théorème de Lucas]
      Soit $P$ un polynôme complexe de degré $n \geq 2$. Les racines de $P’$ sont
      dans l’enveloppe convexe de $P$.
   \end{Theoreme}
   \subsection{Démonstration}
   Pour commencer, on décompose $\frac{P’}{P}$. Posons $P = \lambda \prod_{k = 1}^{n} 
   (X - z_k)^{\alpha_k}$. On a  
   \begin{align*}
      P’ &= \lambda \sum_{j = 1}^n (X - z_1) \ldots (X - z_j)’ \ldots (X - z_n)\\
         &= \sum_{j = 1}^n \lambda \prod_{k \neq j} (X - z_k) \\
         &= \sum_{j = 1}^n \frac{P}{X - z_j}
   \end{align*}
   Montrons le résultat. Soit $z$ une racine de $P’$ qui n’est pas une racine de $P$. 
   \[
      \sum_{k = 1}^n \frac{1}{z - z_k} = \frac{P’(z)}{P(z)} = 0
   \]
   d’où
   \[
   \sum_{k = 1}^n \frac{z - z_k}{\abs{z - z_k}^2} = 0. 
   \]
   Et donc en posant 
   \[
      A_k = \frac{1}{\abs{z - z_k}^2} > 0
   \]
   on obtient
   \[
      z = \frac{\sum_{k = 1}^n A_k z_k}{\sum_{k = 1}^n A_k}.
   \]
  $z$ est donc un barycentre à coefficients réels positifs des racines de 
  $P$ et est donc dans l’enveloppe convexe des racines de $P$. Ceci étant 
  évidemment vrai si $z$ est aussi une racine de $P$, on obtient bien
  le résultat voulu.
\end{document}