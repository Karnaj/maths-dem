\documentclass[fontsize=12pt,twoside=false,parskip=half, french, french]{scrartcl}
\usepackage[utf8]{inputenc}
\usepackage{babel}
\usepackage[T1]{fontenc}
\usepackage{amsmath, amssymb, stmaryrd}
\usepackage[amsmath]{ntheorem}

\title{Théorème de Poisson}
\date{}
\author{}

\input{../outils}
\setcounter{secnumdepth}{0}
\begin{document}
\maketitle
   \begin{Theoreme}[Théorème de Poisson]
      Soit $\suite{p}$ telle que $np_n \to \lambda$ avec $\lambda \in \R_+$.
      Soit $X_n$ une suite de variable aléatoire telle que $X_n \sim 
      B(n, \frac{\lambda}{n})$. Alors $X_n \to X$ avec $X \sim P(\lambda)$.
   \end{Theoreme}
   \subsection{Démonstration}
   Soit $k$ fixé. On a pour $n > k$ 
   \[
      P(X_n = k) = \binom{n}{k} \frac{\lambda^k}{n^k}
                                \left( 1-\frac{\lambda}{n}\right)^{n - k}.
   \]
   On a
   \[
      \binom{n}{k} = \frac{n!}{(n - k)!k!} = \frac{n(n - 1)\ldots(n - k + 1)}{k!} \to \frac{n^k}{k!}.
   \]
   Et à côté de ça,
   \begin{align*}
      \left(1 - \frac{\lambda}{n}\right)^{n -  k} 
      &= \e^{(n - k)\ln \left(1 - \frac{\lambda}{n}\right)}\\
      &\sim \e^{(n - k)\frac{\lambda}{n}}\\
      &\to \e^{\lambda}
   \end{align*}
   car $k$ est fixé.
   
   Par suite,
   \[
      P(X_n = k) \to n^k \frac{\lambda^k}{n^k} \e^{\lambda} 
                 = \frac{\lambda^k}{k!}\e^{\lambda}.
   \]
   On obtient le résultat voulu.
\end{document}