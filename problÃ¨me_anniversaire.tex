\documentclass[fontsize=12pt,twoside=false,parskip=half, french]{scrartcl}
\usepackage[utf8]{inputenc}
\usepackage{babel}
\usepackage[T1]{fontenc}
\usepackage{amsmath, amssymb, stmaryrd}
\usepackage[amsmath]{ntheorem}

\title{Problème des anniversaires}
\date{}
\author{}

\input{../outils}
\setcounter{secnumdepth}{0}
\begin{document}
\maketitle
   \begin{Theoreme}
      Soit $n \in \N$. La probabilité que dans une population de $n$ personnes,
      trois d’entres-elles au moins partagent la même date d’anniersaire est
      \[
            w_n = 1 - \sum_{i = 0}^{\ent{\frac{n}{2}}} \frac{365!n!}{2^i i! (365 - n + i)! (n - 2i)! 365^n}
      \]
   \end{Theoreme}
   \subsection{Démonstration}
      Nous allons calculer la probabilité $P_i(n) $que parmi $n$ personnes,
         $i$ paires de personnes aient une date d’anniversaire commune et que les dates de autres soient toutes distinctes entres elles et de celles des paires. L’évènement « trois personnes sont nés le même jour » est le complémentaire de celui « pour tout $i$ entre $0$ et $\ent{\frac{n}{2}}$,
         il y a exactement $i$ paires de personnes nés le même jour, et tous les
         autres sont nés des jours différents » (c’est la moitié de $n$ car
         c’est le maximum de paires possibles). On a donc
         \[
            \omega_n = 1 - \sum_{i = 1}^{\ent{\frac{n}{2}}} P_i(n). 
         \]
         
         Soit $0 \leq i \leq \ent{\frac{n}{2}}$. Calculons $P_i(n)$.
         
         Pour commencer, il y a $\binom{365}{i}$ manières de choisir les anniversaires des différentes paires, puis $\binom{n}{2}$ manières de choisir la première
         paire, $\binom{n - 2}{2}$ pour la seconde, ..., et $\binom{n - 2(i - 1)}{2}$
         pour la dernière. 
         Ensuite, il nous faut choisir les dates d’anniversaire des $n - 2i$ personnes
         soit restantes, soit $A^{n - 2i}_{365 - i}$. Par suite,
         \begin{align*}
           P_i(n) &= \binom{365}{i} \times \prod_{l = 0}^{i - 1} \binom{n - 2l}{2} \times \frac{A^{n - 2i}_{365 - i}}{365^n}\\
           &=  \binom{365}{i} \times \prod_{l = 0}^{i - 1} \binom{n - 2l}{2} \times 
              \frac{(365 - i)!}{(365 - n + i)! \times 365^n}. 
         \end{align*}
         Et en fait, on se rend compte que, par une sorte de télescopage, 
         \[
            \prod_{l = 0}^{i - 1} \binom{n - 2l}{2} = \frac{n!}{2^i(n - 2i)!}.
         \]
         D’où
         \begin{align*}
            P_i(n) &= \binom{365}{i} \times \frac{n!}{2^i(n - 2i)!} \times 
                                        \frac{(365 - i)!}{(365 - n + i)! \times 365^n}\\
                &= \frac{365!n!}{2^i i! (365 - n + i)! (n - 2i)! 365^n}.
         \end{align*}
         Et on en déduit la probabilité qu’on cherche,
         \[
            w_n = 1 - \sum_{i = 0}^{\ent{\frac{n}{2}}} \frac{365!n!}{2^i i! (365 - n + i)! (n - 2i)! 365^n}
         \]
      
\end{document}