\documentclass[fontsize=12pt,twoside=false,parskip=half, french]{scrartcl}
\usepackage[utf8]{inputenc}
\usepackage{babel}
\usepackage[T1]{fontenc}
\usepackage{amsmath, amssymb, stmaryrd}
\usepackage[amsmath]{ntheorem}

\title{Sous-groupe de $(\Z, +)$}
\date{}
\author{}

\input{../outils}
\setcounter{secnumdepth}{0}
\begin{document}
\maketitle
   \begin{Theoreme}[Sous-groupe de $(\Z, +)$]
      Les sous-groupes de $(\Z, +)$ sont les $n\Z$ avec $n \in \N$ avec $n\Z = \enstq{kn \in \Z}{k \in \Z}$.
   \end{Theoreme}
   \subsection{Démonstration}
      Soit $n \in \N$, $n\Z$ est le sous-groupe engendré par $n$, donc est bien un sous-groupe.

      Soit maintenant $H$ un sous-groupe de $(\Z, +)$.
      
      Si $H = \ens{0}$, alors $H = 0\Z$.

      Sinon $H \neq \ens{0}$, et on pose $H_+ = \enstq{x \in H}{x > 0}$. 
      $H \neq \ens{0}$, donc il existe $x_0 \in H$, $x_0 \neq 0$. Alors, $x_0$ 
      ou $-x_0$ est dans $H_+$, d’où $H_+$ est une partie non vide de $\N$. Par 
      suite, $H_+$ admet un plus petit élément $n$. On a déjà $n\Z \subset H$. 
      Montrons l’inclusion inverse.
      
      Soit $x \in H$. Par division euclidienne, $x = qn + r$ avec $0 \leq r < n$.
      Alors $r = x - qn$, avec $x \in H$ et $qn \in H$ (puisque $n\Z \subset H$),
      donc $r \in H$. Si $r > 0$, alors $r \in H_+$, IMPOSSIBLE car $r < n = \min H_+$,
      donc $r = 0$ et $x = qn$, c’est-à-dire $H \subset n\Z$.
      
      La double inclusion donne bien que $H = n\Z$.
\end{document}