\documentclass[fontsize=12pt,twoside=false,parskip=half, french]{scrartcl}
\usepackage[utf8]{inputenc}
\usepackage{babel}
\usepackage[T1]{fontenc}
\usepackage{amsmath, amssymb, stmaryrd}
\usepackage[amsmath]{ntheorem}

\title{Polynômes de Legendre}
\date{}
\author{}

\input{../outils}
\setcounter{secnumdepth}{0}
\begin{document}
\maketitle
   On munit $E$ l’ensemble des fonctions continues de $\interv[-1; 1]$ dans $\R$ du produit scalaire
   \[
      \scalaire{f}{g} = \int_{-1}^1 f(t)g(t) \dt.
   \]
   On pose $U_n = (X^2 - 1)^n = (X - 1)^n(X + 1)^n$ et
   \[
      P_n = ((X^2 - 1)^n)^{(n)} = U_n^{(n)}.
   \]
   \begin{Theoreme}
      La famille $\suite{P}$ est orthonormale totale et $\forall f \in E$,
      \[
         f = \sum_{n = 0}^{+\infty} \frac{\scalaire{P_n}{f}}{\norme{P_n}^2}P_n
      \]
   \end{Theoreme}
   \subsection{Démonstration}
      $P_k$ est de degré $k$. En effet, $\deg U_k = 2k$ et on dérive $k$ fois.

      De plus, $P_k$ est orthogonal à $\R_{k - 1}[X]$. En effet, pour
      $Q \in \R_{k - 1}[X]$, des intégrations par parties successives donnent
      \[
         \scalaire{P_k}{Q} = (-1)\scalaire{U_k^{k - 1}}{Q’} = \ldots = (-1)^k\scalaire{U_k}{Q^{(n)}} = 0.
      \]
      Ces deux résultat donne l’orthonormalité de $(P_n)$. Puisque cette suite
      de polynômes est de degré étagés, c’est une base de $\R[X]$. Le
      \emph{théorème de Stone-Weirstrass} donne alors que c’est une famille totale.
      Ceci donne également l’égalité.

      PS : Le polynôme
         \[
            f_N = \sum_{n = 0}^{N} \frac{\scalaire{P_n}{f}}{\norme{P_n}^2}P_n
         \]de Legendre constitue la meilleur approximation euclidienne
           de $f$ parmi les polynômes de degré inférieur à $N$.

\end{document}
