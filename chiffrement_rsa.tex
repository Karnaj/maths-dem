\documentclass[fontsize=12pt,twoside=false,parskip=half, french]{scrartcl}
\usepackage[utf8]{inputenc}
\usepackage{babel}
\usepackage[T1]{fontenc}
\usepackage{amsmath, amssymb, stmaryrd}
\usepackage[amsmath]{ntheorem}

\title{Chiffrement RSA}
\date{}
\author{}

\input{../outils}
\setcounter{secnumdepth}{0}
\begin{document}
\maketitle 
   \begin{Theoreme}
       Soit $p$ et $q$ deux nombres premiers distincts et $n = pq$. Soient $c$ et
       $d$ deux entiers tels que $cd \equiv 1 \mod \phi(n)$, alors pour tout 
       $t \in \Z$, $t^{cd} \equiv t \mod n$.
   \end{Theoreme}
   \subsection{Démonstration}
      On a $\phi(n) = (p - 1)(q - 1)$ puisque $p$ et $q$ sont premiers distincts.
      Soit $k$ tel que $cd = 1 + k\phi(n)$ et soit $t \in \Z$. On veut montrer 
      $t^{cd} \equiv t \mod n$; il suffit de montrer que 
      \[
         t^{cd} \equiv t \mod p \text{ et } t^{cd} \equiv t \mod q.
      \]
      On aura alors que $p$ et $q$ divisent $t^{cd} - t$ alors $p \wedge q = 1$,
      d’où $n = pq$ divise $t^{cd - t}$ qui est bien le résultat voulu.
      
      Montrons que $t^{cd} \equiv t \mod p$, (le calcul est similaire pour $q$).
      \begin{itemize}
         \item Si $t \wedge p = 1$, alors on a (\emph{petit théorème de Fermat})
         $t^{p - 1} \equiv 1 \mod n$, et donc $t^{cd} \equiv (t^{p - 1})^{k(q - 1)}t \equiv t \mod p$
         \item Sinon $p$ divise $t$ et alors $t^{cd} \equiv t \equiv 0 \mod p$.
      \end{itemize}
      D’où le résultat.
      
      PS : ce résultat est utilisé pour le chiffrage RSA. Avec $g \colon t \mapsto t^c$
      la fonction de chiffrement et $f \colon t \mapsto t^d$ la fonction de déchiffrement
      dans $\varZquotient{n}$. En effet, on a $f \circ g$ qui est l’identité, ce 
      qui permet de chiffrer avec $g$ et de déchiffrer avec $f$. Le couple $(n, c)$
      est une clé publique et permet de chiffrer (elle est connue par tous), 
      $d$ est la clé privée (secrète).

      Sa sécurité repose sur le fait que trouver $d$ est compliqué. Il faudrait 
      factoriser $n$ pour trouver $p$ et $q$ ; on choisit donc $p$ et $q$ très grands
      et ils sont alors durs à trouver. 
\end{document}