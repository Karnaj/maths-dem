\documentclass[fontsize=12pt,twoside=false,parskip=half]{scrartcl}
\usepackage[utf8]{inputenc}
\usepackage[french]{babel}
\usepackage[T1]{fontenc}
\usepackage{amsmath, amssymb, stmaryrd}
\usepackage[amsmath]{ntheorem}

\title{Théorème de Lagrange}
\date{}
\author{}

\input{../outils_maths}
\setcounter{secnumdepth}{0}
\begin{document}
\maketitle
   \begin{Theoreme}[Théorème de Lagrange]
      Soit $G$ un groupe fini et soit $H$ un groupe de $G$.
      \[
         \card H \mid \card G.
      \]
   \end{Theoreme}
   \subsection{Démonstration}
      On considère la relation d’équivalence
      \[
         x \mathcal{R} y \iff xy^{-1} \in H. 
      \]
      Pour $x \in G$, la classe de $x$ est
      \[
         \cl(x) = xH = \enstq{x \times y}{y \in H}.
      \]
      En effet,
      \begin{align*}
         y \mathcal{R} x &\iff yx^{-1} \in H\\
                         &\iff y \in xH.
      \end{align*}
      
      Pour tout $x \in G$, l’application de $H$ dans $xH$ qui à $y$ associe $yx$ est bijective.
      Donc $\card xH = \card H$. Les classes ont donc $\card H$ éléments, et elles forment une 
      partition de $G$ d’où l’existence de $n$ tel que
      \[
         \card G = n \card H.
      \]  
\end{document}
