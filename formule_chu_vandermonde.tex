\documentclass[fontsize=12pt,twoside=false,parskip=half, french]{scrartcl}
\usepackage[utf8]{inputenc}
\usepackage{babel}
\usepackage[T1]{fontenc}
\usepackage{amsmath, amssymb, stmaryrd}
\usepackage[amsmath]{ntheorem}

\title{Formule de Chu-Vandermonde}
\date{}
\author{}

\input{../outils}
\setcounter{secnumdepth}{0}
\begin{document}
\maketitle
   \begin{Theoreme}[Formule de Chu-Vandermonde]
      Soit $(p, q, r) \in \N^3$, avec $r \leq p + q$. On a
      \[
         \binom{p + q}{r} = \sum_{k = 0}^r \binom{p}{k} \binom{q}{r - k}
      \]
   \end{Theoreme}
   % \subsection{Démonstration (numérique)}
      % On a $(1 + x)^{a + b} = $(1 + x)^a (1 + x)^b$. En utilisant la \emph{formule du binôme de Newton}, on obtient
      % \[
         % (1 + x)^{a + b} = \sum_{r = 0}^{a + b} \binom{r}{a + b} x^k
      % \]
      % et 
      % \begin{align*}
         % (1 + x)^a (1 + x)^b &= \sum_{i = 0}^a \binom{i}{a} x^i \sum_{j = 0}^{b} \binom{j}{b} x^j\\
                             % &= 
      % \end{align*}
   \subsection{Démonstration (ensembliste)}
      Soit $E$, un ensemble à $p + q$ éléments. Le nombre de parties à $r$ éléments de $E$ est
      \[
         \binom{p + q}{n}.
      \]
      Séparons $E$ en deux parties disjointes $F$ et $G$ de cardinaux $p$ et $q$. Faire une partie à $r$ 
      éléments de $E$, c’est choisir $k$ éléments dans $F$ (avec $k \in \intervalleEntier{0}{p}$) et compléter
      avec $r - k$ éléments de $F$. Par suite, le nombre de parties à $r$ éléments de $E$ est aussi
      \[
         \sum_{k = 0}^r \binom{p}{k} \binom{q}{r - k}.
      \]     
      PS : Ce résultat peut aussi être montré en développant les polynômes égaux
      $(1 + X)^{p + q}$ et $(1 + X)^p(1 + X)^q$ et en identifiant les coefficients.
\end{document}