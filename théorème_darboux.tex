\documentclass[fontsize=12pt,twoside=false,parskip=half, french]{scrartcl}
\usepackage[utf8]{inputenc}
\usepackage{babel}
\usepackage[T1]{fontenc}
\usepackage{amsmath, amssymb, stmaryrd}
\usepackage[amsmath]{ntheorem}

\title{Théorème de Darboux}
\date{}
\author{}

\input{../outils}
\setcounter{secnumdepth}{0}
\begin{document}
\maketitle
   \begin{Theoreme}[Théorème de Darboux]
      Soit $f$ une fonction dérivable sur un intervalle $\interv[a; b]$. Alors, $f’(\interv[a; b])$ est un intervalle.
   \end{Theoreme}
   \subsection{Démonstration}
      Si $f$ est de classe $C^1$ sur $\interv[a; b]$, alors le \emph{théroème des valeurs intermédiaires} donne le 
      résultat. Pour montrer le résultat dans le cas général on va introduire deux fonctions. 
      
      Pour $a < x \leq b$, on pose
      \[
         \phi(x) = \frac{f(x) - f(a)}{x - a}
      \]
      et pour $a \leq x < b$, on pose
      \[
         \psi(x) = \frac{f(x) - f(b)}{x - b}.
      \]
      On les prolonge par continuité en posant $\phi(a) = f’(a)$ et $\psi(b) = f’(b)$. Ces deux fonctions sont
      continues sur $\interv[a; b]$ et $\phi(b) = \psi(a)$ donc $\interv[\phi(a); \phi(b)] \cup \interv[\psi(a); \psi(b)]$
      est un segment qui contient $f’(a)$ et f’(b).
      
      Soit $y \in \interv[f’(a); f’(b)]$. On a $y \in \interv[\phi(a); \phi(b)]$ ou $y \in \interv[\psi(a); \psi(b)]$.
      
      Supposons $y \in \interv[\phi(a); \phi(b)]$. $\phi$ est continue, donc il existe $c \in \interv[a; b]$ tel que
      $y = \phi(c)$ (\emph{théorème des valeurs intermédiaires}). Si $c = a$, alors $y = f’(a)$. Sinon,
      \[
         y = \frac{f(c) - f(a)}{c - a}
      \]
      et le \emph{théorème des accroissements finis} donne alors l’existence de $d \in \interv]a; c[$ tel que 
      $f’(d) = y$.
      
      De même, on montre que si $y \in \interv[\psi(a); \psi(b)]$, il existe $d \in \interv]a; c[$ tel que 
      $f’(d) = y$.
      
      Ce résultat est vrai pour tout $y \in \interv[f’(a); f’(b)]$, c’est-à-dire que $f’$ prend toutes les valeurs 
      de cet intervalle.

      PS : Ce résultat montre que seules les fonctions vérifiant la propriété des valeurs intermédiaires peuvent être
      dérivées d’une autre fonction. Par exemple, la fonction partie entière ne peut pas l’être.
\end{document}