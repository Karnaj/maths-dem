\documentclass[fontsize=12pt,twoside=false,parskip=half]{scrartcl}
\usepackage[utf8]{inputenc}
\usepackage[francais]{babel}
\usepackage[T1]{fontenc}
\usepackage{amsmath, amssymb, stmaryrd}
\usepackage[amsmath]{ntheorem}

\title{Série harmonique}
\date{}
\author{}

\input{outils}
\setcounter{secnumdepth}{0}
\begin{document}
\maketitle
    Posons pour tout $n > 0$,
    \[
        H_n  = \sum_{k = 1}^{n} \frac{1}{k}
    \]
    les termes de la série harmonique.
   \begin{Theoreme}[Divergence de la série harmonique]
    La série harmonique diverge.
   \end{Theoreme}
   \subsection{Démonstration}
      Ce résultat se montre classiquement avec une \emph{comparaison série-intégrale}.
      Ici, nous allons le montrer différemment. 
      On a
      \[
        H_{2n} - H_n = \sum_{k = n + 1}^{2n} \frac{1}{k} \geq \frac{n}{2n} = \frac{1}{2}.
      \]
      d'où $H_n \to +\infty$. On peut montrer par récurrence que 
      \[    
        H_{2^n} \geq \frac{n}{2} H_1 = \frac{n}{2}
      \]
      d'où
      \[
        H_n \geq \frac{\ent{\log_2 n}}{2}.
      \]
      
      On a alors une démonstration élémentaire n'utilisant que la définition de limite.
      Une autre démonstration, assez proche de celle par \emph{comparaison série-intégrale}
      consiste à se ramener à une série télescopique en remarquant que
      \[
        \ln\left(n + 1\right) - \ln\left(n \right) = \ln\left(1 + \frac{1}{n}\right) \sim \frac{1}{n}.
      \]
      On conclut alors par \emph{commparaison de série à termes positifs}.
\end{document}