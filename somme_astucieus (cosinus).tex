\documentclass[fontsize=12pt,twoside=false,parskip=half, french]{scrartcl}
\usepackage[utf8]{inputenc}
\usepackage{babel}
\usepackage[T1]{fontenc}
\usepackage{amsmath, amssymb, stmaryrd}
\usepackage[amsmath]{ntheorem}

\title{Somme de cosinus}
\date{}
\author{}

\input{../outils}
\setcounter{secnumdepth}{0}
\begin{document}
\maketitle
   \begin{Theoreme}
      Soit un entier $n > 0$. Alors
      \begin{equation}
         S_n = \sum_{k = 0}^n \cos\left(\frac{k\pi}{n} \right) = 0.
      \end{equation}
   \end{Theoreme}
   \subsection{Démonstration}
       Nous pourrions calculer cette somme en passant par des complexes, mais
       nous allons agir autrement en montrant qu'elle est égale à son inverse.
       En effet, on a en faisant le changement d'indice $l = n - k$ que
       \begin{align*}
           S_n &= \sum_{l = 0}^{n} \cos\left( \frac{(l - n)\pi}{n} \right)\\
               &= \sum_{l = 0}^{n} \cos\left(\frac{l\pi}{n} - \pi \right)\\
               &= \sum_{l = 0}^{n} \cos\left(-\frac{l\pi}{n} \right)\\
               &= \sum_{l = 0}^{n} -\cos\left(\frac{l\pi}{n} \right)\\
               &= -S_n.
       \end{align*}
       D'où le résultat.
\end{document}