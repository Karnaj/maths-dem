\documentclass[fontsize=12pt,twoside=false,parskip=half, french]{scrartcl}
\usepackage[utf8]{inputenc}
\usepackage{babel}
\usepackage[T1]{fontenc}
\usepackage{amsmath, amssymb, stmaryrd}
\usepackage[amsmath]{ntheorem}

\title{Indicatrice d’Euler}
\date{}
\author{}

\input{../outils}
\setcounter{secnumdepth}{0}
\begin{document}
\maketitle
   \begin{Theoreme}
      Notons $\phi$ l’indicatrice d’Euler. Pour tout entier $n > 1$, on a
      \[
         n = \sum_{d \mid n} \phi(d).
      \]
   \end{Theoreme}
   \subsection{Démonstration}
      Considérons les $n$ fractions
      \[
         \frac{1}{n}, \frac{2}{n}, \ldots, \frac{n}{n}.
      \]
      L’écriture irréductible des ces fractions est de la forme $\frac{a}{d}$ avec $d$ qui divise
      $n$ et $a \wedge n = 1$. Pour chaque $d$ divisant $n$, il y a $\phi(d)$ numérateurs possibles.
      En tout, il y a $n$ fractions, d’où le résultat.
\end{document}
