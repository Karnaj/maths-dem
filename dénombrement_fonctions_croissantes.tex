\documentclass[fontsize=12pt,twoside=false,parskip=half, french]{scrartcl}
\usepackage[utf8]{inputenc}
\usepackage{babel}
\usepackage[T1]{fontenc}
\usepackage{amsmath, amssymb, stmaryrd}
\usepackage[amsmath]{ntheorem}

\title{Nombre de fonctions croissantes}
\date{}
\author{}

\input{../outils}
\setcounter{secnumdepth}{0}
\begin{document}
\maketitle
   \begin{Theoreme}
      Soit $p$ et $q$ deux entiers. Le nombre de fonctions croissantes de $\intervalleEntier{1}{k}$ dans 
      $\intervalleEntier{1}{n}$ est
      \[
         \binom{n + k - 1}{k}.
      \] 
   \end{Theoreme}
   \subsection{Démonstration}
      Commençons par dénombrer le nombre de fonctions strictement croissantes. Une application $f$ srictement
      croissante est entièrement définie par son image (les éléments sont ordonnés). En choisir une revient à
      choisir les $k$ éléments de son image. 
      % Soit $f$ une application strictement croissante de $\intervalleEntier{1}{k}$ dans $\intervalleEntier{1}{n}$. 
      % L’image de $f$ est une partie à $k$ éléments de $\intervalleEntier{1}{n}$. Montrons qu’il y a une bijection 
      % entre les fonctions strictement croissantes de $\intervalleEntier{1}{k}$ dans $\intervalleEntier{1}{n}$ et
      % les parties à $k$ éléments de $\intervalleEntier{1}{n}.$
      % \begin{description}
         % \item[Injectivité] : soit $A$ une partie à $k$ éléments de $\intervalleEntier{1}{n}$. Il existe une unique
                              % fonction strictement croissante de $\intervalleEntier{1}{k}$ dans $A$ (on ordonne $A$).
         % \item[Surjectivité] : si $A$ est associée à deux fonctions $f$ et $g$, alors, puisque ces deux fonctions 
                               % sont croissantes, elles sont égales (les deux consistent à ordonner $A$).
      % \end{description}
      Par suite, le nombre de fonctions strictement croissantes de $\intervalleEntier{1}{k}$ dans $\intervalleEntier{1}{n}$ est
      \[
         \binom{n}{k}.
      \]
   
      Soit maintenant $f$ une fonction croissante de $\intervalleEntier{1}{k}$ dans $\intervalleEntier{1}{n}$. Posons
      \[
         g(x) = f(x) + x - 1.
      \]
      La fonction $g$ est strictement croissante à valeurs dans $\intervalleEntier{1}{n + k - 1}$. De plus, toute 
      fonction strictement croissante de $\intervalleEntier{1}{k}$ dans $\intervalleEntier{1}{n + k - 1}$ est associé 
      à une unique fonction croissante de $\intervalleEntier{1}{k}$ dans $\intervalleEntier{1}{n}$ (les deux sont 
      entièrement définis par $g(x) = f(x) + x - 1$). Il y a donc une bijection entre les fonctions strictement 
      croissantes de $\intervalleEntier{1}{k}$ dans $\intervalleEntier{1}{n + k - 1}$ et les fonctions croissantes 
      de $\intervalleEntier{1}{k}$ dans $\intervalleEntier{1}{n}$. Par suite, le nombre de fonctions
      croissantes de $\intervalleEntier{1}{k}$ dans $\intervalleEntier{1}{n}$ est
      \[
         \binom{n + k - 1}{k}.
      \]    
\end{document}