\documentclass[fontsize=12pt,twoside=false,parskip=half, french]{scrartcl}
\usepackage[utf8]{inputenc}
\usepackage{babel}
\usepackage[T1]{fontenc}
\usepackage{amsmath, amssymb, stmaryrd}
\usepackage[amsmath]{ntheorem}

\title{Fonction Gamma d’Euler}
\date{}
\author{}

\input{../outils}
\setcounter{secnumdepth}{0}
\begin{document}
\maketitle
    On définit la fonction Gamma d’Euler en posant pour $x \in I = \R_+^*$.
    \[
      \Gamma(x) = \int_0^{+\infty} t^{x - 1} \e^{-t} \dt.
    \]
   \begin{Theoreme}
      La fonction $\Gamma$ est définie et continue sur $I$ et pour tout $x > 0$,
      \[
         \Gamma(x + 1) = x\Gamma(x).
      \]
   \end{Theoreme}
   \subsection{Démonstration}
      Posons $f(x, t) = t^{x - 1} \e^{-t}$. $f$ est définie sur $I^2$.
      \begin{itemize}
         \item Pour tout $x > 0$, $t \mapsto g(x,  t)$ est continue par morceaux sur $I$.
         \item Pour tout $t > 0$, $x \mapsto g(x,  t)$ est continue sur $I$.
         \item Soit $\interv[a; b] \in I$. Pour tout $x \in \interv[a; b]$, on a $t^{x - 1} \leq t^{a - 1} + t^{b - 1}$, d’où
            \[
               \abs{g(x, t)} \leq (t^{a - 1} + t^{b - 1})\e^{-t} = \phi(t)
            \]
            avec $\phi$ intégrable sur $I$ (continue sur $I$ et négligeable devant $t \mapsto \frac{1}{t^2}$ en $+\infty$). 
      \end{itemize}
      On peut alors appliquer le \emph{théorème de continuité d’une intégrale à paramètres}, ce qui nous permet d’obtenir la continuité de $\Gamma$ sur tout segment de $I$
      et donc sur $I$.
      
      On a 
      \[
         \Gamma(x + 1) = \int_0^{+\infty} t^{x} \e^{-t} \dt.
      \]
      En faisant une \emph{intégration par parties} avec $u(t) = t^x$ et $v(t) = -\e^{-t}$ (possible car $u$ et $v$ sont de classe $C^1$ et $uv$ a une limite nulle en $+\infty$), on obtient
      \[
         \Gamma(x + 1) = \left[-t^x\e^{-t}\right]_0^{+\infty} + \int_0^{+\infty} xt^{x} \e^{-t} \dt = x\Gamma(x)
      \]
      
      PS : une récurrence permet de montrer que pour tout entier $n > 0$, $\Gamma(n) = (n - 1)!$.
\end{document}