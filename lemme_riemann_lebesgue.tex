\documentclass[fontsize=12pt,twoside=false,parskip=half]{scrartcl}
\usepackage[utf8]{inputenc}
\usepackage[french]{babel}
\usepackage[T1]{fontenc}
\usepackage{amsmath, amssymb, stmaryrd}
\usepackage[amsmath]{ntheorem}

\title{Lemme de Riemann-Lebesgue}
\date{}
\author{}

\input{../outils_maths}
\setcounter{secnumdepth}{0}
\begin{document}
\maketitle
   \begin{Theoreme}[Lemme de Riemann-Lebesgue]
      Soit $f$ continue de $\interv[a; b]$ dans $\C$. On pose
      \[
         F_n = \int_a^b f(t)\e^{int} \dt.
      \]
      $F_n$ tend vers 0.
   \end{Theoreme}
   \subsection{Démonstration}
      Ici, nous allons démontrer le résultat pour une classe de fonction et l’étendre aux fonctions continues 
      grâce à un argument de densité. Si $f$ est de classe $C^1$, une \emph{intégration par parties} donne
      \[
         I_n = \left[f(t) \frac{e^{int}}{in}\right]_a^b - \frac{1}{in} \int_a^b f’(t)\e^{int} \dt.
      \]
      $f$ est bornée par une constante $M$ car continue sur un segment (\emph{théorème des bornes}), on obtient alors
      \[
         \abs{I_n} \leq \frac{1}{n} \left(2M + \int_a^b \abs{f’(t)} \dt\right) \to 0.
      \]
      
      Pour étendre le résultat à toutes les fonctions $f$ continues, nous utilisons, le
      \emph{théorème de Stone-Weierstrass} pour approcher uniformément $f$ par une suite de polynômes pour qui nous
      avons déjà montré le résultat (les polynômes sont dérivables).
\end{document}